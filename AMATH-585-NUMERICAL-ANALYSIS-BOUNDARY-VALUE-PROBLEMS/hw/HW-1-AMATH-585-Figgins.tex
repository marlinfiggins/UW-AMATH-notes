%Preamble
\documentclass[12pt]{article}
\usepackage{fancyhdr}
\usepackage{extramarks}
\usepackage{amsmath}
\usepackage{amssymb}
\usepackage{amsthm}
\usepackage{amsrefs}
\usepackage{amsfonts}
\usepackage{mathrsfs}
\usepackage{mathtools}
\usepackage[mathcal]{eucal} %% changes meaning of \mathcal
\usepackage{enumerate}
\usepackage[shortlabels]{enumitem}
\usepackage{verbatim} %% includes comment environment
\usepackage{hyperref}
\usepackage[capitalize]{cleveref}
\crefformat{equation}{~(#2#1#3)}
\usepackage{caption, subcaption}
\usepackage{graphicx}
\usepackage{fullpage} %%smaller margins
\usepackage[all,arc]{xy}
\usepackage{mathrsfs}

\hypersetup{
    linktoc=all,     % set to all if you want both sections and subsections linked
}

\topmargin=-0.45in
\evensidemargin=0in
\oddsidemargin=0in
\textwidth=6.5in
\textheight=9.0in
\headsep=0.25in
\setlength{\headheight}{16pt}

\linespread{1.0}

\pagestyle{fancy}
\lhead{\Name}
\chead{\hwClass: \hwTitle}
\rhead{\hwDueDate}
\lfoot{\lastxmark}
\cfoot{\thepage}

\renewcommand\headrulewidth{0.4pt}
\renewcommand\footrulewidth{0.4pt}

\setlength\parindent{0pt}

%% Title Info
\newcommand{\hwTitle}{HW \# 1}
\newcommand{\hwDueDate}{Jan 15, 2020}
\newcommand{\hwClass}{AMATH 585}
\newcommand{\hwClassTime}{}
\newcommand{\hwClassInstructor}{}
\newcommand{\Name}{\textbf{Marlin Figgins}}


%% MATH MACROS
\newcommand{\bbF}{\mathbb{F}}
\newcommand{\bbN}{\mathbb{N}}
\newcommand{\bbQ}{\mathbb{Q}}
\newcommand{\bbR}{\mathbb{R}}
\newcommand{\bbZ}{\mathbb{Z}}
\newcommand{\bbC}{\mathbb{C}}
\newcommand{\abs}[1]{ \left| #1 \right| }
\newcommand{\diff}[2]{\frac{d #1}{d #2}}
\newcommand{\infsum}[1]{\sum_{#1}^{\infty}}
\newcommand{\norm}[1]{ \left|\left| #1 \right|\right| }
\newcommand{\eval}[1]{ \left. #1 \right| }
\newcommand{\Expect}[1]{\mathbb{E}\left[#1 \right]}
\newcommand{\Var}[1]{\mathbb{V}\left[#1 \right]}
\renewcommand{\vec}[1]{\mathbf{#1}}

\renewcommand{\phi}{\varphi}
\renewcommand{\emptyset}{\O}

%--------Theorem Environments--------
%theoremstyle{plain} --- defaultx
\newtheorem{thm}{Theorem}[section]
\newtheorem{cor}[thm]{Corollary}
\newtheorem{prop}[thm]{Proposition}
\newtheorem{lem}[thm]{Lemma}
\newtheorem{conj}[thm]{Conjecture}
\newtheorem{quest}[thm]{Question}

\theoremstyle{definition}
\newtheorem{defn}[thm]{Definition}
\newtheorem{defns}[thm]{Definitions}
\newtheorem{con}[thm]{Construction}
\newtheorem{exmp}[thm]{Example}
\newtheorem{exmps}[thm]{Examples}
\newtheorem{notn}[thm]{Notation}
\newtheorem{notns}[thm]{Notations}
\newtheorem{addm}[thm]{Addendum}

% Environments for answers and solutions
\newtheorem{exer}{Exercise}
\newtheorem{sol}{Solution}

\theoremstyle{remark}
\newtheorem{rem}[thm]{Remark}
\newtheorem{rems}[thm]{Remarks}
\newtheorem{warn}[thm]{Warning}
\newtheorem{sch}[thm]{Scholium}

\makeatletter
\let\c@equation\c@thm
\makeatother

\begin{document}

\begin{exer}
Use MATLAB to evaluate the second order accurate approximation 
\[
u'' (x) \approx \frac{u(x+h) + u(x-h) - 2 u(x)}{h^2}
\]
for $u(x) = \sin x$ and $x = \pi / 6$.  Try $h = 10^{-1} , 10^{-2} , \ldots ,
10^{-16}$, and make a table of values of $h$, the computed finite difference
quotient, and the error.  Explain your results.
\end{exer}

\begin{sol}\leavevmode
    The table of values can be found in the second part of the document with my code, but the major realization is that the error appears to decrease up until $h = 10^{-4}$ after which the error increases. For values of $h$ smaller than $h = 10^{-7}$ the approximation essentially breaks and is no where close to the true value. This is likely due to rounding error which depend on both $h$ and our choice of $x = \pi / 6$ and how close they are to machine $\epsilon$. Regardless, we see that our error in minimized somewhere near $10^{-4}$.
\end{sol}

\newpage

\begin{exer}
  Use the formula in the previous exercise with $h=0.2$, $h=0.1$, and $h=0.05$ 
to approximate $u'' (x)$, where $u(x) = \sin x$ and $x = \pi / 6$.  Use one 
step of Richardson extrapolation, combining the results from $h=0.2$ and $h=0.1$, 
to obtain a higher order accurate approximation.  Do the same with the results 
from $h=0.1$ and $h=0.05$.  Finally do a second step of Richardson extrapolation, 
combining the two previously extrapolated values, to obtain a still higher 
order accurate approximation.  Make a table of the computed results and their 
errors.  What do you think is the order of accuracy after one step of Richardson 
extrapolation?  How about after two?
\end{exer}

\begin{sol}
    Both tables are also in the code section of the document. My naive guess is that the order after one step of Richardson extrapolation is that the order is around $O(h^3)$ since the original centered difference approximation we've used is of order $O(h^2)$. I'd also guess naively that after two steps of Richardson extrapolation we have order $O(h^4)$. 
\end{sol}

\newpage

\begin{exer}
    Using Taylor series, derive the error term for the approximation
\[
u' (x) \approx \frac{1}{2h} [ -3 u(x) + 4 u(x+h) - u(x+2h) ] .
\]
\end{exer}

\begin{sol}
    We begin by writing out the Talor approximations for $u(x+h)$ and $u(x + 2h)$
    \begin{align*}
        u(x + h) &= u(x) + u'(x)h + u''(x) \frac{h^2}{2!} + u'''(x) \frac{h^{3}}{3!} + u''''(x) \frac{h^{4}}{4!} + O(h^{5})\\
        u(x +2h) &= u(x) +  2u'(x)h + 4 u''(x) \frac{h^2}{2!} + 8 u'''(x) \frac{h^{3}}{3!} + 16 u''''(x) \frac{h^{4}}{4!} + O(h^{5})\\
    \end{align*}
    From here, it follows that 
    \begin{align*}
        -3 u(x) + 4 u(x + h) - u(x + 2h) = 2u'(x)h - 4 u'''(x) \frac{h^{3}}{3!} + O(h^{4}),
    \end{align*}
    so that 
    \begin{align*}
        \frac{ -3 u(x) + 4 u(x + h) - u(x + 2h)}{2h} &= u'(x) + 2 u'''(x) \frac{h^{2}}{3!} + O(h^{3})\\
                                                     &= u'(x) + O(h^{2}).
    \end{align*}
\end{sol}

\newpage

\begin{exer}
    Consider a forward difference approximation for the second derivative
of the form
\[
u'' (x) \approx A u(x) + B u(x+h) + C u(x+2h) .
\]
Use Taylor's theorem to determine the coefficients $A$, $B$, and $C$
that give the maximal order of accuracy and determine what this order is.
\end{exer}

\begin{sol}
    Using the Taylor approximations of $u(x + h)$ and $u(x + 2h)$ from Exercise 2 as a template, we see that
\begin{align*}
    A u(x) + B u(x+h) + C u(x+2h) &= (A + B + C) u(x) + (B + 2C) u'(x) h \\
                                  &+ (B / 2 + 4 C  / 2) u''(x) h^{2}\\
                                  &+ (B / 6 + 8 C / 6) u'''(x)h^{3} + O(h^{4})
\end{align*}

This reduces to a system of equations which ideally satisfy
\begin{align*}
A + B + C = 0\\
B + 2C = 0 \\
B / 2 + 2C = 1 
\end{align*}
and additionally if possible $ B / 6 + 4 C / 3 = 0$. Starting with the second equation, we observe that $B = - 2C$. Plugging this into the third equation, we have that $-C + 2C = 1$, so that $C = 1$ and $B = -2$. By the first equation, it follows $A = - B - C$, so $A = 1$. This leaves us with solution 
\begin{equation*}
A = 1 , B = -2, C = 1.
\end{equation*}
We can additionally test the equation for the coefficients of the $O(h^{3})$ term and see
\begin{equation*}
B / 6 + 4 C / 3 = -1 / 3 + 4 / 3 = 1 \neq 0.
\end{equation*}
Therefore, we have that 

\begin{align*}
    \frac{u(x) -2 u(x + h) + u(x + 2h)}{h^{2}} = u''(x) + O(h).
\end{align*}
\end{sol}

\newpage

\begin{exer}
Consider the two-point boundary value problem
\[ 
u'' + 2xu' - x^2 u = x^2 ,~~~u(0)=1,~~~u(1) = 0 .
\]
Let $h=1/4$ and explicitly write out the difference equations,
using centered differences for all derivatives.
\end{exer}

\begin{sol}
    We'll begin by using the centered differences for both derivatives, so that
    \begin{align*}
        u'(x) &\approx \frac{u(x+h) - u(x-h)}{2h}\\
        u''(x) &\approx \frac{u(x+h) - 2 u(x) + u(x-h)}{h^{2}},
    \end{align*}

    This allows us to write that 
    \begin{align*}
        \frac{u(x+h) - 2 u(x) + u(x-h)}{h^{2}} + x \left(\frac{u(x+h) - u(x-h)}{h} \right) - x^{2}u(x) = x^{2}
    \end{align*}

    Re-arranging this equation, we see that
\begin{align*}
    \left( \frac{1}{h^{2}} - \frac{x}{h}\right) u(x-h) - \left(x^{2} + \frac{2}{h^{2}}\right)u(x) + \left(\frac{1}{h^{2}} + \frac{x}{h}\right) u(x+h) = x^{2}.
\end{align*}

We can now explicitly discretize this problem using step size $h = 1 / 4$ with grid points $x_0 = 0$, $x_1 = 1 / 4$, $x_2 = 1 / 2$, $x_3 = 3 / 4$, $x_4 = 1$. Starting with $x_{1}$, this reduces to
\begin{align*}
    (16 - 4 x_{1}) u(x_{0}) - (x_{1}^{2} + 32) u(x_{1}) + (16 + 4 x_{1}) u(x_{2}) &= x_{1}^{2}\\
                    \implies \quad \quad \quad \\
                             - (x_{1}^{2} + 32) u(x_{1}) + (16 + 4 x_{1}) u(x_{2}) &= x_{1}^{2} + (16 - 4x_{1}),
\end{align*}
where we've used that $u(x_{0}) = u(0) = 1$. The equations for $x_{2} = 1 / 2$ are given by
\begin{align*}
    (16 - 4 x_{2}) u(x_{1}) - (x_{2}^{2} + 32) u(x_{2}) + (16 + 4 x_{2}) u(x_{3}) &= x_{2}^{2}\\
\end{align*}
Now moving onto the $x_{3} = 1$, we have that

\begin{align*}
    (16 - 4 x_{3}) u(x_{2}) - (x_{3}^{2} + 32) u(x_{3}) + (16 + 4 x_{3}) u(x_{4}) &= x_{3}^{2}\\
                    \implies \quad \quad \quad \\
    (16 - 4 x_{3}) u(x_{2}) - (x_{3}^{2} + 32) u(x_{3})  &= x_{3}^{2}\\
\end{align*}
where we've used that $u(x_{4}) = u(1) = 0$. This leaves us a system of three equations with three unknowns $u(x_{1})$ $u(x_{2})$, and $u(x_{3})$. I've neglected to plug in the values of the $x_{i}$ for simplicity.
\end{sol}
\newpage

\begin{exer}
    A rod of length 1 meter has a heat source applied to it and it eventually
reaches a steady-state where the temperature is not changing.  The
conductivity of the rod is a function of position $x$ and is given by
$c(x) = 1 + x^2$.  The left end of the rod is held at a constant
temperature of 1 degree.  The right end of the rod is insulated so
that no heat flows in or out from that end of the rod.  This problem is
described by the boundary value problem:
\[
\frac{d}{dx} \left( (1 + x^2 ) \frac{du}{dx} \right) = f(x) ,~~~ 
0 \leq x \leq 1 ,
\]
\[
u(0) = 1,~~~u'(1) = 0 .
\]
\begin{description}
\item[(a)] Write down a set of difference equations for this problem.
Be sure to show how you do the differencing at the endpoints.
[Note:  It is better {\bf not} to rewrite 
$\frac{d}{dx} ( ( 1+ x^2 ) \frac{du}{dx} )$ as $(1 + x^2 ) u'' (x) + 2x u'(x)$;
leave the equation in the form above.]
\item[(b)] Write a MATLAB code to solve the difference equations.
You can test your code on a problem where you know the solution
by choosing a function $u(x)$ that satisfies the boundary conditions
and determining what $f(x)$ must be in order for $u(x)$ to solve the
problem.  Try $u(x) = (1-x )^2$.  Then $f(x) = 2( 3 x^2 - 2 x + 1 )$.
\item[(c)] Try several different values for the mesh size $h$.  Based
on your results, what would you say is the order of accuracy of your
method?
\end{description}
\end{exer}

\begin{sol}
    (a) In what follows, we'll assume that we take a mesh of the interval $[0,1]$ of size $h$ with grid points $x_{j} = j h $ with $x_{M} = 1$. We also set $ 1 + x^{2} = p(x)$. We begin by approximating the outside derivative at a grid point $x_{j}$ as 
    \begin{equation*}
        \frac{d}{dx} \left(p(x) \frac{du}{dx} \right) \approx \frac{p(x_{j+ \frac{1}{2}})  \frac{du}{dx}(x_{j + \frac{1}{2}}) - p(x_{j - \frac{1}{2}})  \frac{du}{dx}(x_{j - \frac{1}{2}}) }{h}.
    \end{equation*}
    We can then estimate $\frac{du}{dx}$ using a centered difference approximation, so that
    \begin{align*}
        \frac{du}{dx}(x_{j - \frac{1}{2}}) &\approx \frac{u(x_{j}) - u(x_{j-1})}{h}\\
        \frac{du}{dx}(x_{j + \frac{1}{2}}) &\approx \frac{u(x_{j+1}) - u(x_{j})}{h}.
    \end{align*}

    Combining these expressions, we get the approximation
    \begin{align*}
        \frac{ p(x_{j + \frac{1}{2}}) [u(x_{j+1}) - u(x_{j})] - p(x_{j - \frac{1}{2}})[u(x_{j}) - u(x_{j-1})] }{h^{2}} = f(x_{j}).
    \end{align*}
    In order to accommodate for our boundary conditions, we'll make some adjustments to equations for the first and last end points. Accounting for the left boundary condition i.e. that $u(0) = 1$, we change the first equation i.e. when $j = 1$ so that

    \begin{align*}
      \frac{ p(x_{1 + \frac{1}{2}}) [u(x_{2}) - u(x_{1})] - p(x_{\frac{1}{2}})[u(x_{1}) - u(0)] }{h^{2}} = f(x_{1}).
    \end{align*}
    This shows that our first equation is modified so that
    \begin{align*}
        \frac{ p(x_{1 + \frac{1}{2}}) [u(x_{2}) - u(x_{1})] - p(x_{\frac{1}{2}})u(x_{1}) }{h^{2}} = f(x_{1}) - \frac{p(x_{\frac{1}{2}})}{h^{2}}
    \end{align*}
    since $u(0) = 1$. In order to accommodate the Neumann boundary condition at the right end point, we will add a second order backward difference approximation of $u'(1) = 0$ to our current equations
    \begin{align*}
        \frac{3 u(1) - 4 u(1-h) + u(1 - 2h) )}{2h} = 0.
    \end{align*}

    Writing this out as a system of equations where $\vec{U} = [ u(x_{1}), u(x_{2}), ..., u(x_{M}) = u(1) ]^{T}$ and $\vec{F} = [f(x_{1}) - p(x_{\frac{1}{2}}) / h^{2}, f(x_{2}), \ldots, f(x_{M-1}), 0 ]$ with matrix

    \begin{align*}
        \vec{A} = \frac{1}{h^{2}} 
        \begin{pmatrix}
            a_{1} & b_{1} & 0 &   0 & \cdots \\ 
            b_{1} & a_{2} & b_{2} & 0 & \cdots\\
            0     & b_{2} & a_{3} &  b_{3} & &   & \\
            \vdots  & & \ddots & \ddots & \ddots\\
                    &&& b_{M-2} & a_{M-1} & b_{M-1}    \\
                    &&& h/ 2 & -2h & 3h / 2    
        \end{pmatrix},
    \end{align*}
    where the coefficients $a_{j}$ and $b_{j}$ are given by $a_{j} = - p(x_{j + \frac{1}{2}}) - p(x_{j-\frac{1}{2}})$ and $b_{j} = p(x_{j + \frac{1}{2}})$. Finding our solution now reduces to finding the solution of the system of equations

    \begin{align*}
    \vec{A} \vec{U} = \vec{F}.
    \end{align*}

    (b) All code implementing this is found at the end of the document.

    (c) Halving the input $h$ appears to lead to a 4-fold drop in the error, so I would estimate that this method has error on the order of $O(h^2)$ (according to the infinity norm) which is consistent with the choice to use second order approximations in the derivation of the equations above.
\end{sol}


\end{document}
