%Preamble
\documentclass[12pt]{article}
\usepackage{fancyhdr}
\usepackage{extramarks}
\usepackage{amsmath}
\usepackage{amssymb}
\usepackage{amsthm}
\usepackage{amsrefs}
\usepackage{amsfonts}
\usepackage{mathrsfs}
\usepackage{mathtools}
\usepackage[mathcal]{eucal} %% changes meaning of \mathcal
\usepackage{enumerate}
\usepackage[shortlabels]{enumitem}
\usepackage{verbatim} %% includes comment environment
\usepackage{hyperref}
\usepackage[capitalize]{cleveref}
\crefformat{equation}{~(#2#1#3)}
\usepackage{caption, subcaption}
\usepackage{graphicx}
\usepackage{fullpage} %%smaller margins
\usepackage[all,arc]{xy}
\usepackage{mathrsfs}

\hypersetup{
    linktoc=all,     % set to all if you want both sections and subsections linked
}

\topmargin=-0.45in
\evensidemargin=0in
\oddsidemargin=0in
\textwidth=6.5in
\textheight=9.0in
\headsep=0.25in
\setlength{\headheight}{16pt}

\linespread{1.0}

\pagestyle{fancy}
\lhead{\Name}
\chead{\hwClass: \hwTitle}
\rhead{\hwDueDate}
\lfoot{\lastxmark}
\cfoot{\thepage}

\renewcommand\headrulewidth{0.4pt}
\renewcommand\footrulewidth{0.4pt}

\setlength\parindent{0pt}

%% Title Info
\newcommand{\hwTitle}{HW \# 4}
\newcommand{\hwDueDate}{November 16, 2020}
\newcommand{\hwClass}{AMATH 584}
\newcommand{\hwClassTime}{}
\newcommand{\hwClassInstructor}{}
\newcommand{\Name}{\textbf{Marlin Figgins}}


%% MATH MACROS
\newcommand{\bbF}{\mathbb{F}}
\newcommand{\bbN}{\mathbb{N}}
\newcommand{\bbQ}{\mathbb{Q}}
\newcommand{\bbR}{\mathbb{R}}
\newcommand{\bbZ}{\mathbb{Z}}
\newcommand{\bbC}{\mathbb{C}}
\newcommand{\abs}[1]{ \left| #1 \right| }
\newcommand{\diff}[2]{\frac{d #1}{d #2}}
\newcommand{\infsum}[1]{\sum_{#1}^{\infty}}
\newcommand{\norm}[1]{ \left|\left| #1 \right|\right| }
\newcommand{\eval}[1]{ \left. #1 \right| }
\newcommand{\Expect}[1]{\mathbb{E}\left[#1 \right]}
\newcommand{\Var}[1]{\mathbb{V}\left[#1 \right]}
\renewcommand{\vec}[1]{\mathbf{#1}}

\renewcommand{\phi}{\varphi}
\renewcommand{\emptyset}{\O}

%--------Theorem Environments--------
%theoremstyle{plain} --- defaultx
\newtheorem{thm}{Theorem}[section]
\newtheorem{cor}[thm]{Corollary}
\newtheorem{prop}[thm]{Proposition}
\newtheorem{lem}[thm]{Lemma}
\newtheorem{conj}[thm]{Conjecture}
\newtheorem{quest}[thm]{Question}

\theoremstyle{definition}
\newtheorem{defn}[thm]{Definition}
\newtheorem{defns}[thm]{Definitions}
\newtheorem{con}[thm]{Construction}
\newtheorem{exmp}[thm]{Example}
\newtheorem{exmps}[thm]{Examples}
\newtheorem{notn}[thm]{Notation}
\newtheorem{notns}[thm]{Notations}
\newtheorem{addm}[thm]{Addendum}

% Environments for answers and solutions
\newtheorem{exer}{Exercise}
\newtheorem{sol}{Solution}

\theoremstyle{remark}
\newtheorem{rem}[thm]{Remark}
\newtheorem{rems}[thm]{Remarks}
\newtheorem{warn}[thm]{Warning}
\newtheorem{sch}[thm]{Scholium}

\makeatletter
\let\c@equation\c@thm
\makeatother

\begin{document}
\begin{exer}
    (a) Consider the matrix 

    \begin{equation*} 
    \vec{A} = \begin{pmatrix}
        2 & 0 & 0 \\
        0 & 2 & 0 \\
        0 & 0 & 2
    \end{pmatrix}.
    \end{equation*}

Determine its eigenvalues and eigenvectors and the algebraic and geometric multiplicity of each.

    (b) Consider the matrix 

    \begin{equation*} 
    \vec{B} = \begin{pmatrix}
        2 & 1 & 0 \\
        0 & 2 & 1 \\
        0 & 0 & 2
    \end{pmatrix}.
    \end{equation*}
Determine its eigenvalues and eigenvectors and the algebraic and geometric multiplicity of each.
\end{exer}

\begin{sol}\leavevmode

    (a) We can compute that the characteristic polynomial is given by $p(\lambda) = (2 - \lambda)^3$ since  $\vec{A}-\lambda \vec{I}$ is a diagonal matrix and its determinant is the product of its diagonal entries. Therefore, its eigenvalues are simply $\lambda_1 = \lambda_2 = \lambda_3 = 2$ and the algebraic multiplicity of this eigenvalue is 3. We can see that the corresponding eigenvectors are given by the euclidean basis vectors $\vec{e}_1, \vec{e}_2, \vec{e}_3$ since the matrix $\vec{A} = 2\vec{I}$. Using the fact that $\vec{A} - 2 \vec{I} = 0$, we see that the nullspace has dimension 3, so the eigenvalue $\lambda=2$ has geometric multiplicity 3.

        (b) Similarly, we can compute the characteristic polynomial as $p(\lambda) = (2-\lambda)^3$ since  $\vec{B}-\lambda \vec{I}$ is an upper triangular matrix and its determinant is the product of its diagonal entries. Therefore, its eigenvalues are simply $\lambda_1 = \lambda_2 = \lambda_3 = 2$ and the algebraic multiplicity is 3. We can compute the eigenvector as $\vec{v}_1 = \vec{e}_1$. Further, we can compute the geometric multiplicity by looking at 
\begin{equation*}
    \vec{B} - 2 \vec{I} = 
    \begin{pmatrix}
        0 & 1 & 0 \\
        0 & 0 & 1 \\
        0 & 0 & 0
    \end{pmatrix}.
\end{equation*}
We can see that this matrix has rank 2, so the geometric multiplicity of the eigenvalue $\lambda = 2$ is $3-2 = 1$.
\end{sol}

\newpage

\begin{exer}
    For each of the following statements, prove that it is true or give a counter example to show it is false. Here each matrix $A\in \bbC^{m\times m}$ unless otherwise indicated.

    \begin{enumerate}[(a)]
        \item If $\lambda$ is an eigenvalue of $\vec{A}$ and $\mu\in\bbC$, then  $\lambda-\mu$ is an eigenvalue of $ \vec{A}- \mu \vec{I}$.
        \item If $\vec{A}$ is real and $\lambda$ is an eigenvalue of  $\vec{A}$, then so is $-\lambda$.
        \item If $\vec{A}$ is real and $\lambda$ is an eigenvalue of $\vec{A}$, then so is $\overline{\lambda}$.
        \item If $\lambda$ is an eigenvalue of $\vec{A}$ and $\vec{A}$ is nonsingular, then $\lambda^{-1}$ is an eigenvalue of $\vec{A}^{-1}$.
        \item If all the eigenvalues of $\vec{A}$ are zero, then $\vec{A}=0$.
        \item If $\vec{A}$ is Hermitian and $\lambda$ is an eigenvalue of  $\vec{A}$, then $\abs{\lambda}$ is a singular value of $\vec{A}$.
        \item If $\vec{A}$ is diagonalizable and all its eigenvalues are equal, then $\vec{A}$ is diagonal.
\end{enumerate}
\end{exer}

\begin{sol}\leavevmode

(a) Suppose that $\vec{v}$ is the eigenvector corresponding to $\lambda$, then we have that 
    \begin{align*}
        (\vec{A}-\mu \vec{I}) \vec{v} &= \vec{A}\vec{v} - \mu \vec{v}\\
                                      &= \lambda \vec{v} - \mu \vec{v} \\
                                      &= (\lambda-\mu) \vec{v}.
    \end{align*}
    Therefore, $\lambda-\mu$ is an eigenvalue of $\vec{A}-\mu \vec{I}$.

(b) This is false. Consider the matrix $\vec{A} = 2 \vec{I}$. This matrix's eigenvalue is 2 as can be seen by its diagonal entries, but -2 is not an eigenvalue.

(c)  Suppose that $\vec{v}$ is the eigenvector corresponding to $\lambda$, then we have that $\vec{A} \vec{v} = \lambda \vec{v}$. Taking the complex conjugate of this equation, we see that
\begin{equation*}
\overline{ \vec{A} \vec{v} } = \vec{A} \overline{ \vec{v} } = \overline{\lambda} \overline{\vec{v}},
\end{equation*}
where we have used that $\vec{A}$ is real. This shows that $\overline{\lambda}$ is an eigenvalue of $\vec{A}$.
%TODO: Do I need to massage this to use the adjoint instead?

(d)  Suppose that $\vec{v}$ is the eigenvector corresponding to $\lambda$, then we have that $\vec{A} \vec{v} = \lambda \vec{v}$. Multiplying by $\vec{A}^{-1}$, we see that
\begin{equation*}
    \vec{A}^{-1} \vec{A} \vec{v} = \lambda \vec{A}^{-1} \vec{v} \implies \vec{A}^{-1} \vec{v} = \lambda^{-1}\vec{v}.
\end{equation*}
Notice, we assume that the eigenvalue $\lambda\neq 0$. This is because the matrix $\vec{A}$ is non-singular and has non-zero determinant and therefore, cannot have 0 eigenvalues.

(e) This is not true. Consider the matrix 
\begin{equation*}
    \vec{W} = \begin{pmatrix}
        0 & 27 \\
        0 & 0
    \end{pmatrix}.
\end{equation*}
This matrix is upper triangular and therefore, its eigenvalues are its diagonals which are 0.

(f) We know that the non-zero singular values are given by the square roots of the eigenvalues of $\vec{A}^* \vec{A} = \vec{A}^2$ where we've used that $\vec{A}$ is Hermitian. Therefore, the singular values are simply the square roots of the eigenvalues of $\vec{A}^2$ as shown in HW 2. Using the fact that $\vec{A} \vec{v} = \lambda \vec{v}$ for eigenvalue $\lambda$ and corresponding eigenvector $\vec{v}$, we have that 
\begin{equation}
    \vec{A}^2 \vec{v} = \lambda \vec{A} \vec{v} = \lambda^2 \vec{v},
\end{equation}
so $\lambda^2$ is an eigenvalue of $\vec{A}^2$. We see then that the singular values of $\vec{A}$ are just $\sigma = \sqrt{\lambda^2} = \abs{\lambda}$. 

(g) Suppose that the matrix is diagonalizable so that $\vec{A} = \vec{P} \vec{\Lambda} \vec{P}^{-1}$. In the case that the all the eigenvalues are the same, we have that $\vec{\Lambda} = \lambda \vec{I}$. Therefore, we have that 
\begin{equation*}
\vec{A} = \lambda \vec{P} \vec{I} \vec{P}^{-1} = \lambda \vec{I},
\end{equation*}
so that $\vec{A}$ is diagonal.

\end{sol}
\newpage

\begin{exer}
    Let $\vec{A}\in \bbC^{m\times m}$ be tridiagonal and Hermitian, with all its sub and super diagonal entries non-zero. Prove that the eigenvalues of $\vec{A}$ are distinct. (Hint: Show that for any $\lambda\in\bbC$,  $\vec{A} -\lambda \vec{I}$ has rank at least $m-1$)
\end{exer}

\begin{sol}
We begin by writing the matrix $\vec{A}$ in terms of a block diagonal matrix
\begin{equation*}
\vec{A} 
= 
\begin{pmatrix}
    \vec{a} & 0 \\
    \vec{C} & \vec{d},
\end{pmatrix}
\end{equation*}
where $\vec{a}, \vec{d}$ are vectors in $\bbC^{m-1}$ and  $\vec{C}$ is an upper triangular matrix of size $(m-1)\times (m-1)$. Notice that the matrix $\vec{C}$ must be upper triangular and its diagonal entries are all non-zero under our assumption that all sub-diagonal entries of $\vec{A}$ must be non-zero. It follows then that the matrix $\vec{C}$ is non-singular. From this, we know that $\text{rank}(\vec{A} - \lambda \vec{I}) \geq m-1$ for all $\lambda\in\bbC$. In the case that $\lambda$ is an eigenvalue of $\vec{A}$, this means that the geometric multiplicity of $\lambda$ must be exactly one. Since $\vec{A}$ is Hermitian, we know that it is diagonalizable, and therefore, non-defective i.e. the geometric multiplicity of its eigenvalues must be equal to their algebraic multiplicity. Since these multiplicities are 1, each of the eigenvalues of $\vec{A}$ must be distinct.
\end{sol}
\end{document}
