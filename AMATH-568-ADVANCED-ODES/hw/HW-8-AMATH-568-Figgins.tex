%Preamble
\documentclass[12pt]{article}
\usepackage{fancyhdr}
\usepackage{extramarks}
\usepackage{amsmath}
\usepackage{amssymb}
\usepackage{amsthm}
\usepackage{amsrefs}
\usepackage{amsfonts}
\usepackage{mathrsfs}
\usepackage{mathtools}
\usepackage[mathcal]{eucal} %% changes meaning of \mathcal
\usepackage{enumerate}
\usepackage[shortlabels]{enumitem}
\usepackage{verbatim} %% includes comment environment
\usepackage{hyperref}
\usepackage[capitalize]{cleveref}
\crefformat{equation}{~(#2#1#3)}
\usepackage{caption, subcaption}
\usepackage{graphicx}
\usepackage{fullpage} %%smaller margins
\usepackage[all,arc]{xy}
\usepackage{mathrsfs}

\hypersetup{
    linktoc=all,     % set to all if you want both sections and subsections linked
}

\topmargin=-0.45in
\evensidemargin=0in
\oddsidemargin=0in
\textwidth=6.5in
\textheight=9.0in
\headsep=0.25in
\setlength{\headheight}{16pt}

\linespread{1.0}

\pagestyle{fancy}
\lhead{\Name}
\chead{\hwClass: \hwTitle}
\rhead{\hwDueDate}
\lfoot{\lastxmark}
\cfoot{\thepage}

\renewcommand\headrulewidth{0.4pt}
\renewcommand\footrulewidth{0.4pt}

\setlength\parindent{0pt}

%% Title Info
\newcommand{\hwTitle}{HW \# 8}
\newcommand{\hwDueDate}{March 19, 2021}
\newcommand{\hwClass}{AMATH 568}
\newcommand{\hwClassTime}{}
\newcommand{\hwClassInstructor}{}
\newcommand{\Name}{\textbf{Marlin Figgins}}


%% MATH MACROS
\newcommand{\bbF}{\mathbb{F}}
\newcommand{\bbN}{\mathbb{N}}
\newcommand{\bbQ}{\mathbb{Q}}
\newcommand{\bbR}{\mathbb{R}}
\newcommand{\bbZ}{\mathbb{Z}}
\newcommand{\bbC}{\mathbb{C}}
\newcommand{\abs}[1]{ \left| #1 \right| }
\newcommand{\diff}[2]{\frac{d #1}{d #2}}
\newcommand{\infsum}[1]{\sum_{#1}^{\infty}}
\newcommand{\norm}[1]{ \left|\left| #1 \right|\right| }
\newcommand{\eval}[1]{ \left. #1 \right| }
\newcommand{\Expect}[1]{\mathbb{E}\left[#1 \right]}
\newcommand{\Var}[1]{\mathbb{V}\left[#1 \right]}
\renewcommand{\vec}[1]{\mathbf{#1}}

\renewcommand{\phi}{\varphi}
\renewcommand{\emptyset}{\O}

%--------Theorem Environments--------
%theoremstyle{plain} --- defaultx
\newtheorem{thm}{Theorem}[section]
\newtheorem{cor}[thm]{Corollary}
\newtheorem{prop}[thm]{Proposition}
\newtheorem{lem}[thm]{Lemma}
\newtheorem{conj}[thm]{Conjecture}
\newtheorem{quest}[thm]{Question}

\theoremstyle{definition}
\newtheorem{defn}[thm]{Definition}
\newtheorem{defns}[thm]{Definitions}
\newtheorem{con}[thm]{Construction}
\newtheorem{exmp}[thm]{Example}
\newtheorem{exmps}[thm]{Examples}
\newtheorem{notn}[thm]{Notation}
\newtheorem{notns}[thm]{Notations}
\newtheorem{addm}[thm]{Addendum}

% Environments for answers and solutions
\newtheorem{exer}{Exercise}
\newtheorem{sol}{Solution}

\theoremstyle{remark}
\newtheorem{rem}[thm]{Remark}
\newtheorem{rems}[thm]{Remarks}
\newtheorem{warn}[thm]{Warning}
\newtheorem{sch}[thm]{Scholium}

\makeatletter
\let\c@equation\c@thm
\makeatother

\begin{document}

\begin{exer}
    Consider the Optical Parametric Oscillator as given in Lecture 23 of the notes (pg 99-102).

    \begin{enumerate}[(a)]
        \item Assuming slow time $\tau = \epsilon^{2} t$ and slow space $\xi = \epsilon x$, derive the Fisher-Kolmogorov equation for the slow evolution of the instability (the expression after Eq. (518)).
        \item Derive the Swift-Hohenberg type expression which is given by Eq. (519) with the scalings detailed in the notes.
    \end{enumerate}
\end{exer}

\begin{sol}
    Disclaimer: I don't want to type up all the algebra. I will skip a lot of steps but ultimately I should get to where I'm trying to go!

    (a) We start by considering the governing equation
    \begin{align*}
        U_{t} &= \frac{i}{2} U_{xx} + V U^{*} - (1 + i \Delta_{1}) U\\
        V_{t} &= \frac{i}{2} \rho V_{xx} - U^{2} - (\alpha + i \Delta_{2}) V + S.
    \end{align*}

    We can find a steady state to this problem as 
    \begin{align*}
    U = 0. V = \frac{S}{\alpha + i \Delta_{2}},
    \end{align*}
    for which we require that 
\begin{equation*}
    \abs{S} < \abs{S_{c}} = \abs{(\alpha + i \Delta_{2})(1 + i \Delta_{1})}
\end{equation*}
    for stability of this this solution. We'll describe the onset of instability of this equation using the given slow time and slow space expansions. We then write $U$ in terms of an expansion about its steady state
     \begin{align*}
         U &= 0 + \epsilon u(\tau, \xi)\\
         V &= \frac{S}{\alpha + i \Delta_{2}} + \epsilon^{2} v(\tau, \xi).
    \end{align*}
    Plugging this eqution into the governing equation gives a bunch of algebra,
    \begin{align*}
        \epsilon^{2} u_{\tau} &= \epsilon^{2} \frac{i}{2} u_{\xi \xi} + \left(\frac{S}{\alpha + i \Delta_{2}} + \epsilon^{2} v \right)  u^{*}  - (1 + i \Delta_{1}) u(\tau, \xi) + O(\epsilon^{4})\\
        \epsilon^{2} v_{\tau} &= \epsilon^{2} \frac{i}{2} \rho v_{xx} - u - (\alpha + i \Delta_{2}) ( \frac{S \epsilon^{-2}}{\alpha + i \Delta_{2}}  + v ) + O(\epsilon^{4})
    \end{align*}
    Rearranging terms and using the expansion $S = S_{c} + \epsilon^{2} C + \epsilon^{3} C_{1} + \cdots$, we have eqns (516),
    \begin{align*}
        v &= - \frac{u^{2}}{\alpha + i \Delta_{2}} + \frac{\epsilon^{2}}{\alpha + i \Delta_{2}} \left[ \frac{i}{2} \rho v_{\xi \xi}  - v_{\tau} \right] + O(\epsilon^{4})\\
        u^{*} &= u - \frac{\epsilon^{2}}{1 + i \Delta_{1}} \left[ \frac{i}{2} u_{\xi\xi} - u_{\tau} + v u^{*} + \frac{C}{\alpha + i \Delta_{2}}  u^{*} \right] + O(\epsilon^{4})
    \end{align*}
    Muliplying these two equations gives an equation of the form
    \begin{align*}
        v u^{*} = f(vu^{*})
    \end{align*}
    which can be solved iteratively to find
    \begin{align*}
        v u^{*} = - \frac{1}{\alpha + i \Delta_{2}} \abs{u}^{2} u + \frac{\epsilon^{2}}{(\alpha + i \Delta_{2})^{2}} \left( u (u^{2})_{\tau} - \frac{i}{2} \rho u (u^{2})_{\xi\xi}  \right) + O(\epsilon^{2}).
    \end{align*}
    This allows us to solve for the forcing of the system as 
\begin{align*}
    R = \epsilon^{2} \left[  \frac{i}{2} u_{\xi \xi} - u_{\tau} - \frac{\abs{u}^{2} u}{\alpha + i \Delta_{2}} + \frac{C}{\alpha + i \Delta_{2}}\right] + O(\epsilon^{4})
\end{align*}
    Using Fredholm's alternative theorem, we know that $R$ must be orthogonal to the null space of the adjoint corresponding with the leading order behavior. This allows us to derive a solvability conidiotn in terms of the  $R$ above
    \begin{equation*}
        (1 - i \Delta_{1}) R + (1 + i \Delta_{1}) R^{*} = 0.
    \end{equation*}
    This solvability condition can give several different equations depending on the retained scales. Keeping the scales $\tau, \xi$, gives us the equation (after algebra)
     \begin{align*}
         \phi_{\tau} - \phi_{\zeta\zeta} \pm \phi^{3} \pm \abs{C}(1 + \Delta_{1}^{2}) / S_{c},\\
         u = \phi \sqrt{ \frac{\alpha^{2} + \Delta_{2}^{2}}{\Delta_{1} \Delta_{2} - \alpha} }, 
         \quad \xi = \zeta \sqrt{\frac{\Delta_{1}}{2}}.
    \end{align*}
This is the desried Fisher-Komogorov equation.

\newpage

(b) To get to the Swift-Hohenberg type expression, we'll need to back track a bit to find the $O(\epsilon^{4})$ terms in our expression for $u$, so that we have scales scales  $\xi$, $X = \epsilon^{2} c$. $\tau$ and  $T = \epsilon^{4} t$. This will give updated $R$ as
\begin{align*}
    R &= \epsilon^{2} \left[  \frac{i}{2} u_{\xi \xi} - u_{\tau} - \frac{\abs{u}^{2} u}{\alpha + i \Delta_{2}} + \frac{C}{\alpha + i \Delta_{2}}\right]\\
      &+ \epsilon^{4} \left[ \frac{i}{2} u_{XX} - u_{T} + \frac{1}{(\alpha + i \Delta_{2})^{2}} 
      \left( u(u^{2})_{\tau} - \frac{i}{2} \rho u (u^{2})_{\xi\xi} \right)\right] + O(\epsilon^{6}).
\end{align*}
In this case, we retain the scales $\xi$ and $T$ and apply the same solvibility condition with $\Delta_{1} = \epsilon^{2} \kappa$ and $\alpha - \Delta_{1} \Delta_{2} = \epsilon^{2} \beta$.  Setting more variables $\omega = 2 \kappa \abs{\Delta_{2}}$, $\sigma = -2 \beta$, and  $\gamma = \omega^{2} / 4 + 2 C \beta + C^{2}$, we can get another equation in terms of $\phi$ 
\begin{align*}
    \phi_{t} + \frac{1}{4} \left(\partial_{\zeta}^{2} - \omega \right)^{2} \phi - \gamma \phi - \sigma \phi^{3} + \phi^{5} \pm 3 \phi \cdot (\phi_{\zeta})^{2} \pm 2 \phi^{2} \phi_{\zeta \zeta} = 0.
\end{align*}
I'm not showing much of any work here because I am lazy and got stuck several times. Please be merciful!
\end{sol}
\end{document}
