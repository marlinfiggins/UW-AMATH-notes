%Preamble
\documentclass[12pt]{article}
\usepackage{fancyhdr}
\usepackage{extramarks}
\usepackage{amsmath}
\usepackage{amssymb}
\usepackage{amsthm}
\usepackage{amsrefs}
\usepackage{amsfonts}
\usepackage{mathrsfs}
\usepackage{mathtools}
\usepackage[mathcal]{eucal} %% changes meaning of \mathcal
\usepackage{enumerate}
\usepackage[shortlabels]{enumitem}
\usepackage{verbatim} %% includes comment environment
\usepackage{hyperref}
\usepackage[capitalize]{cleveref}
\crefformat{equation}{~(#2#1#3)}
\usepackage{caption, subcaption}
\usepackage{graphicx}
\usepackage{fullpage} %%smaller margins
\usepackage[all,arc]{xy}
\usepackage{mathrsfs}

\hypersetup{
    linktoc=all,     % set to all if you want both sections and subsections linked
}

\topmargin=-0.45in
\evensidemargin=0in
\oddsidemargin=0in
\textwidth=6.5in
\textheight=9.0in
\headsep=0.25in
\setlength{\headheight}{16pt}

\linespread{1.0}

\pagestyle{fancy}
\lhead{\Name}
\chead{\hwClass: \hwTitle}
\rhead{\hwDueDate}
\lfoot{\lastxmark}
\cfoot{\thepage}

\renewcommand\headrulewidth{0.4pt}
\renewcommand\footrulewidth{0.4pt}

\setlength\parindent{0pt}

%% Title Info
\newcommand{\hwTitle}{HW \# 4}
\newcommand{\hwDueDate}{Feb 12, 2020}
\newcommand{\hwClass}{AMATH 568}
\newcommand{\hwClassTime}{}
\newcommand{\hwClassInstructor}{}
\newcommand{\Name}{\textbf{Marlin Figgins}}


%% MATH MACROS
\newcommand{\bbF}{\mathbb{F}}
\newcommand{\bbN}{\mathbb{N}}
\newcommand{\bbQ}{\mathbb{Q}}
\newcommand{\bbR}{\mathbb{R}}
\newcommand{\bbZ}{\mathbb{Z}}
\newcommand{\bbC}{\mathbb{C}}
\newcommand{\abs}[1]{ \left| #1 \right| }
\newcommand{\diff}[2]{\frac{d #1}{d #2}}
\newcommand{\infsum}[1]{\sum_{#1}^{\infty}}
\newcommand{\norm}[1]{ \left|\left| #1 \right|\right| }
\newcommand{\eval}[1]{ \left. #1 \right| }
\newcommand{\Expect}[1]{\mathbb{E}\left[#1 \right]}
\newcommand{\Var}[1]{\mathbb{V}\left[#1 \right]}
\renewcommand{\vec}[1]{\mathbf{#1}}

\renewcommand{\phi}{\varphi}
\renewcommand{\emptyset}{\O}

%--------Theorem Environments--------
%theoremstyle{plain} --- defaultx
\newtheorem{thm}{Theorem}[section]
\newtheorem{cor}[thm]{Corollary}
\newtheorem{prop}[thm]{Proposition}
\newtheorem{lem}[thm]{Lemma}
\newtheorem{conj}[thm]{Conjecture}
\newtheorem{quest}[thm]{Question}

\theoremstyle{definition}
\newtheorem{defn}[thm]{Definition}
\newtheorem{defns}[thm]{Definitions}
\newtheorem{con}[thm]{Construction}
\newtheorem{exmp}[thm]{Example}
\newtheorem{exmps}[thm]{Examples}
\newtheorem{notn}[thm]{Notation}
\newtheorem{notns}[thm]{Notations}
\newtheorem{addm}[thm]{Addendum}

% Environments for answers and solutions
\newtheorem{exer}{Exercise}
\newtheorem{sol}{Solution}

\theoremstyle{remark}
\newtheorem{rem}[thm]{Remark}
\newtheorem{rems}[thm]{Remarks}
\newtheorem{warn}[thm]{Warning}
\newtheorem{sch}[thm]{Scholium}

\makeatletter
\let\c@equation\c@thm
\makeatother

\begin{document}

\begin{exer}

Consider the weakly nonlinear oscillator:

\begin{equation*}
    \frac{d^{2} u}{dt^{2}} + u + \epsilon u^{5} = 0
\end{equation*}
with $u(0) = 0$ and $u'(0) = A > 0$ and with $0 < \epsilon \ll 1$.
\begin{enumerate}[(a)]
    \item  Use a regular perturbation expansion and calculate the first two terms.
    \item Determine at what time the approximation of part (a) fails to hold.
    \item Use a Poincare-Lindstedt expansion and determine the first two terms and frequency corrections. 
    \item For $\epsilon = 0.1$ plot the numerical solution (from MATLAB), the regular expansion solution, and the Poincare-Lindstedt solution for $0 \leq t \leq 20$.
\end{enumerate}
\end{exer}

\begin{sol}
    (a) We begin by writing 
    \begin{equation*}
        u(t) = u_{0}(t) + \epsilon u_{1}(t) + \cdots.
    \end{equation*}
    Plugging this subsitution in the differential equation, we can gather terms of the same order. For example, we have that
    \begin{align*}
        O(1): \quad \quad u_{0 tt} + u_{0} = 0,
    \end{align*}
    since all $u^{5}$ terms are of order $\epsilon$ or higher. We can get the first and second order coeffiecient as
    \begin{align*}
        O(\epsilon):& \quad \quad u_{1tt} + u_{1} = - u_{0}^{5} \\
        O(\epsilon^{2}):& \quad \quad u_{2tt} + u_{2} = - 5 u_{1} u_{0}^{4}. 
    \end{align*}
    We've used the binomial theorem to find the coefficients in the expansion 
\begin{equation*}
    \epsilon u^{5} = \epsilon \left( \sum_{n=0}^{\infty} \epsilon^{n} u_{n} \right)^{5} = \epsilon u_{0}^{5} + \epsilon^{2} u_{1} u_{0}^{4} + \cdots.
\end{equation*}
Starting with the $O(1)$ equation, we see that with the initial conditions $u_{0}(0) = 0$ and $u_{0}'(0) = A > 0$, we have that
\begin{align*}
    u_{0}(t) = A \sin(t).
\end{align*}
This allows us to simplify the first order equation as
\begin{align*}
    u_{1tt} + u_{1} = - A^{5} \sin^{5}(t) &= -\frac{A^{5}}{16} (\sin(5t) - 5\sin(3t) + 10 \sin(t))\\
    u_{1}(0) = 0, &\quad  u_{1}'(0) = 0.
\end{align*}
This equation has solution
\begin{equation*}
    u_{1}(t) = \frac{A^{5}}{384}\left( - 80\sin(t) - 15 \sin(3t) + \sin(5t) + 120 t \cos(t) \right)
\end{equation*}

(b) Given that our approximation is written as
\begin{align*}
    u(t) \approx u_{0}(t) + \epsilon u_{1}(t),
\end{align*}
we see this approximation will grow unbounded as $t \approx \epsilon^{-1}$ due to growth term given by $\frac{120}{384} A^{5} t \cos(t)$ above.

(c) We'll now use a Poincare-Lindstedt expansion. This is using the similar expansion of $u$ but we addditionally change variables to $\tau = \omega t$ so that
\begin{align*}
    \omega &= \omega_{0} + \epsilon \omega_{1} + \epsilon^{2}  \omega_{2} + \cdots\\
    u(\tau) &= u_{0}(\tau) + \epsilon u_{1}(\tau) + \epsilon^{2} u_{2}(\tau) \\
           &\omega^{2} u_{\tau \tau}(\tau) + u(\tau) + \epsilon u(\tau)^{5} = 0.
\end{align*}
This gives leading order equation
\begin{align*}
    \omega_{0}^{2} u_{0\tau \tau} + u_{0} = 0\\
    u_{0}(0) = 0, \quad u_{0}'(0)=0.
\end{align*}
Picking $w_{0} = 1$ for simplicity, we see that this has leading order solution
\begin{align*}
    u_{0} = A \sin(\tau).
\end{align*}
We can now find the first order approximation
\begin{align*}
   u_{1 \tau \tau } + u_{1} = - u_{0}^{5} -   2 \omega_{1} u_{0\tau \tau}.
\end{align*}

This gives us a solution
\begin{equation*}
    u_{1}(\tau) = \frac{A^{5}}{384}\left( - 80\sin(\tau) - 15 \sin(3\tau) + \sin(5\tau) + 120 \tau \cos(\tau) \right) + A \omega_{1} (\tau \cos(\tau) - \sin(\tau)).
\end{equation*}
To remove the secular growth term, we  solve
\begin{align*}
    \left( \frac{120}{384} A^{5} + A\omega_{1}\right) \tau \cos(\tau) = 0.\\
    \omega_{1} = -\frac{120}{384} A^{4}.
\end{align*}


(d) Plotting attached in appendix.

\end{sol}

\newpage

\begin{exer}
Consider Rayleigh’s equation:
\begin{align*}
    \frac{d^{2} u}{dt^{2}} + u + \epsilon \left[ -\frac{du}{dt}  + \frac{1}{3} \left( \frac{du}{dt} \right)^{3}\right] = 0
\end{align*}
which has only one periodic solution called a “limit cycle” $0 < \epsilon \ll 1$. Given
\begin{equation*}
    u(0) = 0 \quad \text{ and } \quad \frac{d u}{dt}(0) = A.
\end{equation*}

\begin{enumerate}[(a)]
    \item Use a multiple scale expansion to calculate the leading order behavior.
    \item Use a Poincare-Lindsted expansion and an expansion of $A = A_0 + \epsilon A_1 + \ldots$ to calculate the leading-order solution and the first non-trivial frequency shift for the limit cycle.
    \item For $\epsilon = 0.01, 0.1, 0.2$ and $0.3$, plot the numerical solution and the multiple scale expansion for $0 \leq t \leq 40$ and for various values of $A$ for your multiple scale solution. Also plot the limit cycle solution calculated from part (b)
    \item Calculate the error
\begin{equation*}
    E(t) = \abs{ y_{\text{numerical}}(t) - y_{\text{approx}}(t) }
\end{equation*}
as a function of time ($0 \leq t \leq 40$) using $\epsilon = 0.01,0.1,0.2$ and $0.3$.
\end{enumerate}
\end{exer}

\begin{sol}
    (a) Adding an independent scaled time variable $\tau = \epsilon t$ to our differential equation, we get that our leading order term is given by
    \begin{align*}
        u_{0tt} + u_{0} = 0\\
        u_{0}(0,0) = 0, \quad u_{0t}(0,0) = A. 
    \end{align*}
    Solving this, gives the leading order behavior as 
    \begin{align*}
        u_{0}(t, \tau) &= A(\tau) \sin(t) + B(\tau) \cos(t)\\
        A(0) = A, & \quad B(0) = 0.
    \end{align*}
    Writing out the $O(\epsilon)$ terms of the expansion, we can derive formulas for $A(\tau)$ and  $B(\tau)$. Writing $\rho(\tau) = 4 A^{2} / ( A^{2} + (4 - A^{2}) \exp(- \tau) )$ as in (268) in the lecture notes, we have
    \begin{align*}
        A(\tau) &=\frac{2A}{\sqrt{A^{2} + (4 - A^{2}) \exp(- \tau) }}\\
        B(\tau) &= 0.
    \end{align*}

    (b) Now using a PL, we'll expand 
    \begin{align*}
       \omega &= \omega_{0} + \epsilon \omega_{1} + \epsilon^{2}  \omega_{2} + \cdots\\
    u(\tau) &= u_{0}(\tau) + \epsilon u_{1}(\tau) + \epsilon^{2} u_{2}(\tau) \\
    A &= A_{0} + \epsilon A_{1} + \epsilon^{2} A_{2}\\
           &\omega^{2} u_{\tau \tau} + u - \epsilon \omega u_{\tau}+ \epsilon \omega^{3}u_{\tau}^{3} / 3 = 0
    \end{align*}
    This allows us to write the leading order solution as
    \begin{align*}
        &u_{0\tau \tau} + u_{0} = 0,\\
        &u_{0}(0,0) = 0, u_{0t}(0,0) = A_{0}
    \end{align*}
    where once again I've set $\omega_{0}$ = 1. This has solution
    \begin{align*}
        u_{0} = A_{0} \sin(\tau) .
    \end{align*}
    Now for the $O(\epsilon)$ term,
    \begin{align*}
    u_{1 \tau \tau} + u_{1} =  u_{0\tau} - u_{0\tau}^{3} / 3  - 2 \omega_{1} u_{0\tau \tau}= 0,\\
    u_{1}(0,0) = 0, \quad u_{1 \tau}(0,0) = A_{1}.
    \end{align*}
    Plugging in our previous solution, we can simplify the baove equation as
    \begin{align*}
        u_{1 \tau \tau} + u_{1} &= A_{0} \cos(\tau) - A_{0}^{3} \cos^{3}(\tau) / 3 + 2 \omega_{1} \sin(\tau)\\
                                &= \left(A_{0} - \frac{A_{0}^{3}}{4}\right ) \cos(\tau) + 2 \omega_{1} \sin(\tau) - \frac{A_{0}^{3}}{12} \cos(3x).
\end{align*}
In order to rid ourselves of the secular growth here (caused by the $\sin(\tau)$ and $\cos(\tau)$), we require that
    \begin{align*}
    A_{0} = 2, \quad \omega_{1} = 0.
    \end{align*}
    This leaves us with 
    \begin{align*}
        u_{1 \tau \tau} + u_{1} = - \frac{2}{3} \left(\cos(3\tau) \right) \\
            u_{1}(0,0) = 0, \quad u_{1 \tau}(0,0) = A_{1}.
    \end{align*}
    This gives solution
\begin{equation*}
    u_{1}(\tau) = A_{1} \sin( \tau ) + \frac{2}{3} \cos(\tau)\cos(3\tau) - \frac{2}{3} \cos(3 \tau)
\end{equation*}
    Going one more level up, we can solve for the $O(\epsilon^{2})$ term as
    \begin{align*}
        2 \omega_{2} u_{0 \tau \tau} &+ u_{2 \tau \tau} + u_{2} - u_{1 \tau} + u_{0 \tau}^{2} u_{1 \tau} = 0.
    \end{align*}
    which we simplify to 
    \begin{align*}
        u_{2 \tau \tau} + u_{2} &= u_{1 \tau} - 2 \omega_{2} u_{0 \tau \tau} - u_{0 \tau}^{2} u_{1 \tau}\\
                                &= - 2 A_{1} \cos( \tau ) - A_{1} \cos(3 \tau) + \frac{1}{4} \sin(\tau) + 4  \omega_{2} \sin( \tau ) + \frac{1}{6} \sin(3 \tau) + \frac{1}{4} \sin(5 \tau)    
    \end{align*}
    Once again, canceling secular growth terms requires
    \begin{align*}
    -2 A_{1} = 0 \quad  \frac{1}{4} = 4 \omega_{2} = 0\\
    A_{1} = 0, \quad \omega_{2} = -\frac{1}{16}.
    \end{align*}

    (c, d) Plots attached in appendix.
\end{sol}
\end{document}
