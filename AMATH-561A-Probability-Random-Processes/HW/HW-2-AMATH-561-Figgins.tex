%Preamble
\documentclass[12pt]{article}
\usepackage{fancyhdr}
\usepackage{extramarks}
\usepackage{amsmath}
\usepackage{amssymb}
\usepackage{amsthm}
\usepackage{amsrefs}
\usepackage{amsfonts}
\usepackage{mathrsfs}
\usepackage{mathtools}
\usepackage[mathcal]{eucal} %% changes meaning of \mathcal
\usepackage{enumerate}
\usepackage[shortlabels]{enumitem}
\usepackage{verbatim} %% includes comment environment
\usepackage{hyperref}
\usepackage[capitalize]{cleveref}
\crefformat{equation}{~(#2#1#3)}
\usepackage{caption, subcaption}
\usepackage{graphicx}
\usepackage{fullpage} %%smaller margins
\usepackage[all,arc]{xy}
\usepackage{mathrsfs}

\hypersetup{
    linktoc=all,     % set to all if you want both sections and subsections linked
}

\topmargin=-0.45in
\evensidemargin=0in
\oddsidemargin=0in
\textwidth=6.5in
\textheight=9.0in
\headsep=0.25in
\setlength{\headheight}{16pt}

\linespread{1.1}

\pagestyle{fancy}
\lhead{\Name}
\chead{\hwTitle}
\rhead{\hwClass}
\lfoot{\lastxmark}
\cfoot{\thepage}

\renewcommand\headrulewidth{0.4pt}
\renewcommand\footrulewidth{0.4pt}

\setlength\parindent{0pt}

%% Title Info
\newcommand{\hwTitle}{HW \# 2}
\newcommand{\hwDueDate}{October 19, 2020}
\newcommand{\hwClass}{AMATH 561A}
\newcommand{\hwClassTime}{}
\newcommand{\hwClassInstructor}{}
\newcommand{\Name}{\textbf{Marlin Figgins}}


%% MATH MACROS
\newcommand{\bbF}{\mathbb{F}}
\newcommand{\bbN}{\mathbb{N}}
\newcommand{\bbQ}{\mathbb{Q}}
\newcommand{\bbR}{\mathbb{R}}
\newcommand{\bbZ}{\mathbb{Z}}
\newcommand{\bbC}{\mathbb{C}}
\newcommand{\Prob}{\mathbb{P}}
\newcommand{\calF}{\mathcal{F}}
\newcommand{\abs}[1]{ \left| #1 \right| }
\newcommand{\diff}[2]{\frac{d #1}{d #2}}
\newcommand{\infsum}[1]{\sum_{#1}^{\infty}}
\newcommand{\norm}[1]{ \left|\left| #1 \right|\right| }
\newcommand{\eval}[1]{ \left. #1 \right| }
\newcommand{\Expect}[1]{\mathbb{E}\left[#1 \right]}
\newcommand{\Var}[1]{\mathbb{V}\left[#1 \right]}
\renewcommand{\phi}{\varphi}
\renewcommand{\emptyset}{\O}

%--------Theorem Environments--------
%theoremstyle{plain} --- defaultx
\newtheorem{thm}{Theorem}[section]
\newtheorem{cor}[thm]{Corollary}
\newtheorem{prop}[thm]{Proposition}
\newtheorem{lem}[thm]{Lemma}
\newtheorem{conj}[thm]{Conjecture}
\newtheorem{quest}[thm]{Question}

\theoremstyle{definition}
\newtheorem{defn}[thm]{Definition}
\newtheorem{defns}[thm]{Definitions}
\newtheorem{con}[thm]{Construction}
\newtheorem{exmp}[thm]{Example}
\newtheorem{exmps}[thm]{Examples}
\newtheorem{notn}[thm]{Notation}
\newtheorem{notns}[thm]{Notations}
\newtheorem{addm}[thm]{Addendum}

% Environments for answers and solutions
\newtheorem{exer}{Exercise}
\newtheorem{sol}{Solution}

\theoremstyle{remark}
\newtheorem{rem}[thm]{Remark}
\newtheorem{rems}[thm]{Remarks}
\newtheorem{warn}[thm]{Warning}
\newtheorem{sch}[thm]{Scholium}

\makeatletter
\let\c@equation\c@thm
\makeatother

\begin{document}
\begin{exer}
    Suppose that $X$ and $Y$ are random variables on $(\Omega, \calF, \Prob)$ and pick $A\in \calF$. Define
    \begin{equation}
        Z(\omega) = \begin{cases}
            X(\omega), \quad \omega\in A \\
            Y(\omega), \quad \omega\in A^c.
        \end{cases}
    \end{equation}
    Show that $Z$ is a random variable.
\end{exer}
\begin{sol}
    Let $B$ be a Borel set and consider the set $\{ Z \in B\}$. We can write $Z^{-1}B = (Z^{-1}B\cap A) \cup (Z^{-1}B \cap A^c)$. This allows us to see that 
\begin{equation}
    \{ Z \in B\} = \{\omega\mid \omega\in (Z^{-1}B\cap A) \cup (Z^{-1}B \cap A^c) \}= \{\omega\mid \omega\in Z^{-1}B\cap A \} \cup \{\omega \mid \omega \in Z^{-1}B\cap A^c \}.
\end{equation}
since the preimage of a set union is the union of the preimages of the sets composing it. Since $ (Z^{-1}B\cap A)\subset A$ and $(Z^{-1}B\cap A^c) \subset A^c$, we know that $Z = X$ on $Z^{-1}B\cap A$ and $Z=Y$ on $Z^{-1}B\cap A^c$. This means that
\begin{equation}
    \{ Z\in B\} = \{\omega\mid \omega \in X^{-1}(B)\cap A \} \cup \{\omega\mid\omega \in Y^{-1}(B)\cap A^c\}.
\end{equation}
Since both the events on the righthand side are in $\calF$ because $X$ and $Y$ are random variables, their union $\{Z\in B\}\in\calF$. This shows $\{ Z\in B\}$ is measurable for any Borel set $B$, meaning $Z$ is a random variable.
\end{sol}

\newpage 
\begin{exer}
    Suppose that $X$ is a continuous random variable with distribution $F_X$. Let $g$ be a strictly increasing continuous function and define $Y=g(X)$. 

    \begin{enumerate}[a)]
        \item What is the distribution function of $Y$ $F_Y$?
        \item What is $f_Y$ the density function of $Y$?
    \end{enumerate}
\end{exer}

\begin{sol}\leavevmode
    
    a) We can compute the distribution function of $Y$ directy as
    \begin{equation}
        F_Y(x) = \Prob(Y\leq x) = \Prob(g(X)\leq x).
    \end{equation}
    Since $g$ is a strictly increasing continuous function it has a continuous inverse $g^{-1}$, therefore,
    \begin{equation}
        F_Y(x) = \Prob(g(X)\leq x) = \Prob(X \leq g^{-1}(x)) = (F_X\circ g^{-1})(x).
    \end{equation}
    
    b) Here we assume that $X$ has a density function $f_X = F'_X$. We can find the density of $Y$ by taking the derivative of its distrubution function, so that
    \begin{equation}
        f_Y(x) = \diff{}{x}(F_Y(x)) = \diff{}{x}(F_X(g^{-1}(x))).
    \end{equation}
    Assuming that $g$ is additionally differentiable and using the chain rule, we can compute that
    \begin{align}
        f_Y(x) &= \diff{}{x} \left[g^{-1}(x)\right] \cdot F'_X(g^{-1}(x))\\
               &= \diff{}{x} \big[ g^{-1}(x) \big]\cdot f_X(g^{-1}(x))
    \end{align}
    The second equality follows from the definition of the density. In this case, the derivative of the inverse function $g^{-1}$ should always be positive since $g$ is strictly increasing, but in general, I believe we have to use $\abs{\diff{}{x}g^{-1}(x)}$ to ensure the density is non-negative.
\end{sol}

\newpage
\begin{exer}
Suppose that $X$ is a continuous random variable with distribution function $F_X$. Find $F_Y$ where $Y$ is given by:
\begin{enumerate}[a)]
    \item $X^2$
    \item $\sqrt{\abs{X}}$
    \item $\sin X$
    \item $F_X(X)$
\end{enumerate}
\end{exer}

\begin{sol}
    a) We'll begin by trying to compute the distribution of $Y$ directly,
\begin{align}
    F_Y(x) &= \Prob(Y \leq x)\\
           &= \Prob(X^2 \leq x).
\end{align}
When $x\geq0$, we can see that
\begin{align}
    F_Y(x) &= \Prob(-\sqrt{x} \leq X \leq \sqrt{x})\\
           &= F_X(\sqrt{x}) - F_X(-\sqrt{x}).
\end{align}
If $x<0$,
\begin{align}
    F_Y(x) &= \Prob( X^2 \leq x )\\
           &= \Prob(X^2 \leq x < x) = 0
\end{align}
since the square of real-valued quantities must be non-negative. We can put this together to show that
\begin{equation}
    F_Y(x) = \begin{cases}
        F_X(\sqrt{x}) - F_X(-\sqrt{x}), \quad x \geq 0\\
        0, \quad x < 0. 
    \end{cases}
\end{equation}

b) Once again computing directly
\begin{align}
    F_Y(x) &= \Prob(Y \leq x)\\
           &= \Prob(\sqrt{\abs{X}} \leq x)
           &= \Prob(\abs{X} \leq x^2)
\end{align}
Like in the previous problem, this is 0 is zero for $x<0$, but for $x\geq0$, we can compute
\begin{align}
    F_Y(x) &= \Prob(Y\leq x)\\
           &= \Prob(-x^2 \leq X \leq x^2)\\
           &= F_X(x^2) - F_X(-x^2).
\end{align}

We can write this compactly as 
\begin{equation}
    F_Y(x) = \begin{cases}
        F_X(x^2) - F_X(-x^2), \quad x \geq 0\\
        0, \quad x < 0. 
    \end{cases}
\end{equation}

c) Computing directly,
\begin{align}
    F_Y(x) &= \Prob(Y \leq x) \\
           &= \Prob(\sin X \leq x).
\end{align}
Since $\sin X$ is bounded between $[0,1]$, we have that $F_Y(x) = 1$ for $x>1$ and $F_Y(x) = 0$ for $x<0$. Otherwise, if $x\in[0,1]$, we can take the inverse of $\sin X$, so that $\Prob(\sin X \leq x) = \Prob(X \leq \arcsin x)$
\begin{align}
    F_Y(x) = \begin{cases}
        1, \quad x > 1\\
        F_X(\arcsin x), \quad x\in [0,1]\\
        0, \quad x < 1
    \end{cases}
\end{align}

d) Computing directly, 
\begin{align}
    F_Y(x) &= \Prob(Y \leq x) \\
           &= \Prob( F_X(X) \leq x ) \\ 
           &= \Prob( X\leq \tilde{F}_X^{-1}(x)),
\end{align}
where $\tilde{F}_X^{-1}$ is the pseudo-inverse of the distribution $F$ defined by
\begin{equation}
    \tilde{F}_X^{-1} = \inf \{x \mid F(x)\geq u  \}.
\end{equation}
\end{sol}

\newpage

\begin{exer}
    Define $X\colon [0,1]\to \bbR$ by 
    \begin{equation}
        X(\omega) = \begin{cases}
            1, \quad \omega \in [0,1]\cap\bbQ^c\\
            0, \quad \omega \in [0,1]\cap\bbQ.
        \end{cases}
    \end{equation}
    As defined, $X$ is the function which takes rational numbers in $[0,1]$ to 0 and irrational numbers in $[0,1]$ to 1. Assuming that $X$ is defined on $([0,1], \mathcal{B}[0,1], \Prob)$ where $\Prob$ is the Lebesgue measure. Show that $X$ is a random variable. If it is, what are its distribution function and expectation? Does $X$ has a density function? Is $X$ discrete?
\end{exer}

\begin{sol}\leavevmode
    
 \textbf{X is a random variable. }% 
    Let $A = \bbQ^c \cap [0,1]$. We will show that $A\in \mathcal{B}[0,1]$. By definition of $\mathcal{B}[0,1]$, we know that every rational number $q\in \bbQ\cap[0,1]$ has $\{ q\} \in \mathcal{B}[0,1]$ since $[0, q)$ and $(q,1]$ are Borel sets and
    \begin{equation}
        [0,1] \setminus ([0,q) \cup (q,1]) = \{q \}.
    \end{equation} We also know that $\Prob(\{q\}) = 0$ under the Lebesgue measure since 
    \begin{equation}
        \Prob([0,q) \cup (q,1]) = 1 \text{ and } \Prob( [0,q ) \cup (q,1] \cup \{q \}) = \Prob( [0,1]) = 1.
    \end{equation}

    Since there are a countable number of rationals, we have that
    \begin{equation}
        \Prob( \bbQ \cap [0,1] ) = \sum_{q \in \bbQ} \Prob(\{ q\}) = 0
    \end{equation}
    by countable additivity. This means that $A$ is a measurable set with $\Prob(A) = 1$ since $A^c = \bbQ \cap [0,1]$ is measurable with $\Prob(A^c) = 0$. Notice that with this choice of $A$, we can see that $X$ is simply the indicator function of $A$ and as it is an indicator function of a measurable event it is a random variable. 

\textbf{Distribution of X. }%    
We can compute the distribution of $X$ using the fact that it is either 1 or 0. If $\omega \in \bbQ$, then $X = 0$ which occurs with probability 0. Similarly, if $\omega\notin\bbQ$, then $X=1$ which occurs with probability 1. Therefore, our distribution is described by
\begin{align}
    F_X(x) = \Prob(X \leq x) = \begin{cases}
        1, \quad x=1\\
        0, \quad x<1.
    \end{cases}
\end{align}

\textbf{Expectation of X.}
Since $X$ is an indicator function, we can compute its expectation quite simply as
\begin{equation}
    \Expect X = \Prob(A) = 1.
\end{equation}
\textbf{Density of X.} 
The random variable has no density since it is simply a point mass at $x=1$.

\textbf{X is discrete.} The random variable $X$ is discrete since it can only take the values 0 and 1.
\end{sol}
\end{document}
