%Preamble
\documentclass[12pt]{article}
\usepackage{fancyhdr}
\usepackage{extramarks}
\usepackage{amsmath}
\usepackage{amssymb}
\usepackage{amsthm}
\usepackage{amsrefs}
\usepackage{amsfonts}
\usepackage{mathrsfs}
\usepackage{mathtools}
\usepackage[mathcal]{eucal} %% changes meaning of \mathcal
\usepackage{enumerate}
\usepackage[shortlabels]{enumitem}
\usepackage{verbatim} %% includes comment environment
\usepackage{hyperref}
\usepackage[capitalize]{cleveref}
\crefformat{equation}{~(#2#1#3)}
\usepackage{caption, subcaption}
\usepackage{graphicx}
\usepackage{fullpage} %%smaller margins
\usepackage[all,arc]{xy}
\usepackage{mathrsfs}

\hypersetup{
    linktoc=all,     % set to all if you want both sections and subsections linked
}

\topmargin=-0.45in
\evensidemargin=0in
\oddsidemargin=0in
\textwidth=6.5in
\textheight=9.0in
\headsep=0.25in
\setlength{\headheight}{16pt}

\linespread{1.1}

\pagestyle{fancy}
\lhead{\Name}
\chead{\hwTitle}
\rhead{\hwClass}
\lfoot{\lastxmark}
\cfoot{\thepage}

\renewcommand\headrulewidth{0.4pt}
\renewcommand\footrulewidth{0.4pt}

\setlength\parindent{0pt}

%% Title Info
\newcommand{\hwTitle}{HW \# 1}
\newcommand{\hwDueDate}{October 12, 2020}
\newcommand{\hwClass}{AMATH 561A}
\newcommand{\hwClassTime}{}
\newcommand{\hwClassInstructor}{}
\newcommand{\Name}{\textbf{Marlin Figgins}}


%% MATH MACROS
\newcommand{\bbF}{\mathbb{F}}
\newcommand{\bbN}{\mathbb{N}}
\newcommand{\bbQ}{\mathbb{Q}}
\newcommand{\bbR}{\mathbb{R}}
\newcommand{\bbZ}{\mathbb{Z}}
\newcommand{\bbC}{\mathbb{C}}
\newcommand{\Prob}{\mathbb{P}}
\newcommand{\abs}[1]{ \left| #1 \right| }
\newcommand{\diff}[2]{\frac{d #1}{d #2}}
\newcommand{\infsum}[1]{\sum_{#1}^{\infty}}
\newcommand{\norm}[1]{ \left|\left| #1 \right|\right| }
\newcommand{\eval}[1]{ \left. #1 \right| }
\newcommand{\Expect}[1]{\mathbb{E}\left[#1 \right]}
\newcommand{\Var}[1]{\mathbb{V}\left[#1 \right]}
\renewcommand{\phi}{\varphi}
\renewcommand{\emptyset}{\O}

%--------Theorem Environments--------
%theoremstyle{plain} --- defaultx
\newtheorem{thm}{Theorem}[section]
\newtheorem{cor}[thm]{Corollary}
\newtheorem{prop}[thm]{Proposition}
\newtheorem{lem}[thm]{Lemma}
\newtheorem{conj}[thm]{Conjecture}
\newtheorem{quest}[thm]{Question}

\theoremstyle{definition}
\newtheorem{defn}[thm]{Definition}
\newtheorem{defns}[thm]{Definitions}
\newtheorem{con}[thm]{Construction}
\newtheorem{exmp}[thm]{Example}
\newtheorem{exmps}[thm]{Examples}
\newtheorem{notn}[thm]{Notation}
\newtheorem{notns}[thm]{Notations}
\newtheorem{addm}[thm]{Addendum}

% Environments for answers and solutions
\newtheorem{exer}{Exercise}
\newtheorem{sol}{Solution}

\theoremstyle{remark}
\newtheorem{rem}[thm]{Remark}
\newtheorem{rems}[thm]{Remarks}
\newtheorem{warn}[thm]{Warning}
\newtheorem{sch}[thm]{Scholium}

\makeatletter
\let\c@equation\c@thm
\makeatother

\begin{document}

\begin{exer}
Describe the probability space for the following experiments: a) a biased coin is tossed three times; b)  two balls are drawn without replacement from an urn which originally contained two blue and two red balls.
\end{exer}

\begin{sol}

\end{sol}

\begin{exer}[No translation-invariant random integer]
Show that there is no probability measure $\Prob$ on the integers $\mathbb{Z}$ with the discrete
$\sigma$-algebra $2^{\mathbb{Z}}$ with the translation-invariance property $\Prob(E + n) = \Prob(E)$ for every event $E \in 2^{\mathbb{Z}}$ and every integer $n$. $E+n$ is obtained by adding $n$ to every element of $E$.
\end{exer}
\begin{sol}

\end{sol}
\begin{exer}[No translation-invariant random real]
 Show that there is no probability measure $\Prob$ on the reals $\mathbb{R}$ with the Borel
$\sigma$-algebra $\mathcal{B}(\mathbb{R})$ with the translation-invariance property $\Prob(E + x) = \Prob(E)$ for every event $E \in \mathcal{B}(\mathbb{R})$ and every real $x$. Borel $\sigma$-algebra $\mathcal{B}(\mathbb{R})$ is the $\sigma$-algebra generated by intervals $(a,b] \subset \mathbb{R}$.
\end{exer}
\begin{sol}

\end{sol}
\begin{exer}
Let $\Omega=\mathbb{R}$, $\mathcal{F}=$ all subsets of $\mathbb{R}$ so that $A$ or $A^c$ is countable. $\Prob(A)=0$ in the first case and $\Prob(A)=1$ in the second. Show that $(\Omega, \mathcal{F}, \Prob)$ is a probability space.
\end{exer}
\begin{sol}

\end{sol}
\begin{exer}
A collection $\mathcal{A}$ of subsets of $\Omega$ is called an {\bf algebra} if $A,B \in \mathcal{A}$ implies $A^c$ and $A\cup B$ are in  $\mathcal{A}$. (a) Show that if $\mathcal{F}_1 \subset \mathcal{F}_2 \subset ...$ are $\sigma$-algebras, then $\cup_i \mathcal{F}_i$ is an algebra. (b) Give an example to show that $\cup_i \mathcal{F}_i$ need not be a $\sigma$-algebra.
\end{exer}
\begin{sol}

\end{sol}
\end{document}
