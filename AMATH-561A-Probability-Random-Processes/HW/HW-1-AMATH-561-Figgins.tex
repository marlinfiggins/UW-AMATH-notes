%Preamble
\documentclass[12pt]{article}
\usepackage{fancyhdr}
\usepackage{extramarks}
\usepackage{amsmath}
\usepackage{amssymb}
\usepackage{amsthm}
\usepackage{amsrefs}
\usepackage{amsfonts}
\usepackage{mathrsfs}
\usepackage{mathtools}
\usepackage[mathcal]{eucal} %% changes meaning of \mathcal
\usepackage{enumerate}
\usepackage[shortlabels]{enumitem}
\usepackage{verbatim} %% includes comment environment
\usepackage{hyperref}
\usepackage[capitalize]{cleveref}
\crefformat{equation}{~(#2#1#3)}
\usepackage{caption, subcaption}
\usepackage{graphicx}
\usepackage{fullpage} %%smaller margins
\usepackage[all,arc]{xy}
\usepackage{mathrsfs}

\hypersetup{
    linktoc=all,     % set to all if you want both sections and subsections linked
}

\topmargin=-0.45in
\evensidemargin=0in
\oddsidemargin=0in
\textwidth=6.5in
\textheight=9.0in
\headsep=0.25in
\setlength{\headheight}{16pt}

\linespread{1.1}

\pagestyle{fancy}
\lhead{\Name}
\chead{\hwTitle}
\rhead{\hwClass}
\lfoot{\lastxmark}
\cfoot{\thepage}

\renewcommand\headrulewidth{0.4pt}
\renewcommand\footrulewidth{0.4pt}

\setlength\parindent{0pt}

%% Title Info
\newcommand{\hwTitle}{HW \# 1}
\newcommand{\hwDueDate}{October 12, 2020}
\newcommand{\hwClass}{AMATH 561A}
\newcommand{\hwClassTime}{}
\newcommand{\hwClassInstructor}{}
\newcommand{\Name}{\textbf{Marlin Figgins}}


%% MATH MACROS
\newcommand{\bbF}{\mathbb{F}}
\newcommand{\bbN}{\mathbb{N}}
\newcommand{\bbQ}{\mathbb{Q}}
\newcommand{\bbR}{\mathbb{R}}
\newcommand{\bbZ}{\mathbb{Z}}
\newcommand{\bbC}{\mathbb{C}}
\newcommand{\Prob}{\mathbb{P}}
\newcommand{\calF}{\mathcal{F}}
\newcommand{\abs}[1]{ \left| #1 \right| }
\newcommand{\diff}[2]{\frac{d #1}{d #2}}
\newcommand{\infsum}[1]{\sum_{#1}^{\infty}}
\newcommand{\norm}[1]{ \left|\left| #1 \right|\right| }
\newcommand{\eval}[1]{ \left. #1 \right| }
\newcommand{\Expect}[1]{\mathbb{E}\left[#1 \right]}
\newcommand{\Var}[1]{\mathbb{V}\left[#1 \right]}
\renewcommand{\phi}{\varphi}
\renewcommand{\emptyset}{\O}

%--------Theorem Environments--------
%theoremstyle{plain} --- defaultx
\newtheorem{thm}{Theorem}[section]
\newtheorem{cor}[thm]{Corollary}
\newtheorem{prop}[thm]{Proposition}
\newtheorem{lem}[thm]{Lemma}
\newtheorem{conj}[thm]{Conjecture}
\newtheorem{quest}[thm]{Question}

\theoremstyle{definition}
\newtheorem{defn}[thm]{Definition}
\newtheorem{defns}[thm]{Definitions}
\newtheorem{con}[thm]{Construction}
\newtheorem{exmp}[thm]{Example}
\newtheorem{exmps}[thm]{Examples}
\newtheorem{notn}[thm]{Notation}
\newtheorem{notns}[thm]{Notations}
\newtheorem{addm}[thm]{Addendum}

% Environments for answers and solutions
\newtheorem{exer}{Exercise}
\newtheorem{sol}{Solution}

\theoremstyle{remark}
\newtheorem{rem}[thm]{Remark}
\newtheorem{rems}[thm]{Remarks}
\newtheorem{warn}[thm]{Warning}
\newtheorem{sch}[thm]{Scholium}

\makeatletter
\let\c@equation\c@thm
\makeatother

\begin{document}

\begin{exer}
Describe the probability space for the following experiments: a) a biased coin is tossed three times; b)  two balls are drawn without replacement from an urn which originally contained two blue and two red balls.
\end{exer}

\begin{sol}\leavevmode

    1a. If we were to flip a biased coin 3 times, we would have 8 possible outcomes:
\begin{equation}
    \Omega = \{HHH, HHT, HTH, HTT, THH, THT, TTH, TTT \}.
\end{equation}
Since $\Omega$ is finite, we can simply take the $\sigma$-algebra of events to be all subsets of $\Omega$ i.e. $\calF = 2^\Omega$. Since the coin is biased, we'll say the probability of heads for each toss is $p$ with the probability of tails being $q=1-p$. For each outcome $\omega\in\Omega$, we define $\abs{H}_\omega$ to be the number of heads in $\omega$. Similarly, we define $\abs{T}_\omega$ to be the number of tails in $\omega$. Then assuming that each toss is independent, we define the probability mass function as
\begin{equation}
    f(\omega) = p^{\abs{H}_\omega}q^{\abs{T}_\omega}.
\end{equation}
We can then compute that
\begin{equation}
    \sum_{\omega\in\Omega} f(\omega) = \sum_{\omega\in\Omega} p^{\abs{H}_\omega}q^{\abs{T}_\omega}.
\end{equation}

Noticing that $\abs{H}_\omega + \abs{T}_\omega = 3$ for all $\omega$, we can write this 
\begin{equation} 
    \sum_{\omega\in\Omega} f(\omega) = \sum_{k=1}^{3} \binom{3}{k} p^{k}q^{3-k},
\end{equation}
since each head-tail count $(k,3-k)$ appears exactly $\binom{3}{k}$ times. We can use the binomial theorem to compute that
\begin{align}
    \sum_{\omega\in\Omega} f(\omega) &= \sum_{k=1}^{3} \binom{3}{k} p^{k}q^{3-k},\\
                                     &= (p + q)^3 = (p + 1 - p)^3 = 1.
\end{align} Therefore, $f$ is indeed a probability mass function and induces a probability measure $\Prob$ on $(\Omega, \calF)$.

\newpage

1b. If we were to draw from an urn which contains two blue and two red balls $U = \{B, B, R, R\}$ without replacements, then our possible samples are 
\begin{equation}
    \Omega = \{BB, BR, RB, RR\}.
\end{equation}
Once again since this probability space is finite, we can just take the $\sigma$-algebra of events to be $\calF = 2^\Omega$. This leaves us to find the probability mass function. Assuming that each ball is equally likely to be pulled and we do not replace the balls, then we have the probability of pulling blue is equal to the fraction of balls currently in the urn which are blue. This same logic applies for red balls. Therefore, we can compute the associated probability mass function $p$ as: 

\begin{align}
    p(BB) = \frac{2}{4} \cdot \frac{1}{3} = \frac{1}{6} &\quad p(BR) = \frac{2}{4}\cdot \frac{2}{3} = \frac{1}{3} \\
    p(RB) = \frac{2}{4} \cdot \frac{2}{3} = \frac{1}{3} &\quad p(RR) = \frac{2}{4}\cdot \frac{1}{3} = \frac{1}{6} 
\end{align}

This is certainly a probability mass function as 
\begin{equation}
    \sum_{\omega\in\Omega}p(\omega) = p(BB) + p(BR) + p(RB) + p(RR) = 1.
\end{equation} Therefore, we have a probability space $(\Omega, \calF, \Prob)$ as $p$ induces a probability measure $\Prob$ on $(\Omega, \calF)$.
\end{sol}

\newpage

\begin{exer}[No translation-invariant random integer]
Show that there is no probability measure $\Prob$ on the integers $\mathbb{Z}$ with the discrete
$\sigma$-algebra $2^{\mathbb{Z}}$ with the translation-invariance property $\Prob(E + n) = \Prob(E)$ for every event $E \in 2^{\mathbb{Z}}$ and every integer $n$. $E+n$ is obtained by adding $n$ to every element of $E$.
\end{exer}
\begin{sol}
    Suppose that we have a translation invariant probability measure on $\bbZ$. Consider the sets $E_0 = \{ 0\}$ and $E_n =\{ n\} = E_0 + n$. By translation invariance, we have that $\Prob(E_0) = \Prob(E_n) = p \in [0,1)$ for all $n\in\bbZ$. Using the countable additivity of disjoint sets and the fact that $\Prob(\bbZ)=1$, we see that
    \begin{align}
        \sum_{n\in\bbZ} p = \sum_{n\in\bbZ} \Prob(E_0) &= \sum_{n\in\bbZ} \Prob(E_n)\\
                                                       &= \sum_{n\in\bbZ} \Prob(\{ n \})\\
                                                       &= \Prob \left( \bigcup_{n\in\bbZ} \{n \}\right)\\
                                                       &= \Prob(\bbZ) = 1.
    \end{align}
    This cannot hold since the sum $\sum_{n\in\bbZ} p$ is divergent for $p\neq 0$ and 0 otherwise. Therefore, there is no such probability measure.
\end{sol}

\newpage
\begin{exer}[No translation-invariant random real]
 Show that there is no probability measure $\Prob$ on the reals $\mathbb{R}$ with the Borel
$\sigma$-algebra $\mathcal{B}(\mathbb{R})$ with the translation-invariance property $\Prob(E + x) = \Prob(E)$ for every event $E \in \mathcal{B}(\mathbb{R})$ and every real $x$. Borel $\sigma$-algebra $\mathcal{B}(\mathbb{R})$ is the $\sigma$-algebra generated by intervals $(a,b] \subset \mathbb{R}$.
\end{exer}
\begin{sol}
    Suppose that we have a translation invariant probability measure on $\bbR$. Consider the sets $E_0 = (0,1] \in \mathcal{B}(\bbR)$ and $E_n = (n, n+1] = E_0 + n \in \mathcal{B}(\bbR)$. Because $\Prob$ is translation invariant, we have that $\Prob(E_0) = \Prob(E_n) = p \in [0,1)$. Using the countable additivity of disjoint sets and the fact that $\Prob(\bbR) = 1$, we see that
    \begin{align}
        \sum_{n\in\bbZ} p = \sum_{n\in\bbZ} \Prob(E_0) &= \sum_{n\in\bbZ} \Prob(E_n)\\
                                                       &= \sum_{n\in\bbZ} \Prob(\{(n, n+1] \})\\
                                                       &= \Prob \left( \bigcup_{n\in\bbZ} \{ (n, n+1]\}\right)\\
                                                       &= \Prob(\bbR) = 1.
    \end{align}
    This equality cannot hold since the sum $\sum_{n\in\bbZ} p$ is divergent for all $p\neq 0$ and 0 otherwise. Therefore, there cannot be such a probability measure on $\bbR$.
\end{sol}

\newpage
\begin{exer}
Let $\Omega=\mathbb{R}$, $\mathcal{F}=$ all subsets of $\mathbb{R}$ so that $A$ or $A^c$ is countable. $\Prob(A)=0$ in the first case and $\Prob(A)=1$ in the second. Show that $(\Omega, \mathcal{F}, \Prob)$ is a probability space.
\end{exer}
\begin{sol}
\emph{We'll first show that $\calF$ is a $\sigma$-algebra on $\Omega$.}

Need $\varnothing,\bbR \in\calF$ as well as countable unions and complements. 

We have that $\varnothing, \bbR \in \calF$ since $\varnothing$ is countable and $\bbR^c = \varnothing$. Similarly, we have that $\calF$ is closed under complements by definition since if $E \in \calF$, then it either is countable in which case $E^c \in \calF$ since it has a countable complement. If $E\in \calF$ is not countable, then $E^c$ is countable and therefore in $\calF$. Now we'll show that $\calF$ is closed under countable unions. Suppose that we have a countable collection of sets $E_i \in \calF$ for $i\in\bbN$. If all these sets are countable, then $E = \cup_{i\in\bbN} E_i$ is countable since it is the countable union of countable sets. If there is an uncountable set $E_k$ in this collection of sets, then we have that 
\begin{equation}
    E^c = \left( \bigcup_{i\in\bbN} E_i \right)^c \subset E_k^c.
\end{equation}

Due to the fact that $E^c \subset E_k^c$ and $E_k^c$ is countable, $E^c$ is as well. Therefore, $E\in\calF$. This shows that $\calF$ is a $\sigma$-algebra.

\emph{We'll now show that $\Prob$ is a probability measure on $(\Omega, \calF)$}

By definition, $\Prob$ is non-negative for all $\calF$. This leaves us to chceck that $\Prob$ has $\Prob(\bbR) = 1$ and is countably additive. By our definition, $\varnothing = \bbR^c$ is countable, and has $\Prob(\varnothing) = 0$. Therefore, $\Prob(\bbR) = 1$. Next, we prove countable additivity.

Suppose that we have a countable collection of sets $E_i \in \calF$ which are disjoint. If there is one such $E_k$ which has $E_k^c$ countable, then because the sets are disjoint, we have 
\begin{equation}
\bigcup_{i\neq k} E_i \subset E_k^c.
\end{equation}
Therefore, all other $E_i$ are countable since they are subsets of a countable set. This means that $\Prob(E_i) = 0$ for all $i\neq k$ and $\Prob(E_k) = 1$. Additionally, 
\begin{equation}
    E^c = \left(  \bigcup_{i = 0}^\infty E_i \right)^c \subset E_k^c. 
    \end{equation} Therefore $E^c$ is countable as well, so $\Prob(E) = 1$. Therefore, 
\begin{equation}
    \Prob(E) = 1 = \Prob(E_k) = \sum_{i = 1}^\infty \Prob(E_i).
\end{equation}

In the case in which all $E_i$ are instead countable, we have that $E$ will be countable since it is the countable union of countable sets. Therefore 
\begin{equation}
    \Prob(E) = 0 = \sum_{i = 1}^\infty \Prob(E_i).
\end{equation}
This means that $\Prob$ is countably additive. This shows that $\Prob$ is a probability meausre on $(\Omega, \calF)$ and therefore, $(\Omega,\calF, \Prob)$ is a probability space.
\end{sol}

\end{document}
