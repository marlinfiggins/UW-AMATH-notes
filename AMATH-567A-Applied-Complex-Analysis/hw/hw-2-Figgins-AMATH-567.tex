%Preamble
\documentclass[12pt]{article}
\usepackage{fancyhdr}
\usepackage{extramarks}
\usepackage{amsmath}
\usepackage{amssymb}
\usepackage{amsthm}
\usepackage{amsrefs}
\usepackage{amsfonts}
\usepackage{mathrsfs}
\usepackage{mathtools}
\usepackage[mathcal]{eucal} %% changes meaning of \mathcal
\usepackage{enumerate}
\usepackage[shortlabels]{enumitem}
\usepackage{verbatim} %% includes comment environment
\usepackage{hyperref}
\usepackage[capitalize]{cleveref}
\crefformat{equation}{~(#2#1#3)}
\usepackage{caption, subcaption}
\usepackage{graphicx}
\usepackage{fullpage} %%smaller margins
\usepackage[all,arc]{xy}
\usepackage{mathrsfs}

\hypersetup{
    linktoc=all,     % set to all if you want both sections and subsections linked
}

\topmargin=-0.45in
\evensidemargin=0in
\oddsidemargin=0in
\textwidth=6.5in
\textheight=9.0in
\headsep=0.25in
\setlength{\headheight}{16pt}

\linespread{1.1}

\pagestyle{fancy}
\lhead{\Name}
\chead{\hwTitle}
\rhead{\hwClass}
\lfoot{\lastxmark}
\cfoot{\thepage}

\renewcommand\headrulewidth{0.4pt}
\renewcommand\footrulewidth{0.4pt}

\setlength\parindent{0pt}

%% Title Info
\newcommand{\hwTitle}{HW \# 2}
\newcommand{\hwDueDate}{\today}
\newcommand{\hwClass}{AMATH 567A}
\newcommand{\hwClassTime}{}
\newcommand{\hwClassInstructor}{}
\newcommand{\Name}{\textbf{Marlin Figgins}}


%% MATH MACROS
\newcommand{\bbF}{\mathbb{F}}
\newcommand{\bbN}{\mathbb{N}}
\newcommand{\bbQ}{\mathbb{Q}}
\newcommand{\bbR}{\mathbb{R}}
\newcommand{\bbZ}{\mathbb{Z}}
\newcommand{\bbC}{\mathbb{C}}
\newcommand{\abs}[1]{ \left| #1 \right| }
\newcommand{\diff}[2]{\frac{d #1}{d #2}}
\newcommand{\infsum}[1]{\sum_{#1}^{\infty}}
\newcommand{\norm}[1]{ \left|\left| #1 \right|\right| }
\newcommand{\eval}[1]{ \left. #1 \right| }
\newcommand{\Expect}[1]{\mathbb{E}\left[#1 \right]}
\newcommand{\Var}[1]{\mathbb{V}\left[#1 \right]}
\newcommand{\Res}{\text{Res}}
\renewcommand{\phi}{\varphi}
\renewcommand{\emptyset}{\O}

%--------Theorem Environments--------
%theoremstyle{plain} --- defaultx
\newtheorem{thm}{Theorem}[section]
\newtheorem{cor}[thm]{Corollary}
\newtheorem{prop}[thm]{Proposition}
\newtheorem{lem}[thm]{Lemma}
\newtheorem{conj}[thm]{Conjecture}
\newtheorem{quest}[thm]{Question}

\theoremstyle{definition}
\newtheorem{defn}[thm]{Definition}
\newtheorem{defns}[thm]{Definitions}
\newtheorem{con}[thm]{Construction}
\newtheorem{exmp}[thm]{Example}
\newtheorem{exmps}[thm]{Examples}
\newtheorem{notn}[thm]{Notation}
\newtheorem{notns}[thm]{Notations}
\newtheorem{addm}[thm]{Addendum}

% Environments for answers and solutions
\newtheorem{exer}{Exercise}
\newtheorem{sol}{Solution}

\theoremstyle{remark}
\newtheorem{rem}[thm]{Remark}
\newtheorem{rems}[thm]{Remarks}
\newtheorem{warn}[thm]{Warning}
\newtheorem{sch}[thm]{Scholium}

\makeatletter
\let\c@equation\c@thm
\makeatother

\begin{document}

\begin{exer}
    A\&F 2.5.1. Evaluate $\oint_\gamma f(z)dz$ where $\gamma$ is the unit circle centered at the origin for the follwoing functions $f$.
\end{exer}
\begin{sol}\leavevmode

    (a) $f(z) = e^{iz}$. The function $f(z)$ is entire as it is the composition of two entire functions $e^{w}$ and $iz$. Its derivative is $f'(z) = ie^{iz}$. By Cauchy's Theorem, this means that for the closed curve $\gamma$, we have
    \begin{equation}
        \oint_\gamma e^{iz}dz = 0.
    \end{equation} 

    (b) $f(z) = e^{z^2}$. Once again $f(z)$ is entire as it is the composition of two entire functions $e^{w}$ and $iz$. By Cauchy's Theorem, this means
    \begin{equation}
        \oint_\gamma e^{iz}dz = 0.
    \end{equation}

    (c) $f(z) = \frac{1}{z-1/2}$. The function $f(z)$ is analytic except at $z = \frac{1}{2}$ which is contained in $\gamma$, so we cannot use Cauchy's theorem. We can instead use the residue theorem. Writing $f$ as its Taylor-Laurent series about $z_0 = \frac{1}{2}$,
    \begin{equation}
        f(z) = \sum_{n\in\bbZ} a_n (z-1/2)^n = 1 (z - 1/2)^{-1}.
    \end{equation}
    Here we can see that $a_n = 0$ for all $n\neq -1$ and $a_{-1} = 1$. Therefore, by the Residue theorem, we have
    \begin{equation}
        \oint_C f(z) dz=  2\pi i a_{-1} = 2\pi i.
    \end{equation}
    %TODO: Double check computation above and write out with logic

    (d) $f(z) = \frac{1}{z^2-4}$. The function $f(z)$ is analytic except at $z = 2, -2$, neither of which are in the contour $\gamma$. Since $f(z)$ is analytic on and within $\gamma$, we can apply Cauchy's theorem, so that
    \begin{equation}
        \oint_\gamma \frac{1}{z^2-4}dz = 0.
    \end{equation}

    (e) $f(z) = \frac{1}{2z^2+1}$. This function is analytic except at $z_{\pm} = i\frac{\sqrt{2}}{2}, - i\frac{\sqrt{2}}{2}$ which are contained in the contour $\gamma$. We can then write 
    \begin{equation}
        f(z) = \frac{1}{2z^2 + 1} = \frac{1}{2(z - i\frac{\sqrt{2}}{2})(z + i\frac{\sqrt{2}}{2})}.
    \end{equation}
    We'll now compute the residues at $z_{\pm}$ using the following formula for the residue of $f$ at $z_0$ 
    \begin{equation}
        \Res(f, z_0) = \lim\limits_{z\to z_0} (z-z_0)f(z).
    \end{equation}
    For $z_{+}$., we can compute
    \begin{align}
        \Res(f,z_{+}) &= \lim\limits_{z\to z_+} \left(z - i\frac{\sqrt{2}}{2} \right)f(z)\\ 
                      &= \lim\limits_{z\to z_+} \frac{1}{2(z + i\frac{\sqrt{2}}{2})} \\
                      &= \frac{1}{2 \left(i\frac{\sqrt{2}}{2} + i\frac{\sqrt{2}}{2} \right)}\\
                      &= \frac{1}{2i\sqrt{2}} = \frac{\sqrt{2}}{4i}
    \end{align}
    Similarly, we can compute the residue at $z_{-}$
    \begin{align}
        \Res(f,z_{-}) &= \lim\limits_{z\to z_-} \left(z + i\frac{\sqrt{2}}{2} \right)f(z)\\ 
                      &= \lim\limits_{z\to z_-} \frac{1}{2(z - i\frac{\sqrt{2}}{2})} \\
                      &= \frac{1}{2 \left(-i\frac{\sqrt{2}}{2} - i\frac{\sqrt{2}}{2} \right)}\\
                      &= -\frac{1}{2i\sqrt{2}} = -\frac{\sqrt{2}}{4i}\\
                      &=  -\Res(f,z_{+}) 
    \end{align}

    We can then use the residue theorem to compute the integral of $f$ over $\gamma$ as follows
    \begin{equation}
        \oint_C \frac{1}{2z^2+1} dz = 2\pi i \left(  \Res(f,z_{+}) +   \Res(f,z_{-})    \right) = 0
    \end{equation}

    (f)
\end{sol}

\newpage

\begin{exer}
    A\&F 2.5.5.
\end{exer}
\begin{sol}

\end{sol}

\newpage

\begin{exer}
    A\&F 2.5.6.
\end{exer}
\begin{sol}

\end{sol}
\newpage

\begin{exer}
    A\&F 3.3.5.
\end{exer}

\begin{sol}
    In order to find the coeffiecients of the Taylor-Laurent Series about 0 of $f(z) = e^{\frac{t}{2}(z - z^{-1})} = \sum_{n\in\bbZ} a_n z^n$, we use the formula
    \begin{equation}
        a_n = \frac{1}{2\pi i} \int_\gamma \frac{ e^{\frac{t}{2}(z - z^{-1})}}{z^{n+1}} dz
    \end{equation}
    where $\gamma$ is the unit circle parameterized as $\gamma(\theta) = e^{i\theta}$ for $\theta \in [-\pi, \pi]$. We can then simplify the integral as
    \begin{align}
        a_n &= \frac{1}{2\pi i} \int_\gamma \frac{ e^{\frac{t}{2}(z - z^{-1})}}{z^{n+1}} dz\\
            &= \frac{1}{2\pi i} \int_{-\pi}^{\pi} \frac{e^{\frac{t}{2}( e^{i\theta} - e^{-i\theta} )}  }{ e^{i(n+1)\theta} } \cdot ie^{i\theta}d\theta.
    \end{align}
    We can simplify $e^{i\theta} - e^{-i\theta}$ as $2i\sin\theta$ and combine the terms $e^{i\theta}$ and $e^{i(n+1) \theta}$, so that
\begin{equation}
  \frac{1}{2\pi i} \int_{-\pi}^{\pi} \frac{e^{\frac{t}{2}( e^{i\theta} - e^{-i\theta} )}  }{ e^{i(n+1)\theta} } \cdot ie^{i\theta}d\theta = \frac{i}{2\pi i} \int_{-\pi}^{\pi} \frac{e^{\frac{t}{2}( 2i\sin\theta )}  }{ e^{in\theta} } d\theta  
\end{equation}

Combining the top and bottom halves of the integrad and canceling the $i$ in front, we get that
\begin{equation}
    a_n = \frac{1}{2\pi} \int_{-\pi}^\pi e^{-i(n\theta - t\sin\theta)}d\theta.
\end{equation}
We can further simplify this using $e^{-ix} = \cos x - i\sin x$, which gives us
\begin{equation}
    a_n = \frac{1}{2\pi}\int_{-\pi}^\pi \cos ( n\theta - t\sin\theta )d\theta +  \frac{i}{2\pi}\int_{-\pi}^{\pi} \sin( n\theta - t\sin\theta ) d\theta.
\end{equation}

Show that $g(\theta) = n\theta - t\sin\theta$ is odd. 

Therefore, $\cos(g(\theta))$ is even and $\sin(g(\theta))$ is odd. This means that

\begin{align}
    \int_{-\pi}^\pi \cos ( n\theta - t\sin\theta )d\theta &= 2 \int_{0}^\pi \cos ( n\theta - t\sin\theta )d\theta \\
    \int_{-\pi}^{\pi} \sin( n\theta - t\sin\theta ) d\theta &= 0
\end{align}
Therefore,
\begin{equation}
    a_n = \frac{1}{2\pi} \int_{-\pi}^\pi e^{-i(n\theta - t\sin\theta)}d\theta = \frac{1}{\pi} \int_{0}^\pi \cos ( n\theta - t\sin\theta )d\theta.
\end{equation}
\end{sol}

\end{document}
