%Preamble
\documentclass[12pt]{article}
\usepackage{fancyhdr}
\usepackage{extramarks}
\usepackage{amsmath}
\usepackage{amssymb}
\usepackage{amsthm}
\usepackage{amsrefs}
\usepackage{amsfonts}
\usepackage{mathrsfs}
\usepackage{mathtools}
\usepackage[mathcal]{eucal} %% changes meaning of \mathcal
\usepackage{enumerate}
\usepackage[shortlabels]{enumitem}
\usepackage{verbatim} %% includes comment environment
\usepackage{hyperref}
\usepackage[capitalize]{cleveref}
\crefformat{equation}{~(#2#1#3)}
\usepackage{caption, subcaption}
\usepackage{graphicx}
\usepackage{fullpage} %%smaller margins
\usepackage[all,arc]{xy}
\usepackage{mathrsfs}

\hypersetup{
    linktoc=all,     % set to all if you want both sections and subsections linked
}

\topmargin=-0.45in
\evensidemargin=0in
\oddsidemargin=0in
\textwidth=6.5in
\textheight=9.0in
\headsep=0.25in
\setlength{\headheight}{16pt}

\linespread{1.1}

\pagestyle{fancy}
\lhead{\Name}
\chead{\hwTitle}
\rhead{\hwClass}
\lfoot{\lastxmark}
\cfoot{\thepage}

\renewcommand\headrulewidth{0.4pt}
\renewcommand\footrulewidth{0.4pt}

\setlength\parindent{0pt}

%% Title Info
\newcommand{\hwTitle}{HW \# 2}
\newcommand{\hwDueDate}{\today}
\newcommand{\hwClass}{AMATH 567A}
\newcommand{\hwClassTime}{}
\newcommand{\hwClassInstructor}{}
\newcommand{\Name}{\textbf{Marlin Figgins}}


%% MATH MACROS
\newcommand{\bbF}{\mathbb{F}}
\newcommand{\bbN}{\mathbb{N}}
\newcommand{\bbQ}{\mathbb{Q}}
\newcommand{\bbR}{\mathbb{R}}
\newcommand{\bbZ}{\mathbb{Z}}
\newcommand{\bbC}{\mathbb{C}}
\newcommand{\abs}[1]{ \left| #1 \right| }
\newcommand{\diff}[2]{\frac{d #1}{d #2}}
\newcommand{\infsum}[1]{\sum_{#1}^{\infty}}
\newcommand{\norm}[1]{ \left|\left| #1 \right|\right| }
\newcommand{\eval}[1]{ \left. #1 \right| }
\newcommand{\Expect}[1]{\mathbb{E}\left[#1 \right]}
\newcommand{\Var}[1]{\mathbb{V}\left[#1 \right]}
\newcommand{\Res}{\text{Res}}
\renewcommand{\phi}{\varphi}
\renewcommand{\emptyset}{\O}

%--------Theorem Environments--------
%theoremstyle{plain} --- defaultx
\newtheorem{thm}{Theorem}[section]
\newtheorem{cor}[thm]{Corollary}
\newtheorem{prop}[thm]{Proposition}
\newtheorem{lem}[thm]{Lemma}
\newtheorem{conj}[thm]{Conjecture}
\newtheorem{quest}[thm]{Question}

\theoremstyle{definition}
\newtheorem{defn}[thm]{Definition}
\newtheorem{defns}[thm]{Definitions}
\newtheorem{con}[thm]{Construction}
\newtheorem{exmp}[thm]{Example}
\newtheorem{exmps}[thm]{Examples}
\newtheorem{notn}[thm]{Notation}
\newtheorem{notns}[thm]{Notations}
\newtheorem{addm}[thm]{Addendum}

% Environments for answers and solutions
\newtheorem{exer}{Exercise}
\newtheorem{sol}{Solution}

\theoremstyle{remark}
\newtheorem{rem}[thm]{Remark}
\newtheorem{rems}[thm]{Remarks}
\newtheorem{warn}[thm]{Warning}
\newtheorem{sch}[thm]{Scholium}

\makeatletter
\let\c@equation\c@thm
\makeatother

\begin{document}

\begin{exer}
    A\&F 2.5.1. Evaluate $\oint_\gamma f(z)dz$ where $\gamma$ is the unit circle centered at the origin for the follwoing functions $f$.
\end{exer}
\begin{sol}\leavevmode

    (a) $f(z) = e^{iz}$. The function $f(z)$ is entire as it is the composition of two entire functions $e^{w}$ and $iz$. Its derivative is $f'(z) = ie^{iz}$. By Cauchy's Theorem, this means that for the closed curve $\gamma$, we have
    \begin{equation}
        \oint_\gamma e^{iz}dz = 0.
    \end{equation} 

    (b) $f(z) = e^{z^2}$. Once again $f(z)$ is entire as it is the composition of two entire functions $e^{w}$ and $iz$. By Cauchy's Theorem, this means
    \begin{equation}
        \oint_\gamma e^{iz}dz = 0.
    \end{equation}

    (c) $f(z) = \frac{1}{z-1/2}$. The function $f(z)$ is analytic except at $z = \frac{1}{2}$ which is contained in $\gamma$, so we cannot use Cauchy's theorem. We can instead use the residue theorem. Writing $f$ as its Taylor-Laurent series about $z_0 = \frac{1}{2}$,
    \begin{equation}
        f(z) = \sum_{n\in\bbZ} a_n (z-1/2)^n = 1 (z - 1/2)^{-1}.
    \end{equation}
    Here we can see that $a_n = 0$ for all $n\neq -1$ and $a_{-1} = 1$. Therefore, by the Residue theorem, we have
    \begin{equation}
        \oint_C f(z) dz=  2\pi i a_{-1} = 2\pi i.
    \end{equation}
    %TODO: Double check computation above and write out with logic

    (d) $f(z) = \frac{1}{z^2-4}$. The function $f(z)$ is analytic except at $z = 2, -2$, neither of which are in the contour $\gamma$. Since $f(z)$ is analytic on and within $\gamma$, we can apply Cauchy's theorem, so that
    \begin{equation}
        \oint_\gamma \frac{1}{z^2-4}dz = 0.
    \end{equation}

    (e) $f(z) = \frac{1}{2z^2+1}$. This function is analytic except at $z_{\pm} = i\frac{\sqrt{2}}{2}, - i\frac{\sqrt{2}}{2}$ which are contained in the contour $\gamma$. We can then write 
    \begin{equation}
        f(z) = \frac{1}{2z^2 + 1} = \frac{1}{2(z - i\frac{\sqrt{2}}{2})(z + i\frac{\sqrt{2}}{2})}.
    \end{equation}
    We'll now compute the residues at $z_{\pm}$ using the following formula for the residue of $f$ at $z_0$ 
    \begin{equation}
        \Res(f, z_0) = \lim\limits_{z\to z_0} (z-z_0)f(z).
    \end{equation}
    For $z_{+}$., we can compute
    \begin{align}
        \Res(f,z_{+}) &= \lim\limits_{z\to z_+} \left(z - i\frac{\sqrt{2}}{2} \right)f(z)\\ 
                      &= \lim\limits_{z\to z_+} \frac{1}{2(z + i\frac{\sqrt{2}}{2})} \\
                      &= \frac{1}{2 \left(i\frac{\sqrt{2}}{2} + i\frac{\sqrt{2}}{2} \right)}\\
                      &= \frac{1}{2i\sqrt{2}} = \frac{\sqrt{2}}{4i}
    \end{align}
    Similarly, we can compute the residue at $z_{-}$
    \begin{align}
        \Res(f,z_{-}) &= \lim\limits_{z\to z_-} \left(z + i\frac{\sqrt{2}}{2} \right)f(z)\\ 
                      &= \lim\limits_{z\to z_-} \frac{1}{2(z - i\frac{\sqrt{2}}{2})} \\
                      &= \frac{1}{2 \left(-i\frac{\sqrt{2}}{2} - i\frac{\sqrt{2}}{2} \right)}\\
                      &= -\frac{1}{2i\sqrt{2}} = -\frac{\sqrt{2}}{4i}\\
                      &=  -\Res(f,z_{+}) 
    \end{align}

    We can then use the residue theorem to compute the integral of $f$ over $\gamma$ as follows
    \begin{equation}
        \oint_C \frac{1}{2z^2+1} dz = 2\pi i \left(  \Res(f,z_{+}) +   \Res(f,z_{-})    \right) = 0
    \end{equation}

    (f) $\sqrt{z-4} 0\leq\arg z<2\pi.$ With the specification on the argument, this function is analytic with derivative $\frac{1}{z-4}$ except on the real line for $x>4$. Therefore, the integral of this function over the unit circle is 0 by Cauchy's Theorem since it is analytic within and on the unit circle
    \begin{equation}
        \oint_{\gamma} \sqrt{z-4}dz = 0.
    \end{equation}
\end{sol}

\newpage

\begin{exer}
    A\&F 2.5.5. Evaluate the integral
    \begin{equation}
        \int_0^\infty e^{ix^2} 
    \end{equation}
    using the contour $C(R)$ which is the closed circular section in the upper half plane with boundary points $(0,0), (0,R),$ and $Re^{i\frac{\pi}{4}}$.
\end{exer}
\begin{sol}
We begin by considering the integral
\begin{equation}
    I_R = \oint_{C(R)} e^{iz^2}dz.
\end{equation}
Since the function $f(x)=e^{iz^2}$ is analytic in the entire complex plane, we have that $I_R = 0$ by Cauchy Theorem.  Breaking the contour $C(R)$ into three parts, we see that 
\begin{equation}
    \label{eq:3_parts}
    \oint_{C(R)} e^{iz^2}dz = \int_{0}^{R} e^{ix^2}dx + \int_{[Re^{i\frac{\pi}{4}},0]} e^{iz^2}dz + \int_{C_1(R)} e^{iz^2}dz. 
\end{equation}
Here, the first integral on the righthand side is the integral on the real line from $(0,R)$, the second is the integral on the line from $[Re^{i\frac{\pi}{4}},0]$, the third integral is the integral on the circular section between $R$ and $Re^{i\frac{\pi}{4}}$. We'll begin by trying to estimate the third integral.

We'll start by working with the contour $C_1(R,\theta_0)$ which is the circular sector between $Re^{i\theta_0}$ and $Re^{i\frac{\pi}{4}}$ which we'll parameterize by the function $\gamma_{R,\theta_0}(t) = Re^{it}$ for $t\in [\theta, \frac{\pi}{4}]$. Then, we can bound the third integral by
\begin{align}
\abs{\int_{C_1(R,\theta_0)} e^{iz^2}dz} &\leq \text{length}(C_1(R,\theta_0)) \cdot \sup_{z\in C_1(R,\theta_0)} \abs{e^{iz^2}}\\
                                           &= \left(\frac{\pi}{4}-\theta_0 \right)\sup_{z\in C_1(R,\theta_0)} \abs{e^{iz^2}} \\
                                           &\leq \frac{\pi}{4}\sup_{z\in C_1(R,\theta_0)} \abs{e^{iz^2}}.
\end{align}
Now we'll compute $\sup_{z\in C_1(R,\theta_0)} \abs{e^{iz^2}}$. Writing $z\in C_1(R,\theta_0)$ in polar exponential form as $z=Re^{i\theta}$ for some $\theta \in [\theta_0, \frac{\pi}{4}$, we can simplify the exponent of $f(z)$ as
\begin{align}
    iz^2 = iR^2 e^{i 2\theta} &= iR^2(\cos2\theta + i\sin2\theta)\\
                              &= -R^2(\sin2\theta + i\cos2\theta).
\end{align}

Exponentiating, we can compute $\abs{f(z)}$ as 
\begin{align}
    \abs{e^{iz^2}} = \abs{ e^{-R^2(\sin2\theta + i\cos2\theta)} } &= \abs{e^{-R^2 \sin2\theta}} \abs{e^{-iR^2\cos2\theta}}\\
                                                                  &= \abs{e^{-R^2 \sin2\theta}},
\end{align}
where $\abs{e^{-iR^2\cos2\theta}} =1$ since $R^2\cos2\theta$ is real. Since $\sin(x)> \frac{2x}{\pi}$ on $x\in[0,\frac{\pi}{2}]$, we have that 
\begin{equation}
    \sup_{z\in  C_1(R,\theta_0)} \abs{e^{iz^2}} \leq \sup_{\theta \in [\theta_0, \frac{\pi}{4}]} \abs{e^{-R^2 \frac{4\theta}{\pi}}} = e^{-R^2 \frac{4\theta_0}{\pi}}. 
\end{equation}
We obtain this suprenum since the function $e^{-R^2\frac{4\theta}{\pi}}$ is positive and monotonically decreasing in $\theta$.

Therefore, we can see that 
\begin{equation}
    \abs{\int_{C_1(R,\theta_0)} e^{iz^2}dz} \leq  \frac{\pi}{4} e^{-R^2\frac{4\theta_0}{\pi}} .
\end{equation}

We can see that since $C_1(R,\theta_0)$ becomes similar to $C_1(R)$ as $\theta_0 \to 0$ then

\begin{align}
    \lim\limits_{R\to\infty}\abs{\int_{C_1(R)} f(z)dz } = \lim\limits_{R\to\infty}\lim\limits_{\theta_0\to 0} \abs{\int_{C_1(R,\theta_0)} e^{iz^2}dz} \leq  \lim\limits_{R\to\infty}\lim\limits_{\theta_0\to 0}\frac{\pi}{4} e^{-R^2\frac{4\theta_0}{\pi}}.
\end{align}
    Since the righthand most function is monotonically decreasing and bounded by 0 in both $R$ and $\theta_0$, we can re-arrange the limits and show that 
    \begin{equation}
        \lim\limits_{R\to\infty}\abs{\int_{C_1(R)}  e^{iz^2}dz } = \lim\limits_{R\to\infty}\lim\limits_{\theta_0\to 0} \abs{\int_{C_1(R,\theta_0)} e^{iz^2}dz} = 0.
    \end{equation}
    This implies that $\lim\limits_{R\to\infty}\int_{C_1(R)}  e^{iz^2}dz = 0$ which we'll use a bit later.

    We'll now work on simplifying the integral $\int_{[Re^{i\frac{\pi}{4}},0]} e^{iz^2}dz$. We can immediately see that we can reverse the orientation of this integral and instead parameterize this integral using the curve $\gamma_2(r) = re^{i \frac{\pi}{4}}$ for $r\in[0, R]$. We can then compute
    \begin{align}
        \int_{[Re^{i\frac{\pi}{4}},0]} e^{iz^2}dz = -\int_{\gamma_2} e^{iz^2}dz.
    \end{align}
    We can simplify the righthand side by computing that
    \begin{align}
        \int_{\gamma_2} e^{iz^2}dz &= \int_0^R e^{i\gamma(r)^2}\gamma'(r)dr\\
                                    &= \int_0^R e^{i(re^{i\pi/4})^2} (e^{i \pi/4}) dr\\
                                    &=  e^{i \pi/4} \int_0^R e^{i(r^2e^{i\pi/2}} dr \\
                                    &= e^{i \pi/4} \int_0^R e^{i(ir^2)} dr\\
                                    &= e^{i \pi/4} \int_0^R e^{-r^2} dr.
    \end{align}

    Therefore, taking the limit of \ref{eq:3_parts} as $R$ approaches $\infty$, we see that $\lim_{R\to\infty} I_R = 0$, and therefore

\begin{align}
    0 = \lim\limits_{R\to\infty}\Big(\int_{-R}^{R} e^{ix^2}dx -e^{i \pi/4} \int_0^R e^{-r^2} dr +  \int_{C_1(R)}  e^{iz^2}dz \Big)
\end{align}

As shown before, the third integral on the righthand side approaches 0 in the limit, so that
\begin{align}
  0 = \lim\limits_{R\to\infty}\Big(\int_{0}^{R} e^{ix^2}dx -e^{i \pi/4} \int_0^R e^{-r^2} dr \Big).
\end{align}

To conclude, we'll use the additive property of limits and the fact that $\int_0^\infty e^{-r^2} dr = \frac{\sqrt{\pi}}{2}$. This means that
\begin{align}
  \int_{0}^{\infty} e^{ix^2}dx =  \lim\limits_{R\to\infty}\int_{0}^{R} e^{ix^2}dx &=   \lim\limits_{R\to\infty} e^{i \pi/4} \int_0^R e^{-r^2} dr \\
                                                    &= e^{i \pi/4}  \int_0^\infty e^{-r^2} dr \\
                                                    &= e^{i \pi/4}  \frac{\sqrt{\pi}}{2}.
\end{align}
This allows us to conclude that 
\begin{equation}
    \int_{0}^{\infty} e^{ix^2}dx =  e^{i \pi/4}  \frac{\sqrt{\pi}}{2}.
\end{equation}
\end{sol}

\newpage

\begin{exer}
    A\&F 2.5.6. Evaluate the integral
    \begin{equation}
        I = \int_\bbR \frac{dx}{x^2+1}
    \end{equation}
    using the contour $C(R)$ which is the closed semicircle in the upper half plane with endopoints $(-R, 0)$ and $(0,R)$.
\end{exer}
\begin{sol}
Throughout, we assume that $R>1$ for simplicity. Following the outliine of the previous, we can decompose the integral of $f(z)=1/(z^2+1)$ over $C(R)$ as follows
\begin{equation}
    \oint_{C(R)} \frac{dz}{z^2 + 1} = \int_{-R}^{R} \frac{dx}{x^2+1} + \int_{C_1(R)}  \frac{dz}{z^2 + 1}.
\end{equation}
Here, $C_1(R)$ represents the rounded section of the upper half circle which we parameter as $\gamma(t) = Re^{i\theta}$ for $\theta\in[0, \pi]$. We can then bound the integral as in the the previous problem by
\begin{equation}
    \abs{\int_{C_1(R)} \frac{dz}{z^2 + 1}} \leq \pi R \cdot \left( \sup_{z\in C_1(R)} \abs{\frac{1}{z^2 + 1}}\right)
\end{equation}
since the curve $C_1(R)$ has length $\pi R$. We can compute the suprenum that noting that each $z=Re^{i\theta}$, so that
\begin{align}
    \abs{ \frac{1}{z^2 + 1}} &\leq \frac{1}{\abs{\abs{z^2}-\abs{1}}}\\
    &= \frac{1}{\abs{\abs{R^2} - 1}}\\
    &= \frac{1}{\abs{R^2 -1}}
\end{align}
by the reverse triagle inequality and the fact that $\abs{z} = \abs{Re^{i\theta}} = \abs{R}$. This shows that
\begin{equation}
       \abs{\int_{C_1(R)} \frac{dz}{z^2 + 1}} \leq \frac{\pi R}{\abs{R^2 -1}}.
\end{equation}
Therefore, we have that 
\begin{equation}
    \lim\limits_{R\to\infty} \int_{C_1(R)} \frac{dz}{z^2 + 1} = 0.
\end{equation}
Next, we'll compute the integral $ \oint_{C(R)} \frac{dz}{z^2 + 1}$ with the residue theorem. Since the contour $C(R)$ only contains the singularity $i$, we can write the integral as 
\begin{equation}
    \oint_{C(R)} \frac{dz}{z^2 + 1} = 2\pi i \Res(f, i).
\end{equation}

We can easily compute this residue as 
\begin{equation}
    \Res(f, i) = \lim\limits_{z\to i} (z-i)(\frac{1}{z^2+1}) = \lim\limits_{z\to i} \frac{1}{z+i} = \frac{1}{2i}.
\end{equation}
Therefore, we have that
\begin{equation}
    \oint_{C(R)} \frac{dz}{z^2 + 1} = 2\pi i \Res(f, i) = \pi.
\end{equation}

Now taking the limit as $R\to\infty$, we see that
\begin{align}
    \pi = \lim\limits_{R\to \infty}\Big( \int_{-R}^{R} \frac{dx}{x^2+1} + \int_{C_1(R)}  \frac{dz}{z^2 + 1} \Big).
\end{align}
Since the rightmost integral vanishes in the limit, we have that
\begin{equation}
    \int_{-\infty}^{\infty} \frac{dx}{x^2 + 1} = \pi.
\end{equation}

\paragraph{Verification by real integration}%
\label{par:verification_by_real_integration}


We can verify this using real integration since 
\begin{equation}
    \int_{-R}^{R} \frac{dx}{x^2 + 1} = \arctan(R) - \arctan(-R).
\end{equation}
Taking the limit as $R\to \infty$, we see that 
\begin{equation}
    \int_{-\infty}^{\infty} \frac{dx}{x^2 + 1} = \lim\limits_{R\to\infty}  \int_{-R}^{R} \frac{dx}{x^2 + 1} =  \lim\limits_{R\to\infty} \Big[ \arctan(R) - \arctan(-R) \Big] = \pi/2 - (-\pi/2) = \pi.
\end{equation}
\end{sol}
\newpage

\begin{exer}
    A\&F 3.3.5.
\end{exer}

\begin{sol}
    In order to find the coeffiecients of the Taylor-Laurent Series about 0 of $f(z) = e^{\frac{t}{2}(z - z^{-1})} = \sum_{n\in\bbZ} a_n z^n$, we use the formula
    \begin{equation}
        a_n = \frac{1}{2\pi i} \int_\gamma \frac{ e^{\frac{t}{2}(z - z^{-1})}}{z^{n+1}} dz
    \end{equation}
    where $\gamma$ is the unit circle parameterized as $\gamma(\theta) = e^{i\theta}$ for $\theta \in [-\pi, \pi]$. We can then simplify the integral as
    \begin{align}
        a_n &= \frac{1}{2\pi i} \int_\gamma \frac{ e^{\frac{t}{2}(z - z^{-1})}}{z^{n+1}} dz\\
            &= \frac{1}{2\pi i} \int_{-\pi}^{\pi} \frac{e^{\frac{t}{2}( e^{i\theta} - e^{-i\theta} )}  }{ e^{i(n+1)\theta} } \cdot ie^{i\theta}d\theta.
    \end{align}
    We can simplify $e^{i\theta} - e^{-i\theta}$ as $2i\sin\theta$ and combine the terms $e^{i\theta}$ and $e^{i(n+1) \theta}$, so that
\begin{equation}
  \frac{1}{2\pi i} \int_{-\pi}^{\pi} \frac{e^{\frac{t}{2}( e^{i\theta} - e^{-i\theta} )}  }{ e^{i(n+1)\theta} } \cdot ie^{i\theta}d\theta = \frac{i}{2\pi i} \int_{-\pi}^{\pi} \frac{e^{\frac{t}{2}( 2i\sin\theta )}  }{ e^{in\theta} } d\theta  
\end{equation}

Combining the top and bottom halves of the integrand and canceling the $i$ in front, we get that
\begin{equation}
    a_n = \frac{1}{2\pi} \int_{-\pi}^\pi e^{-i(n\theta - t\sin\theta)}d\theta.
\end{equation}
We can further simplify this using $e^{-ix} = \cos x - i\sin x$, which gives us
\begin{equation}
    a_n = \frac{1}{2\pi}\int_{-\pi}^\pi \cos ( n\theta - t\sin\theta )d\theta +  \frac{i}{2\pi}\int_{-\pi}^{\pi} \sin( n\theta - t\sin\theta ) d\theta.
\end{equation}
The key to the next equality is that$g(\theta) = n\theta - t\sin\theta$ is odd. We show this directly as
\begin{align}
    g(-\theta) = -n\theta - t\sin-\theta &= - (n\theta +t\sin-\theta)\\
                                         &= - (n\theta -t\sin\theta) \quad (\sin\theta \text{ is odd })\\
                                         &=  - g(\theta).
\end{align}

Therefore, $\cos(g(\theta))$ is even and $\sin(g(\theta))$ is odd due to composition rules for odd and even functions. This means that

\begin{align}
    \int_{-\pi}^\pi \cos ( n\theta - t\sin\theta )d\theta &= 2 \int_{0}^\pi \cos ( n\theta - t\sin\theta )d\theta \\
    \int_{-\pi}^{\pi} \sin( n\theta - t\sin\theta ) d\theta &=  0
\end{align}
by integral theorems for even and odd functions. Therefore,
\begin{equation}
    a_n = \frac{1}{2\pi} \int_{-\pi}^\pi e^{-i(n\theta - t\sin\theta)}d\theta = \frac{1}{\pi} \int_{0}^\pi \cos ( n\theta - t\sin\theta )d\theta.
\end{equation}
\end{sol}

\end{document}
