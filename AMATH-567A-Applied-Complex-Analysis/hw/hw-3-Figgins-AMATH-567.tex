%Preamble
\documentclass[12pt]{article}
\usepackage{fancyhdr}
\usepackage{extramarks}
\usepackage{amsmath}
\usepackage{amssymb}
\usepackage{amsthm}
\usepackage{amsrefs}
\usepackage{amsfonts}
\usepackage{mathrsfs}
\usepackage{mathtools}
\usepackage[mathcal]{eucal} %% changes meaning of \mathcal
\usepackage{enumerate}
\usepackage[shortlabels]{enumitem}
\usepackage{verbatim} %% includes comment environment
\usepackage{hyperref}
\usepackage[capitalize]{cleveref}
\crefformat{equation}{~(#2#1#3)}
\usepackage{caption, subcaption}
\usepackage{graphicx}
\usepackage{fullpage} %%smaller margins
\usepackage[all,arc]{xy}
\usepackage{mathrsfs}

\hypersetup{
    linktoc=all,     % set to all if you want both sections and subsections linked
}

\topmargin=-0.45in
\evensidemargin=0in
\oddsidemargin=0in
\textwidth=6.5in
\textheight=9.0in
\headsep=0.25in
\setlength{\headheight}{16pt}

\linespread{1.1}

\pagestyle{fancy}
\lhead{\Name}
\chead{\hwTitle}
\rhead{\hwClass}
\lfoot{\lastxmark}
\cfoot{\thepage}

\renewcommand\headrulewidth{0.4pt}
\renewcommand\footrulewidth{0.4pt}

\setlength\parindent{0pt}

%% Title Info
\newcommand{\hwTitle}{HW \# 3}
\newcommand{\hwDueDate}{\today}
\newcommand{\hwClass}{AMATH 567}
\newcommand{\hwClassTime}{}
\newcommand{\hwClassInstructor}{}
\newcommand{\Name}{\textbf{Marlin Figgins}}


%% MATH MACROS
\newcommand{\bbF}{\mathbb{F}}
\newcommand{\bbN}{\mathbb{N}}
\newcommand{\bbQ}{\mathbb{Q}}
\newcommand{\bbR}{\mathbb{R}}
\newcommand{\bbZ}{\mathbb{Z}}
\newcommand{\bbC}{\mathbb{C}}
\newcommand{\abs}[1]{ \left| #1 \right| }
\newcommand{\diff}[2]{\frac{d #1}{d #2}}
\newcommand{\infsum}[1]{\sum_{#1}^{\infty}}
\newcommand{\norm}[1]{ \left|\left| #1 \right|\right| }
\newcommand{\eval}[1]{ \left. #1 \right| }
\newcommand{\Expect}[1]{\mathbb{E}\left[#1 \right]}
\newcommand{\Var}[1]{\mathbb{V}\left[#1 \right]}
\newcommand{\Res}{\text{Res}}

\renewcommand{\phi}{\varphi}
\renewcommand{\emptyset}{\O}
\renewcommand{\Im}{\text{Im}}
\renewcommand{\Re}{\text{Re}}

%--------Theorem Environments--------
%theoremstyle{plain} --- defaultx
\newtheorem{thm}{Theorem}[section]
\newtheorem{cor}[thm]{Corollary}
\newtheorem{prop}[thm]{Proposition}
\newtheorem{lem}[thm]{Lemma}
\newtheorem{conj}[thm]{Conjecture}
\newtheorem{quest}[thm]{Question}

\theoremstyle{definition}
\newtheorem{defn}[thm]{Definition}
\newtheorem{defns}[thm]{Definitions}
\newtheorem{con}[thm]{Construction}
\newtheorem{exmp}[thm]{Example}
\newtheorem{exmps}[thm]{Examples}
\newtheorem{notn}[thm]{Notation}
\newtheorem{notns}[thm]{Notations}
\newtheorem{addm}[thm]{Addendum}

% Environments for answers and solutions
\newtheorem{exer}{Exercise}
\newtheorem{sol}{Solution}

\theoremstyle{remark}
\newtheorem{rem}[thm]{Remark}
\newtheorem{rems}[thm]{Remarks}
\newtheorem{warn}[thm]{Warning}
\newtheorem{sch}[thm]{Scholium}

\makeatletter
\let\c@equation\c@thm
\makeatother

\begin{document}

\begin{exer}
    \leavevmode
    
    (c) Evaluate 
    \begin{equation*}
        \int_0^\infty \frac{dx}{(x^2+a^2)(x^2+b^2)}, \quad a^2, b^2 > 0.
    \end{equation*}

    (d) Evaluate
    \begin{equation*}
        \int_0^\infty \frac{dx}{x^6+1}. 
    \end{equation*}
\end{exer}
\begin{sol}
    \leavevmode
    
    (c) We'll begin by considering the curve $C(R)$ which is the upper semi-circle in $\bbC$ centered at 0 with radius $R$. We can divide this into two parts $[-R,R]$ and $C_R$ which is the circular section of the semi-circle. We can then write
    \begin{equation}
    \label{eq:1c-decomp}
        \int_{C(R)}  \frac{dz}{(z^2+a^2)(z^2+b^2)} =   \int_{-R}^{R} \frac{dx}{(x^2+a^2)(x^2+b^2)} +   \int_{C_R}  \frac{dz}{(z^2+a^2)(z^2+b^2)}.
    \end{equation}
    Theorem 4.2.1. in A\&F shows that that $f(z) = \frac{1}{(z^2+a^2)(z^2+b^2)}$ has 
    \begin{equation*}
        \lim\limits_{R\to\infty}  \int_{C_R}  \frac{dz}{(z^2+a^2)(z^2+b^2)} = 0
    \end{equation*}
    since $f(z)$ is rational and the degree of the denominator (4) exceeds the degree of the numerator $(0)$ by more than two. Therefore, taking the limit as of $R\to\infty$ of equation \cref{eq:1c-decomp}, we see that
    \begin{equation}
        \lim\limits_{R\to\infty}  \int_{C(R)}  \frac{dz}{(z^2+a^2)(z^2+b^2)} =  \int_{-\infty}^{\infty} \frac{dx}{(x^2+a^2)(x^2+b^2)} = 2 \int_{0}^\infty \frac{dx}{(x^2+a^2)(x^2+b^2)},
    \end{equation}
    where the last equation follows from the fact the integrand is even. We can compute the leftmost integral using the Residue theorem. The function $\frac{1}{(z^2+a^2)(z^2+b^2)}$ has singularities at $\pm ia$ and $\pm ib$. Since we're dealing with the upper half circle, only two of this of these will be contained within our contour for large enough $R$. These are $z_1 = ia$ and $z_2 = ib$ assuming $a,b>0$. Using the fact that the numerator of the integrand is analytic and that the zeros of the denominator are simple, we can write the residues as
    \begin{equation}
        \Res(f;z_0) = \frac{1}{2z_0(z_0^2 + a^2)  +2z_0 (z_0^2 + b^2)}.
    \end{equation}
Evaluating at $z_1$ and $z_2$, we see that
\begin{align}
    \Res(f; ia) = \frac{1}{2ia(2a^2) + 2ia (a^2 + b^2) } = \frac{1}{2ai(3a^2 + b^2)}\\
    \Res(f; ib) = \frac{1}{2ib(2b^2) + 2ib (a^2 + b^2) } = \frac{1}{2bi(a^2 +3b^2)}
\end{align}

Therefore, we can compute the final integral as
\begin{align}
    \int_{0}^\infty \frac{dx}{(x^2+a^2)(x^2+b^2)} &= \pi i ( \Res(f; ia) +  \Res(f; ib)   )\\
                                                  &= \frac{\pi}{2} \left( \frac{1}{a(3a^2 + b^2)} + \frac{1}{b(a^2 +3b^2)}\right)\\
                                                  &= \frac{\pi}{2} \left( \frac{1}{3a^3 + ab^2} + \frac{1}{a^2b +3b^3} \right).
\end{align}
    %TODO: Finish reference for formula. Double check

\newpage

    (d) We consider the same general approach as in the previous part. We keep the definitions of $C(R)$ and $C_R$ the same. Since $f(z) = \frac{1}{z^6+1}$ also satisfies the conditions of Theorem 4.2.1, we have that 
    \begin{equation*}
        \lim\limits_{R\to\infty}  \int_{C_R}  \frac{dz}{z^6+1} = 0.
    \end{equation*}
    Additionally, our integrand is even again, so that 
    \begin{equation}
            \lim\limits_{R\to\infty}  \int_{C(R)} \frac{dz}{z^6+1}   = 2 \int_{0}^\infty \frac{dz}{z^6+1}. 
    \end{equation}
    Once again, we can compute the integral on the left using the residue theorem. The integrand $ \frac{1}{z^6+1}$ has singularities at $e^{i\pi/6}, e^{i\pi/2}, e^{i5\pi/6}$ in the upper half plane. Therefore, the integral can be written as
\begin{equation}
    \lim\limits_{R\to\infty}  \int_{C(R)} \frac{dz}{z^6+1} = 2\pi i( \Res(f;e^{i\pi/6}) + \Res(f; e^{i\pi/2}) + \Res(f;  e^{i5\pi/6}) ). 
\end{equation}
Since the poles of $ \frac{1}{z^6+1}$ are simple, we can compute the residues using the fact that the numerator is analytic and the denominator has only simple zero at $z_0$ as
\begin{equation}
    \Res(f; z_0) = \frac{1}{6z_0^5} 
\end{equation}
as in section 5.4 of Prof. Tung's notes. Therefore, 
\begin{equation}
    \Res(f;e^{i\pi/6}) = \frac{1}{6e^{i5\pi/6}},\quad \Res(f; e^{i\pi/2}) = \frac{1}{6e^{i5\pi/2}},\quad \Res(f; e^{i 5\pi/6}) = \frac{1}{6 e^{i25\pi/6}}. 
\end{equation}
We can then compute the desired integral as
\begin{align}
    \int_{0}^\infty \frac{dz}{z^6+1} &= \frac{\pi i}{6} \left( \frac{1}{e^{i5\pi/6}} +  \frac{1}{e^{i5\pi/2}} +  \frac{1}{e^{i25\pi/6}} \right)\\
                                     &=  \frac{\pi i}{6}\left(e^{-5\pi/6} + e^{-\pi/2} + e^{-i\pi/5}  \right)\\ 
                                     &= \frac{\pi i}{6} \left(-2i \right) =  \frac{\pi}{3}. 
\end{align}
Above, we used that $e^{-5\pi i /2} = e^{-i\pi/2} = -i$, $e^{-i 25\pi / 6} = e^{-i\pi/6} = -\frac{\sqrt{3}}{2} - i/2$ and $e^{-i5\pi/6} =\frac{\sqrt{3}}{2} - i/2$.
%TODO: Double check. Find formula reference. Section 5.4. from Tung's manuscript. pg. 46.
\end{sol}

\newpage

\begin{exer}
    Evaluate the following integrals

    (a) \begin{equation*}
        \int_{-\infty}^{\infty} \frac{x\sin x}{x^2+a^2}dx, \quad a^2 > 0. 
    \end{equation*}

    (b) \begin{equation*}
            \int_{-\infty}^{\infty} \frac{\cos(kx)dx}{(x^2+a^2)(x^2+b^2)}, \quad a^2, b^2, k > 0. 
        \end{equation*}

    (h) \begin{equation*}
        \int_0^{2\pi} \frac{d\theta}{(5-3\sin\theta)^2}. 
    \end{equation*}
\end{exer}

\begin{sol}
\leavevmode
(a) Once again, we'll work in the upper half plane and use the contours $C(R), [-R, R],$ and $C_R$. Instead of working directly with the integral at hand, we'll instead using the integrand $f(z) = \frac{ze^{-z}}{z^2+a^2}$ since
\begin{equation}
    \Im\left(\int_{-\infty}^{\infty} \frac{z e^{iz}}{z^2+a^2}dz \right) =  \int_{-\infty}^{\infty} \frac{x\sin x}{x^2+a^2}dx.
\end{equation}

Here we can use Jordan's Lemma since $g(z) = \frac{z}{z^2 + a^2}$ converges to 0 uniformly as $z\to\infty$, so that we see that $\int_{C_R}  \frac{z e^{iz}}{z^2+a^2}dz \to 0$ as $R\to\infty$. It follows that 
\begin{equation}
    \lim\limits_{R\to\infty} \int_{C(R)} \frac{z e^{iz}}{z^2+a^2}dz =  \int_{-\infty}^{\infty} \frac{z e^{iz}}{z^2+a^2}dz =  2\pi i \sum_j \Res(f;z_j).
\end{equation} We can then compute the residues in the upper half plane at the singularity $z=ia$ for $a>0$ since it is the sole isolated singularity of $f$ in the upper half plane. Since the numerator is analytic and the denominator has only simple poles, we can compute the residue as 
\begin{equation}
    \Res(f; ia) = \left(\frac{z_0 e^{iz_0}}{2z_0}\right)_{z_0=ia} = \frac{iae^{i (ia)}}{2ia} = \frac{e^{-a}}{2}. 
\end{equation}
Plugging this in,
\begin{equation}
    \int_{-\infty}^{\infty} \frac{z e^{iz}}{z^2+a^2}dz =  2\pi i  \frac{e^{-a}}{2} = \pi i e^{-a}.
\end{equation}
Taking the imaginary part of this, we conclude 
\begin{equation}
     \int_{-\infty}^{\infty} \frac{x\sin x}{x^2+a^2}dx =  \pi e^{-a}.
\end{equation}

\newpage

(b) We use the integrand $\frac{e^{ikz}}{(z^2+a^2)(z^2+b^2)}$ and same contours as above. Since the norm of this function is the same as the integrand in (1c) because $\abs{e^{ikz}} = 1$, we know that 
\begin{equation}
    \int_{C_R} \frac{e^{ikz}dz}{(z^2+a^2)(z^2+b^2)} \to 0 
\end{equation}
as $R\to \infty$. We could alternatively use Jordan's Lemma. Next, using the residue theorem and the fact that $\int_{C(R)} = \int_{C_R} + \int_{-R}^{R}$, we can compute that for $f(z) = \frac{e^{ikz}}{(z^2+a^2)(z^2+b^2)}$
\begin{equation}
    \int_{-\infty}^{\infty} \frac{e^{ikz}dz}{(z^2+a^2)(z^2+b^2)} = 2\pi i \left( \Res \left(f; ia \right)  +  \Res \left( f ; ib \right) \right). 
\end{equation}

Since the numerator of this is analytic and the denominator has only simple poles, we can compute the residues here using the formula
\begin{equation}
    \Res(f, z_0) = \frac{e^{ikz}}{2z_0(z_0^2 + a^2) + 2z_0 (z_0^2 + b^2)}
\end{equation}
as in Section 5.4 of Prof. Tung's notes. Since the only two singularities in the upper half circle are $ia, ib$ for $a,b>0$, we can then compute that the integral is
\begin{equation}
    \int_{-\infty}^{\infty} \frac{e^{ikz}dz}{(z^2+a^2)(z^2+b^2)} = \pi \left( \frac{e^{-ka}}{3a^3 + ab^2} + \frac{e^{ikb}}{a^2b +3b^3} \right).
\end{equation}

The real part of this integral is our desired integral since the above quantity is entirely real, we just have that
\begin{equation}
    \int_{-\infty}^{\infty} \frac{e^{ikz}dz}{(z^2+a^2)(z^2+b^2)} = \int_{-\infty}^{\infty} \frac{\cos(kx) dx}{(x^2+a^2)(x^2+b^2)}
\end{equation}
Therefore, our final answer
\begin{equation}
    \int_{-\infty}^{\infty} \frac{\cos(kx) dx}{(x^2+a^2)(x^2+b^2)} = \pi \left( \frac{e^{-ka}}{3a^3 + ab^2} + \frac{e^{-kb}}{a^2b +3b^3} \right).
\end{equation}

\newpage
(h)  %TODO: Proceed as in Tung's book. Section 6.4. pg. 54.
To solve this integral, we'll write $\frac{1}{(5-3\sin\theta)^2}$ in terms of complex exponentials, so that
\begin{equation}
    U(\theta) = \frac{1}{(5-3\sin\theta)^2} = \frac{1}{(5- \frac{3}{2i}  (z-z^{-1}))^2} = U(z) .
\end{equation}

We can then re-write the desired integral in terms of a contour integral on the unit circle $C$
\begin{equation}
    \int_0^{2\pi} U(\theta)d\theta = \oint_C U(z) \frac{dz}{iz} = 2\pi i \sum_j \Res\left( \frac{U}{iz}; z_j  \right),
\end{equation}
where $z_j$ are the isolated singularities of $\frac{U}{iz}$  in the unit circle $C$. Therefore,
\begin{equation}
      \int_0^{2\pi} \frac{1}{(5-3\sin\theta)^2} d\theta = \oint_C \frac{1}{(5- \frac{3}{2i}  (z-z^{-1}))^2} \frac{dz}{iz}. 
\end{equation}
We can simplify the integrand of the right hand side as 
\begin{align}
    \frac{U(z)}{iz} &= \frac{1}{iz}\cdot \frac{1}{(5- \frac{3}{2i}  (z-z^{-1}))^2} =   \frac{-iz}{z^2(5-\frac{3}{2i}(z-z^{-1}))^2}\\
                    &= \frac{-iz}{(5z - \frac{3}{2i}(z^2-1))^2} \\ 
                    &= \frac{-iz}{(-\frac{3}{2i}z^2 + 5z + \frac{3}{2i} )^2}\\
                    &= \frac{-iz}{(\frac{1}{2}( 1 + 3iz )(z-3i))^2} \\
                    &= \frac{-4iz}{(1+3iz)^2(z-3i)^2} \\
                    &= \frac{-4iz}{(3i)^2(z-i/3)^2(z-3i)^2}\\ 
                    &= \frac{4i}{9(z-i/3)^2(z-3i)^2} 
\end{align}
We can compute the zeroes of the denominator as $\frac{i}{3}$ and $3i$. Since only $\frac{i}{3}$ is in the unit circle, we can compute the integral as
\begin{equation}
\int_0^{2\pi} U(\theta)d\theta = 2\pi i \Res\left( \frac{U}{iz}; \frac{i}{3}  \right )
\end{equation}
In this case, the residue can be computed using the double pole formula
\begin{align}
    \Res\left( \frac{U}{iz}; \frac{i}{3}  \right ) &= \lim\limits_{z\to i/3} \frac{d}{dz} \left[  \frac{U}{iz} \cdot \left(z-\frac{i}{3} \right)^2 \right]\\
                                                   &= \lim\limits_{z\to i/3} \frac{d}{dz} \left[ \frac{4i}{9(z-i/3)^2(z-3i)^2}  \cdot  \left(z-\frac{i}{3} \right)^2  \right] \\
                                                   &=  \frac{4i}{9} \lim\limits_{z\to i/3}\frac{d}{dz} \left[ \frac{z}{(z-3i)^2} \right]\\
                                                   &=\frac{4i}{9} \lim\limits_{z\to i/3} -\frac{(z+3i)}{(z-3i)^3}\\
                                                   &= -\frac{4i}{9} \frac{10i/3}{512i/27} \\
                                                   &= - \frac{40i}{512} = -i\frac{5}{64}
\end{align}
We now finish computing the integral as
\begin{equation}
    \int_0^{2\pi} U(\theta)d\theta = 2\pi i  \left(-i\frac{5}{64}\right) = \frac{5\pi}{32}.
\end{equation}
\end{sol}

\newpage

\begin{exer}
    Use a sector contour $C$ with radius $R$ centered at the origin with angles $0\leq \theta\leq \frac{2\pi}{5}$ to find for $a>0$,
    \begin{equation*}
       I =  \int_0^\infty \frac{dx}{x^5+a^5} = \frac{\pi}{5a^4\sin(\pi/5)}.  
    \end{equation*}
\end{exer}

\begin{sol}
    We'll break down the contour $C$ mentioned above in three parts, so that $\oint_C=\int_{C_x} + \int_{C_R} + \int_{C_L}$. Let $C_L$ is the straight line between $Re^{i 2\pi/5}$ and $0$, $C_R$ be the circular section, and $C_x$ be the line segment from $[0, R]$ on the real axis. We then have that
\begin{equation}
   \int_{0}^{R}  \frac{dx}{x^5+a^5} +   \int_{C_R}  \frac{dz}{z^5+a^5} +  \int_{C_L}  \frac{dz}{z^5+a^5} = 2\pi i \sum_j \Res \left(  \frac{1}{z^5+a^5}; z_j  \right) 
\end{equation}
by the residue theorem. By theorem 4.2.1 of A\&F, 
\begin{equation}
    \int_{C_R}  \frac{dz}{z^5+a^5} \to 0,
\end{equation} as $R\to\infty$. We can also make the substitution on $C_L$ that $z = xe^{2\pi i/5}$ for $x\in [0,R]$, so that
\begin{equation}
    \int_{C_L}  \frac{dz}{z^5+a^5} = \int_{R}^{0} \frac{e^{2\pi i/5}}{x^5+a^5}dx = - e^{2\pi i/5}  \int_{0}^{R}  \frac{dx}{x^5+a^5}. 
\end{equation}
Since the only pole of $\frac{1}{z^5+a^5}$ inside $C$ is $z_0 = ae^{i\pi/5}$ for $R$ large enough, we have that in the limit as $R\to\infty$
\begin{equation}
    \int_0^\infty \frac{dx}{x^5+a^5} = \frac{2\pi i}{1 - e^{2\pi i/5}} \Res \left(  \frac{1}{z^5+a^5};  ae^{i\pi/5} \right). 
\end{equation}
All that remains is to compute the residue, which is given by
\begin{equation}
    \Res \left(  \frac{1}{z^5+a^5}; ae^{i\pi/5} \right) = \left( \frac{1}{5z^4} \right)_{z_0} = \frac{1}{5a^4} e^{-4\pi i / 5}.
\end{equation}
Combining the last two equations allows us to conclude that
\begin{align}
    \int_0^\infty \frac{dx}{x^5+a^5} &= \frac{2\pi i}{1 - e^{2\pi i/5}}\frac{1}{5a^4} e^{-4\pi i / 5}\\
                                     &= \frac{\pi}{5a^4}\left( \frac{2i e^{-4\pi i/5}}{1 - e^{2\pi i/5}}\right)\\
                                     &= \frac{\pi}{5a^4}\left( \frac{2i}{e^{-\pi i/5} - e^{\pi i/5}}\right) e^{-\pi i}\\
                                     &= \frac{\pi}{5a^4}\left( \frac{2i}{e^{\pi i/5} - e^{-\pi i/5}}\right)\\ 
                                     &=  \frac{\pi}{5a^4 \sin \pi/5}
\end{align}
where we have multiplied by $e^{-\pi i /5} /e^{-\pi i /5} =1$ in the third line and simplified $e^{-i\pi} = -1$ in the fourth and used that $\sin x =(e^{ix} - e^{-iz} )/2i$ in the last.
\end{sol}

\newpage

\begin{exer}
    A function that is analytic for all $z\in\bbC$ is called entire. 

    \begin{enumerate}[(a)]
        \item Show that any bounded entire function is necessarily constant.
        \item Suppose $f(z)$ is an entire function, not necessarily, bounded, but such that $\Im(f(z))\leq 0$. Show that $f(z)$ is necessarily constant.
    \end{enumerate}
\end{exer}

\begin{sol}
    (a) We'll show that the derivative of a bounded entire function must be 0 on $\bbC$. Suppose that the $\abs{f(z)}\leq M$ for all $z\in\bbC$. For fixed $a\in \bbC$, $f(z)$ is analytic in and on the circle of radius $R$ centered at $a$ since $f(z)$ is entire. Therefore, we can use Cauchy's bound for derivatives (page 30 in Prof. Tung's book), to see that
    \begin{equation}
        \abs{f'(a)} \leq \frac{M}{R}. 
    \end{equation}
    Since this holds for any circle of radius $R$, we see that $\abs{f'(a)} = 0$. Since this holds for arbitrary $a\in\bbC$, we know that $f'(z) = 0$ for all $z\in\bbC$. This means the function $f(z)$ must be constant.
    \newpage


    (b) Since the function $f(z)$ has $\Im( f(z)) \leq 0$, the function $f(z)-i$ has $\Im(f(z)-i) \leq -1$ for all $z\in\bbC$ and therefore it is non-zero on the entire complex plane. It follows that the function 
   \begin{equation}
       h(z) = \frac{1}{f(z)-i} 
   \end{equation} 
   is entire since it is reciprocal of a non-zero entire function. This function $h(z)$ is bounded since
   \begin{align}
       \abs{h(z)} &= \frac{1}{\abs{f(z)-i}}\\
                  &= \frac{1}{\Re(f(z) -i)^2 + \Im(f(z) - i)^2 }\\
        &\leq \frac{1}{\Re(f(z)-i)^2 + 1} \leq 1,
   \end{align}
   where we've used that $\abs{z} = \Re(z)^2 + \Im(z)^2$ and $ \Im(f(z)-i)^2 \geq 1$. By part $a$, it follows that $h(z)$ is constant over $\bbC$ i.e. $h(z) = c$ for some constant $c\in\bbC$. We can compute that $f(z) = \frac{1}{c} + i$, so $f(z)$ is constant over $\bbC$.
\end{sol}

\end{document}
