%Preamble
\documentclass[12pt]{article}
\usepackage{fancyhdr}
\usepackage{extramarks}
\usepackage{amsmath}
\usepackage{amssymb}
\usepackage{amsthm}
\usepackage{amsrefs}
\usepackage{amsfonts}
\usepackage{mathrsfs}
\usepackage{mathtools}
\usepackage[mathcal]{eucal} %% changes meaning of \mathcal
\usepackage{enumerate}
\usepackage[shortlabels]{enumitem}
\usepackage{verbatim} %% includes comment environment
\usepackage{hyperref}
\usepackage[capitalize]{cleveref}
\crefformat{equation}{~(#2#1#3)}
\usepackage{caption, subcaption}
\usepackage{graphicx}
\usepackage{fullpage} %%smaller margins
\usepackage[all,arc]{xy}
\usepackage{mathrsfs}

\hypersetup{
    linktoc=all,     % set to all if you want both sections and subsections linked
}

\topmargin=-0.45in
\evensidemargin=0in
\oddsidemargin=0in
\textwidth=6.5in
\textheight=9.0in
\headsep=0.25in
\setlength{\headheight}{16pt}

\linespread{1.1}

\pagestyle{fancy}
\lhead{\Name}
\chead{\hwTitle}
\rhead{\hwClass}
\lfoot{\lastxmark}
\cfoot{\thepage}

\renewcommand\headrulewidth{0.4pt}
\renewcommand\footrulewidth{0.4pt}

\setlength\parindent{0pt}

%% Title Info
\newcommand{\hwTitle}{HW \# 7}
\newcommand{\hwDueDate}{November 24, 2020}
\newcommand{\hwClass}{AMATH 567}
\newcommand{\hwClassTime}{}
\newcommand{\hwClassInstructor}{}
\newcommand{\Name}{\textbf{Marlin Figgins}}


%% MATH MACROS
\newcommand{\bbF}{\mathbb{F}}
\newcommand{\bbN}{\mathbb{N}}
\newcommand{\bbQ}{\mathbb{Q}}
\newcommand{\bbR}{\mathbb{R}}
\newcommand{\bbZ}{\mathbb{Z}}
\newcommand{\bbC}{\mathbb{C}}
\newcommand{\abs}[1]{ \left| #1 \right| }
\newcommand{\diff}[2]{\frac{d #1}{d #2}}
\newcommand{\infsum}[1]{\sum_{#1}^{\infty}}
\newcommand{\norm}[1]{ \left|\left| #1 \right|\right| }
\newcommand{\eval}[1]{ \left. #1 \right| }
\newcommand{\Expect}[1]{\mathbb{E}\left[#1 \right]}
\newcommand{\Var}[1]{\mathbb{V}\left[#1 \right]}
\newcommand{\Res}{\text{Res}}

\renewcommand{\phi}{\varphi}
\renewcommand{\emptyset}{\O}
\renewcommand{\Im}{\text{Im}}
\renewcommand{\Re}{\text{Re}}

%--------Theorem Environments--------
%theoremstyle{plain} --- defaultx
\newtheorem{thm}{Theorem}[section]
\newtheorem{cor}[thm]{Corollary}
\newtheorem{prop}[thm]{Proposition}
\newtheorem{lem}[thm]{Lemma}
\newtheorem{conj}[thm]{Conjecture}
\newtheorem{quest}[thm]{Question}

\theoremstyle{definition}
\newtheorem{defn}[thm]{Definition}
\newtheorem{defns}[thm]{Definitions}
\newtheorem{con}[thm]{Construction}
\newtheorem{exmp}[thm]{Example}
\newtheorem{exmps}[thm]{Examples}
\newtheorem{notn}[thm]{Notation}
\newtheorem{notns}[thm]{Notations}
\newtheorem{addm}[thm]{Addendum}

% Environments for answers and solutions
\newtheorem{exer}{Exercise}
\newtheorem{sol}{Solution}

\theoremstyle{remark}
\newtheorem{rem}[thm]{Remark}
\newtheorem{rems}[thm]{Remarks}
\newtheorem{warn}[thm]{Warning}
\newtheorem{sch}[thm]{Scholium}

\makeatletter
\let\c@equation\c@thm
\makeatother

\begin{document}

\begin{exer}
    (a) Let $\hat{f}(s)$ and $\hat{g}(s)$ be the Laplace transforms of one-sided function $f(t)$ and $g(t)$ respectively. Show that the inverse Laplace transform of $\hat{f}(s)\hat{g}(s)$ is
\begin{equation*}
    \int_{0}^{t} f(t-\tau) g(\tau) d\tau.
\end{equation*}

(b) Use Laplace transform and the result in (a) to solve the following ordinary differential equation
\begin{equation*}
    \frac{d^{2}}{dt^{2}} y + 4y = f(t),
\end{equation*}
subject to the initial conditions: $y(0)=0$,  $\frac{dy}{dt}(0) = y'(0) = 0$.
\end{exer}
 
\begin{sol}\leavevmode
    (a) We write the definition of the inverse Laplace transform as 
    \begin{align*}
        \mathcal{L}^{-1}([ \hat{f}(s)\hat{g}(s)] &= \frac{1}{2\pi i} \int_{L} e^{st}   \hat{f}(s)\hat{g}(s) ds,
    \end{align*}
    where $L$ is chosen to be to the right of any singularities of $\hat{f}$ and $\hat{g}$ as discussed in class. Writing the definition of $\hat{g}$ we see that 
    \begin{align*}
        \mathcal{L}^{-1}[  \hat{f}(s)\hat{g}(s)] &= \frac{1}{2\pi i} \int_{L} e^{st}   \hat{f}(s)\hat{g}(s) ds\\
                                              &= \frac{1}{2\pi i} \int_{L} e^{st}   \hat{f}(s) \left(  \int_{0}^{\infty} e^{-s\tau } g(\tau) d\tau \right)ds.
    \end{align*}

    Re-arranging the integrals and using the definition of the inverse Laplace transform, we have that
    \begin{align*}
        \mathcal{L}^{-1}[  \hat{f}(s)\hat{g}(s)]  &= \frac{1}{2\pi i} \int_{0}^{\infty} g(\tau) \int_{L} e^{ s(t-\tau)} \hat{f}(s)ds d\tau\\
                                                  &= \frac{1}{2\pi i} \int_{0}^{\infty} g(\tau) \int_{L} e^{ s(t-\tau)} \hat{f}(s)ds d\tau\\
                                                  &= \int_{0}^{\infty} g(\tau) \mathcal{L}^{-1}[\hat{f}(s)](t-\tau)\\
                                                  &=  \int_{0}^{\infty} g(\tau) f(t-\tau) d\tau\\
    \end{align*}
    Since the $f$ and $g$ are one sided, we have that $f(t-\tau) = 0$ when $\tau > t$ and $g(\tau) = 0$ for $\tau < 0$. Therefore,
    \begin{equation*}
    \mathcal{L}^{-1}[  \hat{f}(s)\hat{g}(s)] = \int_{0}^{t} g(\tau) f(t-\tau) d\tau
    \end{equation*}

    \newpage 

    (b) We begin by taking the Laplace transform of both sides of the equation

    \begin{align*}
        \mathcal{L}[ y'' + 4y  ] = \mathcal{L}[ y'' ] +  4\mathcal{L}[y]  =\mathcal{L}[f(t)].
    \end{align*}

    Using the transform formulas for derivatives and initial conditions, we have that
    \begin{equation*}
        \mathcal{L}[ y'' ] = s^{2} \hat{y} - s y(0) - y'(0) = s^{2} \hat{y}.
    \end{equation*}
    Therefore, we have that 
    \begin{align*}
    (s^{2} + 4) \hat{y} = \hat{f}(s) \implies \hat{y} = \frac{\hat{f}(s)}{s^{2} + 4}.
    \end{align*}

    This means our desired solution $y(t)$ is given by
    \begin{equation*}
        y(t) = \mathcal{L}^{-1}\left[ \hat{f}(s) \cdot \frac{1}{s^{2}+4} \right]
    \end{equation*}
    We can solve this using the formula derived in part (a). Setting $\hat{g}(s) = \frac{1}{s^{2}+4}$, we solve first for this function's inverse transform. Looking at pg. 90 of Prof. Tung's book, we can see that 
    \begin{equation*}
        \mathcal{L}\left[\frac{\sin(2t)}{2}\right] = \frac{1}{s^{2} + 4} \implies g(t) = \frac{\sin(2t)}{2}.
    \end{equation*}

    Using the formula derived in (a), we have that
      \begin{align*}
        y(t) &= \mathcal{L}^{-1}\left[ \hat{f}(s) \cdot \frac{1}{s^{2}+4} \right]\\
             &= \frac{1}{2}\int_{0}^{t} f(t-\tau) \sin(2\tau) d\tau.
    \end{align*}


\end{sol}

\newpage

\begin{exer}
    Solve the following Laplace equation
    \begin{equation*}
    \frac{\partial^{2}}{\partial x^{2}} \phi + \frac{\partial^{2}}{\partial y^{2}} \phi = 0 
    \end{equation*}
    in the upper half plane subject to the boundary conditions: $\phi\to 0$ as $y\to \infty$ and $\phi\to 0$ as $x\to \pm \infty$ and 
    \begin{equation*}
        \phi(x,0) = \frac{x}{x^{2} + a^{2}}.
    \end{equation*}
  \end{exer}

\begin{sol}
    Assuming a solution $\phi(x,y)$ to this equation exists, we take its Fourier transform in $x$, so that our differential equation satisfies
\begin{align*}
    -\mathcal{F}\left[ \frac{\partial^{2}}{\partial x^{2}} \phi \right] =  \mathcal{F} \left[\frac{\partial^{2}}{\partial y^{2}} \phi\right]  
\end{align*}

By the differentiation rules of Fourier series, we have that
\begin{equation*}
  \mathcal{F}\left[\frac{\partial^{2}}{\partial x^{2}}  \phi \right]  = - \lambda^{2} \mathcal{F}[ \phi ].
\end{equation*}

Therefore, our transformed differential equation satisfies
\begin{equation*}
    \lambda^{2} \mathcal{F}[ \phi ] = \mathcal{F} \left[\frac{\partial^{2}}{\partial y^{2}} \phi\right] = \frac{\partial^{2}}{\partial y^{2}} \mathcal{F} \left[\phi\right], 
\end{equation*}
where we've interchanged the $y$ derivative and our integration in the Fourier transform. For simplicity, we'll set $\mathcal{F}[\phi] = \hat{\phi}$, so that we see
\begin{align*}
    \lambda^{2} \hat{\phi} =  \frac{\partial^{2}}{\partial y^{2}}  \hat{\phi} \implies 
    \hat{\phi}(\lambda, y) = A(\lambda) e^{\lambda y} + B(\lambda) e^{- \lambda y}
\end{align*}
for functions $A$ and $B$ which depend on $\lambda$. Since we require that $\phi \to 0$ as $y\to \infty$ and $\phi$ is bounded for $y>0$, this means that $A = 0$ for $\lambda > 0$ and $B = 0$ for $\lambda < 0$, so we write
\begin{equation*}
    \hat{\phi}(\lambda, y) = C(\lambda) e^{- \abs{\lambda}y},
 \end{equation*}
 for some function $C(\lambda)$. Using that $\phi(x, 0) = \frac{x}{x^{2} + a^{2}} = f(x)$, we have that 
 \begin{equation*}
     \hat{\phi}(\lambda, 0) = \hat{f}(\lambda) = C(\lambda) e^{-\abs{\lambda}\cdot 0}.
 \end{equation*}
 This shows that our transformed solution satisfies $C(\lambda) = \hat{f}(\lambda)$, so that
 \begin{equation*}
  \hat{\phi}(\lambda, y) = \hat{f}(\lambda) e^{- \abs{\lambda}y}.
 \end{equation*}
 To return our solution to the desired $(x,y)$ coordinates, we'll use the convolution theorem for Fourier transforms. As shown on page 286 of A\& F, we have that the inverse transform of $e^{-\abs{\lambda}y}$ is given by $g(x,y) = \frac{1}{\pi} \frac{y}{x^{2}+ y^{2}}$. Therefore, our solution is given by the convolution theorem for Fourier transforms as

 \begin{align*}
     \phi(x,y) &= \int_{-\infty }^{\infty } f(t) g(x-t, y) dt\\
               &= \frac{1}{\pi} \int_{-\infty }^{\infty }  f(t) \cdot \frac{y}{(x-t)^{2}+ y^{2}} dt\\
               &= \frac{1}{\pi} \int_{-\infty }^{\infty }  \frac{t}{t^{2} + a^{2}} \cdot \frac{y}{(x-t)^{2}+ y^{2}} dt\\
               &= \frac{y}{\pi} \int_{-\infty }^{\infty } \frac{t}{(t^{2} + a^{2})((x-t)^{2}+ y^{2})}dt. 
 \end{align*}
 We can solve this integral with contour integration. Let $\Gamma$ be the contour for the upper half circle, and for $h(t) = \frac{t}{(t^{2} + a^{2})((x-t)^{2}+ y^{2})}$ we have that
 \begin{equation*}
     \int_{\Gamma} h(t) dt = \int_{C_{R}} h(t) dt + \int_{-R}^{R} h(t) dt.
 \end{equation*}
 As $h(t)$ is a rational function with denominator degree at least 2 greater than the numerator degree, we have that the integral over $C_{R}$ goes to 0 in the limit as $R\to \infty$. This means that
 \begin{equation*}
     \int_{\Gamma} h(t) dt = \int_{-\infty }^{\infty } h(t) \text{ as  } R\to \infty .
 \end{equation*}
 We can solve for this integral using the residue theorem. In the upper half plane where $\Gamma$ is defined, $h(t)$ has singularities at $ia$ and $iy + x$ in the upper half plane as $a>0$ and $y > 0$. Therefore, this shows that

 \begin{equation*}
     \int_{-\infty }^{\infty } h(t)dt = 2\pi i( \Res(h; ia)  + \Res(h; iy + x)).
 \end{equation*}
 Since each of the singularities are simple poles, we can write a formula for the Residues using the derivative of the denominator, so that

 \begin{equation*}
     \Res(h; t_{0}) = \frac{t}{2t( (x-t)^{2} + y^{2} ) - 2(t^{2} + a^{2})(x-t)}.
 \end{equation*}

 Plugging in the singularity $t_{0} = ia$, we have that
 \begin{align*}
     \Res(h; ia) &= \frac{ia}{2ia((x-ia)^{2} + y^{2}) - 2((ia)^{2} + a^{2})(x-ia)}\\
                    &= \frac{ia}{2ia((x-ia)^{2} + y^{2})}\\ 
                    &= \frac{1}{2((x-ia)^{2} + y^{2})}. 
 \end{align*}
 Next, for $t_{0} = iy + x$, we have that
 \begin{align*}
     \Res(h; iy + x) &= \frac{iy + x}{2(iy + x) ( [x - (iy+x)]^{2} + y^{2} ) - 2((iy + x)^{2} + a^{2})( x - (iy + x) )}\\
                     &=- \frac{iy + x}{2((iy + x)^{2} + a^{2})(-iy)}\\
                     &= \frac{x}{2iy((iy + x)^{2} + a^{2})} + \frac{1}{2 ((iy+x)^{2} + a^{2})}
 \end{align*}

 We have that 
 \begin{align*}
     \int_{-\infty }^{\infty } h(t)dt &= 2\pi i(\frac{1}{2((x-ia)^{2} + y^{2})} +  \frac{x}{2iy((iy + x)^{2} + a^{2})} + \frac{1}{2 ((iy+x)^{2} + a^{2})})\\
                                      &= \pi ( \frac{i}{(x-ia)^{2} + y^{2}}  + \frac{x}{y((iy + x)^{2} + a^{2})} + \frac{i}{(iy+x)^{2} + a^{2}}) 
 \end{align*}
Plugging this into the integral solution for $\phi$, we have
\begin{align*}
  \phi(x,y) &= \frac{yi}{(x-ia)^{2} + y^{2}}  + \frac{x}{(iy + x)^{2} + a^{2}} + \frac{yi}{(iy+x)^{2} + a^{2}}.
\end{align*}
As a simple check, we can see that this solution satisfies our initial condition when $y=0$.
\end{sol}

Happy Thanksgiving! Thank you for your thorough grading.
\end{document}
