%Preamble
\documentclass[12pt]{article}
\usepackage{fancyhdr}
\usepackage{extramarks}
\usepackage{amsmath}
\usepackage{amssymb}
\usepackage{amsthm}
\usepackage{amsrefs}
\usepackage{amsfonts}
\usepackage{mathrsfs}
\usepackage{mathtools}
\usepackage[mathcal]{eucal} %% changes meaning of \mathcal
\usepackage{enumerate}
\usepackage[shortlabels]{enumitem}
\usepackage{verbatim} %% includes comment environment
\usepackage{hyperref}
\usepackage[capitalize]{cleveref}
\crefformat{equation}{~(#2#1#3)}
\usepackage{caption, subcaption}
\usepackage{graphicx}
\usepackage{fullpage} %%smaller margins
\usepackage[all,arc]{xy}
\usepackage{mathrsfs}

\hypersetup{
    linktoc=all,     % set to all if you want both sections and subsections linked
}

\topmargin=-0.45in
\evensidemargin=0in
\oddsidemargin=0in
\textwidth=6.5in
\textheight=9.0in
\headsep=0.25in
\setlength{\headheight}{16pt}

\linespread{1.1}

\pagestyle{fancy}
\lhead{\Name}
\chead{\hwTitle}
\rhead{\hwClass}
\lfoot{\lastxmark}
\cfoot{\thepage}

\renewcommand\headrulewidth{0.4pt}
\renewcommand\footrulewidth{0.4pt}

\setlength\parindent{0pt}

%% Title Info
\newcommand{\hwTitle}{HW \# 5}
\newcommand{\hwDueDate}{Novemeber 11, 2020}
\newcommand{\hwClass}{AMATH 567}
\newcommand{\hwClassTime}{}
\newcommand{\hwClassInstructor}{}
\newcommand{\Name}{\textbf{Marlin Figgins}}


%% MATH MACROS
\newcommand{\bbF}{\mathbb{F}}
\newcommand{\bbN}{\mathbb{N}}
\newcommand{\bbQ}{\mathbb{Q}}
\newcommand{\bbR}{\mathbb{R}}
\newcommand{\bbZ}{\mathbb{Z}}
\newcommand{\bbC}{\mathbb{C}}
\newcommand{\abs}[1]{ \left| #1 \right| }
\newcommand{\diff}[2]{\frac{d #1}{d #2}}
\newcommand{\infsum}[1]{\sum_{#1}^{\infty}}
\newcommand{\norm}[1]{ \left|\left| #1 \right|\right| }
\newcommand{\eval}[1]{ \left. #1 \right| }
\newcommand{\Expect}[1]{\mathbb{E}\left[#1 \right]}
\newcommand{\Var}[1]{\mathbb{V}\left[#1 \right]}
\newcommand{\Res}{\text{Res}}

\renewcommand{\phi}{\varphi}
\renewcommand{\emptyset}{\O}
\renewcommand{\Im}{\text{Im}}
\renewcommand{\Re}{\text{Re}}

%--------Theorem Environments--------
%theoremstyle{plain} --- defaultx
\newtheorem{thm}{Theorem}[section]
\newtheorem{cor}[thm]{Corollary}
\newtheorem{prop}[thm]{Proposition}
\newtheorem{lem}[thm]{Lemma}
\newtheorem{conj}[thm]{Conjecture}
\newtheorem{quest}[thm]{Question}

\theoremstyle{definition}
\newtheorem{defn}[thm]{Definition}
\newtheorem{defns}[thm]{Definitions}
\newtheorem{con}[thm]{Construction}
\newtheorem{exmp}[thm]{Example}
\newtheorem{exmps}[thm]{Examples}
\newtheorem{notn}[thm]{Notation}
\newtheorem{notns}[thm]{Notations}
\newtheorem{addm}[thm]{Addendum}

% Environments for answers and solutions
\newtheorem{exer}{Exercise}
\newtheorem{sol}{Solution}

\theoremstyle{remark}
\newtheorem{rem}[thm]{Remark}
\newtheorem{rems}[thm]{Remarks}
\newtheorem{warn}[thm]{Warning}
\newtheorem{sch}[thm]{Scholium}

\makeatletter
\let\c@equation\c@thm
\makeatother

\begin{document}

\begin{exer}
    Evaluate the integrals 
    \begin{equation}
        \frac{1}{2\pi i} \oint_C f(z)dz,
    \end{equation}
    where $C$ is the unit circle centered at the origin with $f(z)$ given below. Do these problems by both
    \begin{enumerate}[(i)]
        \item enclosing the singular points in side $C$
        \item enclosing the singular points outside $C$ (by including the point at infinity)
    \end{enumerate}
    Show that you obtain the same result in both cases.
    \begin{enumerate}[(a)]
        \item  $ \frac{z^2+1}{z^2-a^2}, \quad a^2 <1$.
        \item $ \frac{z^2 + 1}{z^3} $.
        \item $ z^2e^{-1/z}$.
    \end{enumerate}
    Hint: the point at infinity is defined as $t = 1/z$ as $z\to 0$.
\end{exer}
 
\begin{sol}\leavevmode

    (a) We'll begin by computing the integral of $f(z) = \frac{z^2+1}{z^2-a^2} = \frac{(z-i)(z-i)}{(z-a)(z+a)}$. Notice that both $\pm a$ are in the unit circle $C$ since $a^2$ is less than 1 and are the only singularities in $C$. Therefore, the integral
    \begin{equation}
        \frac{1}{2\pi i} \oint_{C} f(z)dz = \Res(f; a) + \Res(f;-a).
    \end{equation}
Since both of these residues are taken at simple poles, we can compute them easily as
\begin{align}
    \Res(f;a) &=\lim_{z\to a} (z-a)f(z) = \lim_{z\to a} \frac{z^2+1}{z+a} = \frac{a^2 +1}{2a}\\
    \Res(f;-a) &=\lim_{z\to -a} (z+a)f(z) = \lim_{z\to -a} \frac{z^2 + 1}{z-a} = -\frac{a^2+1}{2a}.  
\end{align}
We then conclude that 
\begin{equation}
     \frac{1}{2\pi i} \oint_{C}  \frac{z^2+1}{z^2-a^2} dz = 0.
\end{equation}

Since the function $f(z)$ has no singularities outside the unit circle (other than at $\infty$), we can alternatively compute that 
\begin{align}
        \frac{1}{2\pi i} \oint_{C} f(z)dz = -\Res(f; \infty).
\end{align}
Using that the point at infinity is defined by $t = 1/z$ as $z\to 0$, we have that 
\begin{equation}
    \Res(f;\infty) = - \Res(f(z^{-1})z^{-2};0)
\end{equation}
as shown in class. Therefore, we compute the residue of 
\begin{equation}
    f(z^{-1})z^{-2} = \frac{z^{-2} + 1}{z^{-2} + a^2} \cdot z^{-2} = \frac{z^{-2}}{1+a^2z^2} + \frac{1}{1+a^2z^2}.   
\end{equation}
This is the sum of a function which is analytic at 0 ($\frac{1}{1+a^2z^2}$) and another which has a double pole at 0. Since the coefficients of the Taylor-Laurent series are additive, we can ignore the contribution of the analytic function, so that 
\begin{align}
    \Res(f(z^{-1})z^{-2};0) = \Res \left(\frac{z^{-2}}{1+a^2z^2};0\right) &= \lim_{z\to 0} \frac{d}{dz} \left( z^2 \cdot \frac{z^{-2}}{1+a^2z^2} \right)\\
                                                                          &= \lim_{z\to 0} \frac{d}{dz} \left( \frac{1}{1+a^2z^2} \right)\\
                                                                          &= \lim_{z\to 0}\frac{-2a^2z}{1 + a^2z^2} = 0.
\end{align}
Once again, this allows us to conclude that
\begin{equation}
     \frac{1}{2\pi i} \oint_{C}  \frac{z^2+1}{z^2-a^2} dz = 0.
\end{equation}

\newpage 

(b) We'll now repeat this procedure but with the function $f(z) = \frac{z^2+1}{z^3} = z^{-3} + z^{-1}$. The only singularities of this function in $C$ occur at the point 0 and since this is already written in form of its Taylor-Laurent series about 0, we have that 
\begin{equation}
    \frac{1}{2\pi i} \oint_C  \frac{z^2+1}{z^3} = \Res\left(\frac{z^2+1}{z^3}; 0\right) = a_{-1} = 1.
\end{equation}

We can now compute this same integral using the singularity at 0 by noting that 
\begin{align}
    \frac{1}{2\pi i} \oint_{C} \frac{z^2+1}{z^3} dz = \Res(f(z^{-1})z^{-2};0).
\end{align}
In this case,
\begin{equation}
    f(z^{-1})z^{-2} = (z^{3} + z^{1}) z^{-2} = z + z^{-1}.
\end{equation}
Computing the residue of this function at 0 is simple since it is in terms of its Taylor-Laurent series, we see that 
\begin{equation}
    \frac{1}{2\pi i} \oint_{C} \frac{z^2+1}{z^3} dz = \Res(f(z^{-1})z^{-2};0)= 1.
\end{equation}

\newpage

(c) We now switch to the function 
\begin{equation}
    f(z) = z^2 e^{-1/z} = z^2 \left(1 - z^{-1} + \frac{z^{-2}}{2!} - \frac{z^{-3}}{3!} + \cdots   \right) = z^2 - z + \frac{1}{2} - \frac{z^{-1}}{3!} + \cdots,
\end{equation}
where we have expanded $e^{-1/z}$ in terms of its Taylor-Laurent series about 0. Using the fact the only singularity of $f(z)$ in $C$ is 0, we can compute that
\begin{equation}   
    \frac{1}{2\pi i} \oint_{C} z^2 e^{-1/z} dz = \Res( z^2 e^{-1/z} ;0) = - \frac{1}{3!} = -1/6 .
\end{equation}

We can now solve this problem by instead using the singularity at $\infty$. We see that
\begin{align}
    f(1/z) 1/z^2 &= (z^{-2} e^{-z} )\cdot z^{-2} = z^{-4}e^{-z}\\
                 &= z^{-4}\left(1 - z + \frac{z^2}{2!} - \frac{z^3}{3!} + \cdots \right)\\
                 &= z^{-4} - z^{-3} + \frac{z^{-2}}{2!} - \frac{z^{-1}}{3!} + \cdots,
\end{align}
where we have expanded $e^{-z}$ in terms of its Taylor-Laurent series about 0. With this expansion, we can clearly see $\Res(f(z)z^{-2};0) = -1/6$ and conclude
\begin{equation}
    \frac{1}{2\pi i} \oint_{C}  z^2 e^{-1/z}dz = \Res(f(z^{-1})z^{-2};0) = - 1/6.
\end{equation}
\end{sol}

\newpage

\begin{exer}
    Find the Fourier transform of 
    \begin{equation}
        f(t) = \begin{cases}
            1, \quad t\in(-a,a)\\
            0, \quad \text{ otherwise}.
        \end{cases}
    \end{equation}
    Then, do the inverse transform using techniques of contour integration, e.g. Jordan's lemma, principal values, etc.
\end{exer}

\begin{sol}
    We begin by doing the Fourier transform of $f$, so that
    \begin{equation}
        F(\lambda) = \int_{-\infty}^{\infty} e^{-i\lambda t} f(t) dt = \int_{-a}^{a} e^{-i\lambda t}dt,
    \end{equation}
    since the integrand is 0 for $t$ such that $\abs{t} \geq a$ and just $e^{i\lambda t}$ for $\abs{t}<a$. We can now evaluate this integral using the anti-derivative of an exponential, so that
    \begin{align}
        F(\lambda) &= \int_{-a}^{a} e^{-i\lambda t}dt\\
                   &= -\frac{e^{-i\lambda t}}{i\lambda} \vert_{-a}^{a}\\
                   &= -\frac{1}{i\lambda} \left(e^{-i\lambda a} - e^{i\lambda a} \right ) \\
                   &= \frac{2}{\lambda}\sin(\lambda a). 
    \end{align}

    We'll now take the inverse Fourier transform of this function as 
    \begin{equation}
        \hat f(t) = \frac{1}{2\pi} \int_{-\infty}^{\infty} e^{i\lambda t} F(\lambda) d\lambda. 
    \end{equation}

    We'll start by writing this integral in terms of exponentials 
    \begin{align}
        \hat f(t) &= \frac{1}{2\pi i}  \int_{-\infty}^{\infty} e^{i\lambda t} \frac{(e^{i\lambda a} - e^{-i\lambda a} )}{\lambda} d\lambda\\
                  &= \frac{1}{2\pi i}  \int_{-\infty}^{\infty} \frac{e^{i\lambda (t+a)} - e^{i\lambda (t - a)}}{\lambda} d\lambda \\
                  &= 
                  \frac{1}{2\pi i} \left( \int_{-\infty}^{\infty} \frac{e^{i z (t+a)}}{z} dz -  \int_{-\infty}^{\infty} \frac{e^{iz (t-a)}}{z} dz \right)  \\
    \end{align}

    \paragraph{Case 1: $t\in(-a, a)$}
    
    We'll begin by integrating the first integral. Since $ t > -a$, we have that the first integral disappears on $C_R$ the upper semicircle centered at $0$ with radius $R\to \infty$. That is,
    \begin{equation}
        \int_\Gamma \frac{e^{i z (t+a)}}{z} dz = \left(\int_{-R}^{-\epsilon} + \int_{C_\epsilon} + \int_{\epsilon}^{R} + \int_{C_R}\right) \frac{e^{i z (t+a)}}{z} dz, 
    \end{equation}
    where $C_\epsilon$ is the upper half circle centered at 0 with radius $\epsilon$ and oriented clockwise and $\Gamma$ the contour consisting of the contours on the right hand side of the integral. By Cauchy's theorem, we have that $\int_\Gamma = 0$. We can compute the integral over $C_\epsilon$ using 
    \begin{equation}
        \int_{C_\epsilon}\frac{e^{i z (t+a)}}{z} dz = -\pi i \Res\left(\frac{e^{i z (t+a)}}{z},0\right) = -\pi i\text{ as  } \epsilon \to 0.
    \end{equation}
    We can compute this residue by simply using the simple pole formula from Prof. Tung's notes. Therefore, in the limit, we have 
    \begin{equation}
        \int_{-\infty}^{\infty}\frac{e^{i z (t+a)}}{z} dz = \pi i
    \end{equation}

    For the second integral, we cannot use Jordan's lemma since $t-a < 0$. We can instead do the same kind of integration after flipping the contour we use to integrate about the real axis. Therefore, using the mirrored contours, we have that 
    \begin{equation}
    0 = \int_\Gamma \frac{e^{i z (t-a)}}{z} dz  = \left(\int_{-R}^{-\epsilon} + \int_{C_\epsilon} + \int_{\epsilon}^{R} + \int_{C_R}\right) \frac{e^{i z (t-a)}}{z} dz,
    \end{equation}
    where once again the integral over $C_R$ disappears since we now use the lower half semi-circle. This leaves us to compute the integral as 
    \begin{equation}
        \int_{-\infty}^{\infty} \frac{e^{i z (t-a)}}{z} dz = -\pi i,
    \end{equation}
    where we have used the same methods for calculating the residue. The only difference is the sign of the integral over $C_\epsilon$ which is reversed due to the reversal of orientation of $C_\epsilon$ we get from reflecting about the real axis. That is, $C_\epsilon$ has become the lower half circle about 0 with radius $\epsilon$, but is now oriented counterclockwise, which gives us the flipped sign of our integral. The other integrals are left unchanged by this reflection. Therefore, we can compute that for $t\in (-a,a )$

    \begin{equation}
       \hat f(t) = \frac{1}{2\pi i}  \int_{-\infty}^{\infty} e^{i\lambda t} \frac{(e^{i\lambda a} - e^{-i\lambda a} )}{\lambda} d\lambda = \frac{1}{2\pi i} (\pi i - (-\pi i)) = 1. 
    \end{equation}

    \paragraph{Case 2: $\abs{t} > a$}%
    \label{par:case_2_t_a_}
   In this case, there are two sub-cases $t < -a < a$ and $-a < a < t$. In either cases, we can repeat the analysis done above but with both integrals having the same contour $\Gamma$. Due to both integrands having the same residue at $0$, this then means that 
   \begin{equation}
      I =  \int_{-\infty}^{\infty} \frac{e^{i z (t-a)}}{z} dz = \int_{-\infty}^{\infty} \frac{e^{i z (t-a)}}{z} dz .
   \end{equation}
   Therefore, for $t$ with $\abs{t} > a$,
    \begin{equation}
       \hat f(t) = \frac{1}{2\pi i}  \int_{-\infty}^{\infty} e^{i\lambda t} \frac{(e^{i\lambda a} - e^{-i\lambda a} )}{\lambda} d\lambda = \frac{1}{2\pi i} (I - I) = 0. 
    \end{equation}

    \paragraph{Case 3: $t = \pm a$}%
    In the case that $t = a$, we have to compute the integral
    \begin{align}
        \hat f(a) &= \frac{1}{2\pi i} \int_{-\infty}^{\infty} \frac{e^{i2az}-1}{z} dz \\
             &= \frac{1}{\pi} \int_{-\infty}^{\infty} e^{aiz}\frac{e^{iaz}-e^{-iaz}}{2iz}dz \\
             & = \frac{1}{\pi} \int_{-\infty}^{\infty} e^{aiz} \frac{\sin (ax)}{x}\\
             &= \frac{1}{\pi}  \left(\int_{-\infty}^{\infty} \frac{\cos(ax)\sin(ax)}{x}dx +  i\int_{-\infty}^{\infty} \frac{\sin^2(ax)}{x}dx \right)\\
             &=  \frac{1}{\pi} \int_{-\infty}^{\infty} \frac{\cos(ax)\sin(ax)}{x}dx
    \end{align}
    The function $\frac{\sin^2(ax)}{x}dx$ is odd, so its integral goes to 0. Translating back to complex exponentials, we see that $\cos(ax)\sin(ax) = (1/4i) (e^{i2ax} - e^{-i2ax}) = \sin(2ax)/2$. Therefore, we have that
    \begin{equation}
    \hat f(a) = \frac{1}{2\pi} \int_{-\infty}^{\infty} \frac{\sin 2ax}{x} dx = 1/2,  
    \end{equation}
    where we have used that $\int_{-\infty}^{\infty} \sin tx / x dx = \pi$ for $t>0$ as shown in Prof. Tung's book.

  Now we let $t=-a$, we solve the integral
\begin{equation}
   \hat f(-a) = \frac{1}{2\pi i} \int_{-\infty}^{\infty} \frac{1 - e^{-i2az}}{z} dz.    
\end{equation}
We can follow essentially the same arithmetic to see that
\begin{align}
   \hat f(-a) &= \frac{1}{2\pi i} \int_{-\infty}^{\infty} e^{-iaz}\frac{e^{iaz} - e^{-iaz}}{z} dz\\
          &= \frac{1}{\pi}  \left(\int_{-\infty}^{\infty} \frac{\cos(ax)\sin(ax)}{x}dx - i \int_{-\infty}^{\infty} \frac{\sin^2(ax)}{x}dx \right)\\
          &= \frac{1}{\pi} \int_{-\infty}^{\infty} \frac{\cos(ax)\sin(ax)}{x}dx.
\end{align}
Therefore, $\hat f(a) = \hat f(-a) = 1/2$. We can see that this is the same as our original function $f(t)$ for all $t\neq \pm a$.
\end{sol}

%TODO: Mention that the a_{-1} coefficient of the TL expansion is the Residue 

\end{document}
