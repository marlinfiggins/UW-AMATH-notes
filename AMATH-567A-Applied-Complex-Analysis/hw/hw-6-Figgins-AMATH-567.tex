%Preamble
\documentclass[12pt]{article}
\usepackage{fancyhdr}
\usepackage{extramarks}
\usepackage{amsmath}
\usepackage{amssymb}
\usepackage{amsthm}
\usepackage{amsrefs}
\usepackage{amsfonts}
\usepackage{mathrsfs}
\usepackage{mathtools}
\usepackage[mathcal]{eucal} %% changes meaning of \mathcal
\usepackage{enumerate}
\usepackage[shortlabels]{enumitem}
\usepackage{verbatim} %% includes comment environment
\usepackage{hyperref}
\usepackage[capitalize]{cleveref}
\crefformat{equation}{~(#2#1#3)}
\usepackage{caption, subcaption}
\usepackage{graphicx}
\usepackage{fullpage} %%smaller margins
\usepackage[all,arc]{xy}
\usepackage{mathrsfs}

\hypersetup{
    linktoc=all,     % set to all if you want both sections and subsections linked
}

\topmargin=-0.45in
\evensidemargin=0in
\oddsidemargin=0in
\textwidth=6.5in
\textheight=9.0in
\headsep=0.25in
\setlength{\headheight}{16pt}

\linespread{1.1}

\pagestyle{fancy}
\lhead{\Name}
\chead{\hwTitle}
\rhead{\hwClass}
\lfoot{\lastxmark}
\cfoot{\thepage}

\renewcommand\headrulewidth{0.4pt}
\renewcommand\footrulewidth{0.4pt}

\setlength\parindent{0pt}

%% Title Info
\newcommand{\hwTitle}{HW \# 6}
\newcommand{\hwDueDate}{Novemeber 18, 2020}
\newcommand{\hwClass}{AMATH 567}
\newcommand{\hwClassTime}{}
\newcommand{\hwClassInstructor}{}
\newcommand{\Name}{\textbf{Marlin Figgins}}


%% MATH MACROS
\newcommand{\bbF}{\mathbb{F}}
\newcommand{\bbN}{\mathbb{N}}
\newcommand{\bbQ}{\mathbb{Q}}
\newcommand{\bbR}{\mathbb{R}}
\newcommand{\bbZ}{\mathbb{Z}}
\newcommand{\bbC}{\mathbb{C}}
\newcommand{\abs}[1]{ \left| #1 \right| }
\newcommand{\diff}[2]{\frac{d #1}{d #2}}
\newcommand{\infsum}[1]{\sum_{#1}^{\infty}}
\newcommand{\norm}[1]{ \left|\left| #1 \right|\right| }
\newcommand{\eval}[1]{ \left. #1 \right| }
\newcommand{\Expect}[1]{\mathbb{E}\left[#1 \right]}
\newcommand{\Var}[1]{\mathbb{V}\left[#1 \right]}
\newcommand{\Res}{\text{Res}}

\renewcommand{\phi}{\varphi}
\renewcommand{\emptyset}{\O}
\renewcommand{\Im}{\text{Im}}
\renewcommand{\Re}{\text{Re}}

%--------Theorem Environments--------
%theoremstyle{plain} --- defaultx
\newtheorem{thm}{Theorem}[section]
\newtheorem{cor}[thm]{Corollary}
\newtheorem{prop}[thm]{Proposition}
\newtheorem{lem}[thm]{Lemma}
\newtheorem{conj}[thm]{Conjecture}
\newtheorem{quest}[thm]{Question}

\theoremstyle{definition}
\newtheorem{defn}[thm]{Definition}
\newtheorem{defns}[thm]{Definitions}
\newtheorem{con}[thm]{Construction}
\newtheorem{exmp}[thm]{Example}
\newtheorem{exmps}[thm]{Examples}
\newtheorem{notn}[thm]{Notation}
\newtheorem{notns}[thm]{Notations}
\newtheorem{addm}[thm]{Addendum}

% Environments for answers and solutions
\newtheorem{exer}{Exercise}
\newtheorem{sol}{Solution}

\theoremstyle{remark}
\newtheorem{rem}[thm]{Remark}
\newtheorem{rems}[thm]{Remarks}
\newtheorem{warn}[thm]{Warning}
\newtheorem{sch}[thm]{Scholium}

\makeatletter
\let\c@equation\c@thm
\makeatother

\begin{document}

\begin{exer}
    Evaluate the integrals 
    \begin{equation}
        \frac{1}{2\pi i} \oint_C f(z)dz,
    \end{equation}
    where $C$ is the unit circle centered at the origin with $f(z)$ given below.    

    \begin{enumerate}[(a)]
        \item  $ \frac{z+1}{2z^3 - 3z^2 -2z} $.
        \item $ \frac{\cosh(z^{-1})}{z}$.
        \item $ \frac{e^{-\cosh(z)}}{4z^2 + \pi^2}$.
        \item $\frac{\ln(z+2)}{2z+1}, \quad -\pi < \arg(z+2)\leq \pi$ 
        \item $\frac{z + z^{-1}}{z (2z - (2z)^{-1})}$
    \end{enumerate}
\end{exer}
 
\begin{sol}\leavevmode

    (a) We begin by factoring the denominator of the integrand
    \begin{equation*}
        f(z) = \frac{z+1}{2z^3 - 3z^2 -2z} = \frac{z+1}{z(2z+1)(z-2)}.
    \end{equation*}
    This function has singularities in $C$ at $z= -1/2, 0$, but is otherwise analytic in $C$. By the Residue theorem, computing the integral is as simple as noting that
\begin{equation*}
    \frac{1}{2\pi i} \oint_C f(z)dz = \Res(f; -1/2) + \Res(f; 0).
\end{equation*}
    All of the poles of this function are simple, so we can compute the residues as follows
    \begin{align*}
        \Res(f; -1/2) &= \lim_{z \to -1/2} (z + \frac{1}{2})\left(\frac{z+1}{z(2z+1)(z-2)}\right) \\
                      &= \lim_{z \to -1/2} \frac{z+1}{2z(z-2)} = - 1 / 5 
    \end{align*}

    We can compute the other residue as 
    \begin{align*}
        \Res(f;0) &= \lim_{z \to 0} (z )\left(\frac{z+1}{z(2z+1)(z-2)}\right) \\
                  &= \lim_{z \to 0} \frac{z+1}{(2z+1)(z-2)}  = - 1 / 2  
    \end{align*}
    We can now conclude that

    \begin{equation*}
     \frac{1}{2\pi i} \oint_C \frac{z+1}{2z^3 - 3z^2 -2z} dz = - 7  / 10
    \end{equation*}

    \newpage

    (b) We note that the only singularity of $f(z) = \frac{\cosh(z^{-1})}{z}$ in  $C$ occurs at 0, so our desired integral is equal to $\Res(f, 0)$. Writing the numerator in terms of its Taylor-Laurent series around 0, we see that
     \begin{equation*}
         \cosh(z^{-1}) = \sum_{n=0}^{\infty} \frac{z^{-2n}}{(2n)!},
    \end{equation*}
    we then have that 
    \begin{equation*}
        f(z) = \frac{\cosh(z^{-1})}{z} = \sum_{n=0}^{\infty} \frac{z^{-2n-1}}{(2n)!}.
    \end{equation*}
    From this, we can see that the residue is given by the coefficient of this series when $n=0$, so that 
    \begin{equation*}
        \frac{1}{2\pi i} \oint_C \frac{\cosh(z^{-1})}{z} dz = \Res(f;0) = 1 
    \end{equation*}

    \newpage

    (c) The only singularities of  
\begin{equation*}
    f(z) = \frac{e^{-\cosh(z)}}{4z^2 + \pi^2} = \frac{e^{-\cosh(z)}}{4(z-i\frac{\pi}{2})(z + i \frac{\pi}{2})},
\end{equation*}
are given by $\pm i \frac{\pi}{2}, 0$. None of these occur in $C$. Therefore, the function is analytic within and on $C$ and we have
\begin{equation*}
        \frac{1}{2\pi i} \oint_C \frac{e^{-\cosh(z)}}{4z^2 + \pi^2} dz = 0
    \end{equation*}

    \newpage

    (d) Since we've restricted to $-\pi < \arg(z+2) \leq \pi$, the function $f(z) = \frac{\ln(z+2)}{2z+1}$ only has a singularity at $z = -\frac{1}{2}$.  Therefore, the integral over $C$ can be computed using only the residue of this function at $-\frac{1}{2}$. We compute this as 
    \begin{align*}
        \Res(f; -1/2) &= \lim_{z\to -1/2} (z +1/2) \left(\frac{\ln(z+2)}{2z+1} \right)\\
                      &= \lim_{z\to -1/2} \frac{\ln(z+2)}{2} = \frac{\ln(3/2)}{2}. 
    \end{align*}

    Therefore, our integral is simply
    \begin{equation*}
            \frac{1}{2\pi i} \oint_C \frac{\ln(z+2)}{2z+1} dz = \Res(f; -1/2) = \frac{\ln(3/2)}{2}. 
    \end{equation*}

    \newpage

    (e) We begin by simplifying the integrand
    \begin{equation*}
        f(z)= \frac{z + z^{-1}}{z (2z - (2z)^{-1})} = \frac{z+z^{-1}}{2z^2 - \frac{1}{2}} = \frac{z + z^{-1}}{2 (z- \frac{1}{2})(z+\frac{1}{2})}.
    \end{equation*}
    We can then see that this function has singularities at $0, \pm\frac{1}{2}$, all of which are contained in $C$. We can then compute the residues at these various points using that they are all simple poles. Beginning with $\frac{1}{2}$, 
\begin{align*}
    \Res(f; 1/2) &= \lim_{z\to 1/2} (z-1/2) \left(\frac{z + z^{-1}}{2 (z- \frac{1}{2})(z+\frac{1}{2})}\right)\\
                 &= \lim_{z\to 1/2} \frac{z + z^{-1}}{2 (z+\frac{1}{2})}\\
                 &= 5 / 4
\end{align*}

Next, 
\begin{align*}
    \Res(f; -1/2) &= \lim_{z\to -1/2} (z+1/2) \left(\frac{z + z^{-1}}{2 (z- \frac{1}{2})(z+\frac{1}{2})}\right)\\
                 &= \lim_{z\to -1/2} \frac{z + z^{-1}}{2 (z-\frac{1}{2})}\\
                 &= 5 / 4
\end{align*}

We can compute the final residue as 
\begin{align*}
    \Res(f; 0) &= \lim_{z\to 0} z \left(\frac{z}{2z^2 - \frac{1}{2}} + \frac{z^{-1}}{2z^2 - \frac{1}{2}}  \right)\\
               &= \lim_{z\to 0} \left(\frac{z^{2}}{2z^2 - \frac{1}{2}} + \frac{1}{2z^2 - \frac{1}{2}}  \right)\\
               &= 0 + 2.
\end{align*}
We can then compute the integral as 
\begin{equation*}
    \frac{1}{2\pi i} \oint_C f(z)dz = \Res(f; 1/2) + \Res(f; -1/2) + \Res(f;0) = 9 / 2.
    \end{equation*}
\end{sol}

\newpage

\begin{exer}
    Show that 
    \begin{equation*}
        I = \int_{0}^{\infty} \frac{\sin x}{x(x^2+1)} dx = \frac{\pi}{2} (1 - e^{-1})
    \end{equation*}
\end{exer}

\begin{sol}
    We instead will work with the integrand $f(z) = \frac{e^{iz}}{z(z^{2} + 1)}$ and compute the integral
\begin{equation*}
    J = \int_{-\infty}^{\infty} \frac{e^{iz}}{z(z^{2} + 1)} dz
\end{equation*}
since $\Im(J) = I$. We will compute the integral $J$ with the contour $\Gamma = C_\epsilon + C_R + [-R, -\epsilon] + [\epsilon, R]$, where $C_\epsilon$ is the upper half circle centered at 0 with radius $\epsilon$ oriented clockwise and $C_{R}$ the upper half circle centered at 0 with radius $R$ and oriented counter-clockwise. We then have that
\begin{equation*}
    \int_{\Gamma} f(z) dz = \int_{C_{\epsilon}} f(z)dz +  \int_{\epsilon}^{R} f(z)dz +  \int_{C_{R}} f(z)dz +  \int_{-R}^{-\epsilon} f(z)dz. 
\end{equation*}

We can compute the integral of the left hand side by Residue theorem since the contour $\Gamma$ contains the singularity of $f(z)$ at $i$, so that 
\begin{equation*}
    \int_{\Gamma} f(z) dz = 2\pi i \Res(f, i).
\end{equation*}
We can compute this residue as 
\begin{align*}
    \Res(f;i) &= \lim_{z\to i} (z - i)  \left( \frac{e^{iz}}{z(z + i)(z - i)}\right)\\
              &= \lim_{z\to i} \frac{e^{iz}}{z(z + i)}\\
              &= -\frac{e^{-1}}{2}.
\end{align*}
Therefore, 
\begin{equation*}
-\pi i e^{-1} = \int_{C_{\epsilon}} f(z)dz +  \int_{\epsilon}^{R} f(z)dz +  \int_{C_{R}} f(z)dz +  \int_{-R}^{-\epsilon} f(z)dz. 
\end{equation*}
As $\epsilon \to 0$, we can compute the integral over $C_\epsilon$ as
\begin{equation*}
    \int_{C_{\epsilon}} f(z)dz = -\pi i \Res(f,0).
\end{equation*}
We can compute this residue as 
\begin{align*}
    \Res(f;0) &= \lim_{z\to 0} z \cdot  \frac{e^{iz}}{z(z^{2} + 1)}\\
              &= \lim_{z\to 0} \frac{e^{iz}}{z^{2} + 1} = 1.
\end{align*}
Therefore in the limit as $\epsilon\to 0$, we have 
\begin{equation*}
    \pi i(1 - e^{-1})= -\pi i e^{-1}  + \pi i = \int_{-R}^{R} f(z)dz +  \int_{C_{R}} f(z)dz. 
\end{equation*}
By Jordan's lemma, the integral over $C_R$ goes to 0 in the limit as $R\to \infty$, which leaves us with
\begin{equation*}
    \pi i (1 - e^{-1})= \int_{-\infty}^{\infty} f(z)dz. 
\end{equation*}
We now note that this implies
\begin{equation*}
  \int_{- \infty }^{\infty} \frac{\sin x}{x(x^2+1)} dx = \pi (1 - e^{-1})
\end{equation*}
due to our choice of $f(z)$. This integrand of the above integral is even, so we can conclude that
\begin{equation*}
  \int_{0}^{\infty} \frac{\sin x}{x(x^2+1)} dx = \frac{\pi}{2} (1 - e^{-1}).
\end{equation*}
\end{sol}

\newpage

\begin{exer}
Consider the function
\begin{equation*}
    f(z) = \ln(z^2 -1),
\end{equation*}
made single-valued by restricting the angles in the following ways, with 
\begin{equation*}
    z_1 = z - 1 = r_{1} e^{i\theta_{1}}, \quad z_2 = z +1 = r_{2} e^{i\theta_{2}}
\end{equation*}

\begin{enumerate}[(a)]
    \item $-\frac{3\pi}{2}< \theta_{1}\leq \frac{\pi}{2}, \quad -\frac{3\pi}{2}<\theta_{2}\leq \frac{\pi}{2}$
    \item $0< \theta_{1}\leq 2\pi, \quad 0< \theta_{2}\leq 2\pi$
    \item $-\pi < \theta_{1} \leq \pi,\quad 0<\theta_{2} \leq 2\pi$
\end{enumerate}

Find where the branch cuts are for each case by locating where the function is discontinuous. Use the AB tests and show your results.
\end{exer}
\begin{sol}
    We begin by simplifying $f(z)$ in terms of $z_1$ and $z_2$, so that
    \begin{align*}
        f(z) &= \ln(z^2  -1) = \ln(z-1) + \ln(z+1)\\
             &= \ln(z_1) + \ln(z_2)\\
             & \ln(r_{1}r_{2}) + i(\theta_{1} + \theta_{2}).
    \end{align*}

    (a) Let's pick a point $A=iy-\epsilon$ with $y>0$ and $\epsilon$ small which is slightly to the left of a point on the positive imaginary axis.  We can then compute that
    \begin{align*}
        z_1(A) = A - 1 = -(1+\epsilon) + iy \quad z_{2}(A) = A + 1 = 1-\epsilon + iy.
    \end{align*}
    We can additionally find the angle using the arctangent, so that 
    \begin{equation}
        \theta_{1}(A) = \arctan(- \frac{y}{1+\epsilon} ) - \pi, \quad  \theta_{2}(A) = \arctan( \frac{y}{1-\epsilon} ),
    \end{equation}
    where we've added a constant to shift the angle into the proper range.
    Then for $B = iy + \epsilon$ which is slightly to the right of the positive imaginary axis. We can compute that
    \begin{align*}
        z_1(B) &= B - 1 = -(1-\epsilon) + iy \quad z_{2}(B) = B + 1 = 1+\epsilon + iy\\
        \theta_{1}(B) &= \arctan( -\frac{y}{1-\epsilon} ) - \pi, \quad  \theta_{2}(B) = \arctan( \frac{y}{1+\epsilon} ).
    \end{align*}

    In the limit as $\epsilon \to 0$, the various magnitudes $r$ converge, and we can see that 
    \begin{align*}
        \lim_{\epsilon\to 0}  \theta_{1}(A) + \theta_{2}(A) = \lim_{\epsilon \to 0} \theta_{1}(B) + \theta_{2}(B),
    \end{align*}
    so the function is continuous along the positive imaginary axis.

    \newpage

    (b) 
\paragraph{Case 1: $x\geq1$}%

    Now picking $A = x + i\epsilon$ for $x\geq 1$ and $\epsilon>0$ which is slightly above the real axis. We can compute that 
    \begin{equation*}
        z_1(A) = A - 1 = x - 1 + i\epsilon, \quad z_2(A) = A + 1 = x + 1 + i \epsilon.
    \end{equation*}
    We can compute the angles $\theta_{1}(A)$ and $\theta_{2}(A) $ as being slightly above 0 due to the side of the branch cut they are on. Similarly, for a point $B = x - i \epsilon$ for $x\geq1$ and $\epsilon>0$ which is slightly below the real axis. We see that the angle $\theta_{1}(B)$ and $\theta_{2}(B) $ are slightly below $2\pi$.   Taking the limit as $\epsilon\to 0$ from above, we have that 

    \begin{equation*}
        \lim_{\epsilon\to 0^+} \theta_{1}(A) + \theta_{2}(A) = 0 + 0 \neq 2\pi + 2\pi = \lim_{\epsilon\to 0^+} \theta_{1}(B) + \theta_{2}(B),
    \end{equation*}
    so $f(z)$ is discontinuous along the real axis where $x > 1$.

    \paragraph{Case 2: $0<x<1$}%

    Now when $A = x + i \epsilon$, we have the same $z_1(A)$, $z_2(A)$, $z_1(B)$, and $z_2(B)$ but now $z_1(A)$ and $z_1(B)$ have negative real part so that in the limit
    \begin{equation*}
        \lim_{\epsilon\to 0^+} \theta_{1}(A) + \theta_{2}(A) = \pi + 0 \neq \pi + 2\pi = \lim_{\epsilon\to 0^+} \theta_{1}(B) + \theta_{2}(B).
    \end{equation*}
    The case where $x=1$ is similar, but instead $z_1(A)$ and $z_1(B)$ have 0 real part and $\theta_1(A)$ and $\theta_1(B)$ converge to $\pi / 2$ and $3\pi / 2 $ as $z_1(A)$ and $z_1(B)$ live on the positive and negative imaginary axes respectively , so the function $f(z)$ is discontinuous along the positive real axis.

    \newpage

    (c) In this case, we have one branch cut along the negative real axis and another along the positive axis. We'll split into a couple of cases.

    Picking $A = x + i\epsilon$ for $\epsilon > 0$ and $x\in\bbR$ gives a point which is slightly above the real axis. We have that 
    \begin{equation}
        z_1(A) = A - 1 = x - 1 + i\epsilon, \quad z_2(A) = A + 1 = x + 1 + i \epsilon.
    \end{equation}
    For $B = x - i\epsilon$, we have a point slightly below the real axis, so that 

    \begin{equation*}
            z_1(B) = B - 1 = x - 1 - i\epsilon, \quad z_2(B) = B + 1 = x + 1 - i \epsilon.
    \end{equation*}

    \paragraph{Case 1: $x>1$}%

    We have that in the limit as $\epsilon\to 0$, $\theta_1(A) = 0$ since $z_1(A)$ is slightly above the positive real axis. Also, we have that in the same limit, $\theta_{2}(A) = 0$ since $z_2(A)$ is slightly above the real axis as well. For similar reasons, we have that $\theta_{1}(B) = 0 $ and $\theta_{2}(B)= 2\pi$. This shows that the function is discontinuous along $x\geq 1$

    \paragraph{Case 2: $x< -1$}%
  In this case, the four points of interest $z_1, z_2$ for $A$ and $B$ all have negative real part, so that in the limit as $\epsilon \to 0$,
   \begin{align*}
   \theta_1(A) = \pi \quad \theta_2(A) = \pi\\
   \theta_1(B) = -\pi \quad \theta_2(B) = \pi.
   \end{align*}
   This shows that the function is also discontinuous along $x\leq -1$.

   \paragraph{Case 3: $x\in (-1,1)$}
   In this case, we have that $z_1(A)$ and $z_1(B)$ have negative real part and $z_2(A)$ and $z_2(B)$ have positive real part, so that in the limit
   \begin{align*}
   \theta_1(A) = \pi \quad \theta_2(A) = 0\\
   \theta_1(B) = -\pi \quad \theta_2(B) = 2\pi.
   \end{align*} 

   This shows that the function is actually continuous on $(-1,1)$.

   \paragraph{Case 4: $x=1$}%
   
   In this case, $z_1(A)$ and $z_1(B)$ have zero real part and $z_2(A)$ and $z_2(B)$ have positive real part, so that in the limit
   \begin{align*}
   \theta_1(A) = \pi / 2 \quad \theta_2(A) = 0\\
   \theta_1(B) = -\pi  / 2\quad \theta_2(B) = 2\pi.
   \end{align*} 
   So the function is discontinuous here.

   \paragraph{Case 5: $x=-1$}%
    In this case, we have that $z_1(A)$ and $z_1(B)$ have negative real part and $z_2(A)$ and $z_2(B)$ have zero real part, so that in the limit
   \begin{align*}
   \theta_1(A) = \pi \quad \theta_2(A) = \pi / 2\\
   \theta_1(B) = -\pi \quad \theta_2(B) = 3\pi / 2.
   \end{align*} 
   So the function is discontinuous here.

   
   This shows that the function is continuous along $(-1,1)$ and that $f(z)$ is discontinuous along the real axis where $\abs{x}\geq1$.
   
\end{sol}
\end{document}
