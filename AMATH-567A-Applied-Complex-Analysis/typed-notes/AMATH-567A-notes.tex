\documentclass[12pt]{article}

%Preamble

\usepackage{amsmath}
\usepackage{amssymb}
\usepackage{amsthm}
\usepackage{amsrefs}
\usepackage{amsfonts}
%\usepackage{dsfont}
\usepackage{mathrsfs}
\usepackage{mathtools}
%\usepackage{stmaryrd}
%\usepackage[all]{xy}
\usepackage{enumerate}
\usepackage[shortlabels]{enumitem}
\usepackage{verbatim} %% includes comment environment
\usepackage{hyperref}
\usepackage[capitalize]{cleveref}
\crefformat{equation}{~(#2#1#3)}
\usepackage{caption, subcaption}
\usepackage{graphicx}
\graphicspath{{figures/}}
\usepackage{fullpage} %%smaller margins
\usepackage[all,arc]{xy}
\usepackage{mathrsfs}

%% Sectioning, Header / Footer, ToC
\usepackage{titlesec}
\usepackage{fancyhdr}
\usepackage{tocloft}

\hypersetup{
    linktoc=all,     %set to all if you want both sections and subsections linked
}

\newcommand{\bbF}{\mathbb{F}}
\newcommand{\bbN}{\mathbb{N}}
\newcommand{\bbQ}{\mathbb{Q}}
\newcommand{\bbR}{\mathbb{R}}
\newcommand{\bbZ}{\mathbb{Z}}
\newcommand{\bbC}{\mathbb{C}}

\newcommand{\abs}[1]{ \left| #1 \right| }
\newcommand{\diff}[2]{\frac{d #1}{d #2}}
\newcommand{\pdiff}[2]{\frac{\partial #1}{\partial #2}}
\newcommand{\infsum}[1]{\sum_{#1}^{\infty}}
\newcommand{\norm}[1]{ \left|\left| #1 \right|\right| }
\newcommand{\eval}[1]{ \left. #1 \right| }

\renewcommand{\phi}{\varphi}

% Declare real and imaginary operators 
\renewcommand{\Re}{\text{Re}}
\renewcommand{\Im}{\text{Im}}

%--------Theorem Environments--------
%theoremstyle{plain} --- default
\newtheorem{thm}{Theorem}[section]
\newtheorem{cor}[thm]{Corollary}
\newtheorem{prop}[thm]{Proposition}
\newtheorem{lem}[thm]{Lemma}
\newtheorem{conj}[thm]{Conjecture}
\newtheorem{quest}[thm]{Question}

\theoremstyle{definition}
\newtheorem{defn}[thm]{Definition}
\newtheorem{defns}[thm]{Definitions}
\newtheorem{con}[thm]{Construction}
\newtheorem{exmp}[thm]{Example}
\newtheorem{exmps}[thm]{Examples}
\newtheorem{notn}[thm]{Notation}
\newtheorem{notns}[thm]{Notations}
\newtheorem{addm}[thm]{Addendum}
\newtheorem{exer}[thm]{Exercise}

\theoremstyle{remark}
\newtheorem{rem}[thm]{Remark}
\newtheorem{rems}[thm]{Remarks}
\newtheorem{warn}[thm]{Warning}
\newtheorem{sch}[thm]{Scholium}

\numberwithin{equation}{section}

\bibliographystyle{plain}

%% Sectioning Aesthetics
\titleformat{\section}
{\normalfont\LARGE\bfseries}{\thesection.}{1em}{}
\titleformat{\subsection}
{\normalfont\Large\bfseries}{\thesubsection}{1em}{}
\titleformat{\subsubsection}
{\normalfont\normalsize\bfseries}{\thesubsubsection}{1em}{}
\titleformat{\paragraph}[runin]
{\normalfont\normalsize\bfseries}{\theparagraph}{1em}{}
\titleformat{\subparagraph}[runin]
{\normalfont\normalsize\bfseries}{\thesubparagraph}{1em}{}


%% Header Aesthetics
\pagestyle{fancy}

\setlength{\headheight}{16pt}
\setlength{\headsep}{0.3in}
\renewcommand{\headrulewidth}{0.4pt}
\renewcommand{\footrulewidth}{0.4pt}
\renewcommand{\contentsname}{\hfill\bfseries\Large Table of Contents\hfill}
\renewcommand{\sectionmark}[1]{\markright{ #1}}

\lhead{\textbf{}} % controls the left corner of the header
%\chead{\fancyplain{}{\rightmark }}
 % controls the center of the header / adds section # to top
\rhead[]{Marlin Figgins} % controls the right corner of the header
\lfoot{Last updated: \today} % controls the left corner of the footer
\cfoot{} % controls the center of the footer
\rfoot{Page~\thepage} % controls the right corner of the footer

\title{\bfseries\huge{AMATH 567A: Applied Complex Analysis}\vspace{-1ex}} \author{\href{marlinfiggins@gmail.com}{\Large{Marlin Figgins}}\vspace{-2ex}}
\date{\large{Oct. 1, 2020}}

\begin{document}

\maketitle

	\section*{\hfill Introduction \hfill}

  This document is a collection of my personal notes on complex analysis during my time as a student in AMATH 567A in fall quarter 2020 at the University of Washington.

  \subsection*{Logistics}%
  \label{sub:logistics}
  
  Homework is 70 \% of the the grade and is due on Wed (at midnight?). Final is worth 30 \% of the grade and will be take-home. There will be office hours with Tung on Tues. and one day to be determined. TA office hours are Mon. and Wed. from 4 to 5pm.  

  \thispagestyle{empty}

  %% Table of Contents Page/
  \newpage
  \tableofcontents
  \thispagestyle{empty}
  \newpage

  %% Set first page after ToC
  \setcounter{page}{1}


  %% Start here.

  \section{Complex Numbers}
  \label{sec:complex_numbers}

We begin our discussion of complex analysis with their motivation. Consider the polynomial equation with integer coefficients

\begin{equation}
  x^2 + 1 = 0. %\rightarrow z_* = \pm 1
\end{equation}

In the real numbers $\bbR$, this has no solution. Since $x^2 \geq 0$ for any real number $x$, we know that

\begin{equation*}
x^2 + 1 > 0 \text{ for all } x\in \bbR.
\end{equation*}

In an ideal world, we would want to have a number system in which every polynomial equation has a solution. This is where the complex numbers arise. Breaking from our traditions of real numbers, we can simply invent some other number $i$, so that
%% Why do we want the fundamental theorem of algebra?
\begin{equation*}
  i^2 = -1 \Rightarrow i = \sqrt{-1}.
\end{equation*}

We call this number $i$ an \emph{imaginary} or \emph{complex} number. As we can see, the number $i$ is clearly \textbf{not} a real number due to it having a negative square. Using $i$, we can define the set of complex numbers as follows. The complex numbers are numbers of the form 

\begin{defn}
  We define the set of complex numbers as 
\begin{equation}
  \bbC = \{ a + bi \mid a, b \in \bbR \text{ and } i = \sqrt{-1} \}.
\end{equation}
\end{defn}

Notice that under this construction, $\bbR$ is the subset of $\bbC$ such that $b=0$. In fact, we can define two binary operations \emph{complex addition} $+_\bbC$ and \emph{complex muliplication} $\cdot_\bbC$ on $\bbC$ which turn out to operate simuliarly to those in the real numbers.

\begin{defn}
  Given complex numbers $z=a+ib$ and $w=c+id$, define
\begin{align}
  z +_\bbC w=& (a+ib)+(c+id)=(a+c)+i(b+d)\\
  z\cdot_\bbC w=&= (a+ib)\cdot(c+id)=(ac-bd)+ i(ad+bc).
\end{align}
\end{defn}

\begin{rem}
  We often write $zw$ in place of $z\cdot_\bbC w$ for simplicity.
\end{rem}

% TODO: Insert figure showing how complex addition works.

You can derive these operations directly from the definition of the complex numbers yourself by using the substitution $i = \sqrt{-1}$. One can show that under these operations $\bbC$ is a field much like $\bbR$ and that in fact, real and complex addition and multiplication are exactly the same when working on $\bbR \subset \bbC$. In this way, we can think of $\bbC$ as an extension of $\bbR$. Because of this, we'll drop the subscript $_\bbC$ when dealing with complex operations and simply use the same notation as their real counterparts $+$ and $\cdot$.

\begin{prop}
  The complex numbers $\bbC$ are a field when endowed with complex addition and multiplication as above.
\end{prop}

Since complex numebers are a field, we are also able to define divison for these numbers as follows

\begin{equation}
  z %Put a formula here for dividing complex numbers.
\end{equation}

% TODO: Show how to divide complex numbers with a similar formuala.

%%%%% Ordered Field Stuff%%%%
%Though $\bbC$ is a field, it is not an ordered field. We say that $F$ is an ordered field if there exists $P\subset F$ such that $P$ is closed under addition and multiplication and exactly one holds: $x=0, x\in P, -x\in P$. This cannot be not the case for the complex numbers since $x^2\in P$ i.e. $i^2=-1\in P$ and $1^2=1\in P$.
%%%%%%%

One significant benefit of extending the real numbers to the complex numbers is that it guarantees us that any polynomial with complex coefficients of degree $n$ has exactly $n$ complex roots. This statement is called the \emph{Fundamental Theorem of Algebra} and we'll prove it for later. For now, let's return to our beginning example, but now using complex numbers. We can rely on the algebraic rules defined above to see that

\begin{equation}
  z^2 + 1 = 0 \Rightarrow z^2 = -1 \Rightarrow z = \pm i.
\end{equation}

By extending our problem to the complex numbers, we can see that $z^2 + 1$ had two roots all along. They just lived in the complex numbers, outside of the real number line.

%Complex analysis also allows us to understand questions like:

%\begin{equation}
%  \text{Why does } \frac{1}{1+z^2} = \sum_{n \in \bbN} (-1)^n z^{2n} \text{ diverge at } \abs{z} = 1?
%\end{equation}

The reality is that complex numbers also behave like vectors in nice ways. Instead of thinking of the complex number $z = x + iy$ as being a sum, we can write it in terms of coordinates as a point in the plane $(x,y)$. This allows us to define the magnitude of complex numbers as the distance from the origin $0 \in \bbC$ as

\begin{equation}
\abs{z} = \sqrt{x^2 + y^2}.
\end{equation}
This is simply the Pythagorean theorem that you may have seen in earlier mathematics classes. We can use this similarity to show that this definition of magnitude defines a distance metric on $\bbC$.

\begin{prop}
  We can use the magnitude of complex numbers to define a distance metric on $\bbC$. The distance between any two points $z$ and $w$ in $\bbC$ is given by the magnitude of their difference.
  \begin{equation}
    d(z,w) = \abs{z - w}. 
  \end{equation}

\end{prop}
Writing $z = a + ib$ and $w = c + id$, we can compute this by looking at the real and imaginary parts of $z$ and $w$ individually, so that
\begin{equation}
  \abs{z - w} = \sqrt{(a-c)^2 + (b-d)^2}.
\end{equation}

\begin{rem}
  Since $\abs{\cdot}$ is a distance metric on $\bbC$, it satisfies the triangle inequality i.e. that
  \begin{equation}
    \abs{z+w}\leq \abs{z} + \abs{w}.
  \end{equation}
\end{rem}

Notice that this is equivalent to the distance between these points if we were to consider them as $z=(a,b)$ and $w=(c,d)$ in $\bbR^2$. With this representation in mind, we'll define some more operations for dealing with geometry of the complex numbers. To start, we'll break our complex number's down into two seperate components.

\begin{defn}
Given a complex number $z = a + bi$, we define the real and imaginary parts of $z$ to be
\begin{equation}
  \Re[z] = a \text{ and  } \Im[z] = b.
\end{equation}
\end{defn}

We also define the complex conjugate of $z$.
\begin{defn}
For any complex number $z = a + bi$, we define the \emph{complex conjugate} of $z$ to be 
\begin{equation}
  \bar{z} = a - bi
\end{equation}
\end{defn}

The complex conjugate is extremely useful in discussion of the geometry of the complex plane. One direct consequence of our definition of it is that complex conjugate of a product of two complex numbers is the product of the complex conjugates of the individual numbers.

\begin{prop}
  For any complex numbers $z$ and $w$, $\bar{zw} = \bar{z}\bar{w}$
\end{prop}

With a quick computation, we can see that magnitude of a complex number can also be found using its complex conjugate.
\begin{equation}
  \abs{z} = \sqrt{z\bar{z}}. 
\end{equation}

The complex conjugate also gives us an easy formula for computing the multiplicative inverse of any non-zero complex number.

\begin{prop}
  If $z\neq 0$, then $z\cdot \left(\frac{\bar{z}}{\left|z\right|^2}\right)=1$. Therefore, we can compute the multplicative inverse of $z$ as

  \begin{equation}
    z^{-1}  = \frac{\bar{z}}{\left|z\right|^2}. 
  \end{equation}
\end{prop}

% TODO: Show how to take inverse of complex numbers with formulas like those for addition, mult, and dicidusion 

\subsection{Polar Coordinates}%
\label{sub:polar_coordinates}
% TODO: Clean up maybe move second half to elementary functions

As we've seen, we can represent the complex numbers using Cartesian coordinates as ordered pairs in $\bbR^2$. We can also represent a complex number $z\in\bbC$ with polar coordinates.

\begin{equation}
  z = x + iy = r(\cos\theta + i\sin\theta)
\end{equation}

Euler was able to prove that the right had side of this equation was equivalent to

\begin{equation}
  e^{i\theta} = \cos\theta + i\sin\theta.
\end{equation}

This allows us to define $\sin$ and $\cos$ in terms of the complex exponential as

\begin{align}
  \sin\theta = \frac{1}{2}(e^{i\theta} - e^{-i\theta}),\\
  \cos\theta = \frac{1}{2}(e^{i\theta} + e^{-i\theta}).
\end{align}

\begin{proof}[Proof of Euler's formula]\leavevmode

  \noindent Let $f(\theta) = \cos\theta + i\sin\theta$. Since both $\sin$ and $\cos$ satisfy the second order equation,
  \begin{equation}
    \frac{d^2}{d\theta^2}g + g = 0
  \end{equation}
  then we know that $f$ does as well with initial conditions $f(0) = 1$ and $f'(0) = i$. The exponential function should have the property that
  \begin{equation}
    \frac{d}{d\theta} e^{\lambda \theta} = \lambda e^{\lambda \theta}.
  \end{equation}
  By taking another derivative, we can see that 
  \begin{equation} 
    \frac{d^2}{d\theta^2} e^{\lambda \theta} + e^{\lambda \theta} = (\lambda^2 + 1)(e^{\lambda \theta}).
  \end{equation}
  We see that for $\lambda = i$, this function satisfies the same differential equation as $f$. Likewise, $e^{\lambda 0} = 1$ and $\frac{d}{d\theta} e^{\lambda \theta} \mid_{\theta = 0} = \lambda$. Therefore, by the uniqueness of this differential equation, we have
  \begin{equation}  
  e^{i\theta} = \cos\theta + i\sin\theta.
  \end{equation}
\end{proof}

\section{Elementary Functions}%
\label{sec:elementary_functions}

\subsection{Complex Exponental}%
\label{sub:complex_exponental}


\begin{defn}[Complex exponential]
  Define 
  \begin{equation}
    e^z = 1 + z + \frac{z^2}{2!} + \cdots = \sum_{i=1}^{\infty}\frac{z^n}{n!}
  \end{equation}
\end{defn}

\begin{prop}[Properties of the complex exponential]
  %TODO: Fill out properties of complex exponential

The complex exponential has the following properties:
\begin{enumerate}[(i)]
  \item First
  \item Second %TODO: The complex exponentional is 2\pi periodic
\end{enumerate}
\end{prop}

%TODO: Define the complex power series for each of the $\sin $ and $\cos$. Define hyperbolic sine and cosine.

\subsection{Complex logarithm}%
\label{sub:the_logarithm}

Can the familiar logarithm function in the real numbers also be generalized easily in the complex numbers? Let's start from the definition of the logarithm. Suppose $w=f(z)$ is a function such that $z = e^w$ with $e$ defined as above. With $z = re^{i\theta}$ and $w = u + iv$. We see that 

\begin{equation*}
  r = e^u \text{ and } \theta = v.
\end{equation*}

Therefore, $u = \ln r$ is real and singular-valued, but $\theta$ can only be determined up to a shift by $2n\pi$ since the complex exponential $e^{i\theta}$ is equivalent for 
\begin{equation}
  \theta = \theta_p + 2n\pi \text{ with } -\pi < \theta_p \leq \pi.
\end{equation}
%%TODO: Prehaps, just say that this is due to the complex exponential being periodic. Mention that we pick the princple logarithm which is...
We can somewhat solve this issue by restricting $\theta$ to the interval $[0,2\pi]$ or $[-\pi, \pi]$. 

\subsection{Analytic functions}%
\label{sub:analytic_functions}

%TODO: Why do we discuss differentiability?

\begin{defn}
  A function $f$ of a complex variable $z$ is \emph{analytic} at a point $z_0$ if it is single-valued and its derivative exists not only at $z_0$ but at each point $z$ in some neighborhood of $z_0$. We call the function analytic in a region $R$ if it is analytic at every point in $R$.
\end{defn}
This definition is a bit terse. Let's explore what this means through a couple of examples and non-examples of analytic functions.

%TODO: Review the limit definition of the derivative. If you remember from calculus, we define the derivative of a function of a real variable essentially the same way.

%TODO: Talk about the derivative existance and path independence. For the derivative to exist in a consistent way, we need it to be independent of the path taken to it.

%TODO: Tease that you can differentiate an analytic function infinitely often.

\begin{exmp}[Differentiating $z^2$]
  As an example, let's differentiation the function $f(z)=z^2$.

  \begin{align}
    \diff{}{z}z^2 = ... = 2z
  \end{align}
\end{exmp}

\begin{exmp}[Non-differentiability of $\bar{z}$]
  Consider now $f(z)=\bar{z}$.

  %TODO: Show this is not differentiable anywhere because the path you take matters
\end{exmp}

The reality is that functions like $f(z)=\abs{z}^2=z\bar{z}$ are really functions of two variables $z$ and $\bar{z}$ since 
\begin{equation}
  x = \frac{1}{2}(z + \bar{z}) \text{ and } y = \frac{1}{2}(z - \bar{z}).
\end{equation}

% The bulk of the theory is the theory of complex variables is to define the conditions under which $f(z)$ is a function of $z$ only (is analytic). 

\paragraph{Informal perspective on path independence.}
Consider a complex function of two variables $f(z, \bar{z})$. Supposing we expand $f$ in a Taylor series, we see that 
\begin{equation}
  f(z, \bar{z}) - f(z_0, \bar{z}_0) = a(z-z_0) + b(\bar{z}-\bar{z}_0) + \text{higher order terms.}
\end{equation}

Taking the derivative allows the higher order terms to fall out giving:
\begin{equation}
  \lim\limits_{z\to z_0} \frac{f(z, z_0) - f(z_0, \bar{z}_0)}{z - z_0} = a + b \frac{\bar{z}-\bar{z}_0}{z-z_0}.
\end{equation}

%TODO: Why does this mean that $\frac{df}{dz^*} = 0$?

%TODO: Rewatch lecture 3 to finish this.

%%TODO: Consider making multivalued functions a seperate subsetion
\paragraph{Multivalued Functions}%
\label{par:multivalued_functions}
Our definition of analyticity required that the function in question be single-valued. As we've seen, this requirement automatically disqualifies complex functions like 
\begin{equation*}
  z^{1/2} \text{ and } \ln z
\end{equation*}
from being analytic despite the fact they are at least somewhere differentiable in the real case. These functions are nowhere analytic due to the fact they are multi-valued, but we can make these singular valued by restricting the argument of $z$ to an inteval of length $2\pi$ e.g. we might define that $\arg z \in (0, 2\pi]$. This restriction creates a line of discontinuity, so it is certainly not differentiable along that line. In hopes to find a region where it is analytic we must first cut away. 

\begin{exmp}[Re-defining $z^{1/2}$]
  Let's redefine a single-valued $z^{1/2}$ by 
  \begin{equation}
    f(z) = z^{1/2}, \quad 0 \leq \arg z < 2\pi.
  \end{equation}

  Consider points $A = re^{i0}$ and $B = re^{i2\pi}$. Then wihtout restricting our argument, we would have
\begin{equation}
  f(A) = r^{1/2} \text{ and } f(B) = (re^{i2\pi})^{1/2}= r^{1/2}e^{i\pi} = -r^{1/2}. 
\end{equation}
%TODO: Rewatch lecture 3 to finish this.

Therefore, without restricting our argument we would run into issues rather quickly. Due to our actual definition which includes $A$ in its domain and not $B$, we would observe that $f(A) = r^{1/2}$. Regardless of our choice of whether to include $A$ or $B$, this introduces a line of discontuity along the positive real axis. This is called the \emph{branch cut}.
\end{exmp}

\paragraph{Branch cuts} The above example shows that a branch cut goes from the origin to the point at infinity $z_\infty = \lim_{z\to 0}\frac{1}{z}$. We call these two points the \emph{branch points}.
%TODO: Add visualize aid here: https://flothesof.github.io/branch-cuts-with-square-roots.html 

%TODO: Make sure my reader can understand why throughout. Never define something completely unmotivated. There is a reason behind each decision we make.

\subsection{Cauchy-Riemann equations.}%
\label{sub:cauchy_riemann_equations}

Here is the first interesting conseequence of our requirement that the derivative of $f$ be path-independent. As we showed before, if $f$ is analytic at $z$, then
\begin{equation}
  f'(z) = \lim_{\Delta z \to 0} = \frac{f(z + \Delta z) - f(z)}{\Delta z}.
\end{equation}

We can take the derivative along two seperate directions. First, along the real axis.


Now along the imaginary axis.

%TODO: Finish this to derive the cauchy riemann equations.

This gives us the Cauchy-Riemann equations

\begin{equation}
  \pdiff{u}{x} =\pdiff{v}{y} \quad\quad \pdiff{v}{x} = - \pdiff{v}{x}.
\end{equation}

\begin{exmp}
Let's consider an example
\begin{equation}
  f(z) = z \cdot z^* = x^2 + y^2
\end{equation}
\end{exmp}
%TODO: Show that CR is only satisified at a point. Therefore, the function is nowhere analytic.

%TODO: Other example from notes / lecture 4.

%TODO: Application: Laplace equation

\paragraph{Laplace's Equation.}%
\label{par:laplaces_equation}

%TODO: Give problem background from lecture 5.
One application of this is Laplace's equation

\begin{equation}
  \nabla^2 \phi = 0,
\end{equation}

where $\nabla^2 = \frac{\partial^2}{\partial x^2} + \frac{\partial^2}{\partial y^2}$. The solution is a harmonic function.

\begin{thm}
  The real and imaginary parts of an analytic function are both harmonic functions.
\end{thm}

\begin{proof}
  Since $f$ is analytic, it satisifies the Cauchy-Riemann equations. Therefore,

  %TODO: Finish derivation by differentiating the Cauchy-Riemann equations.
\end{proof}

%TODO: If I have energy, equipotential curves / surfaces. Define level curves
%TODO: Level curves of real and imaginary parts are orthogonal.
%TODO: Show that \nabla u \cdot nabla v = 0 

\begin{exmp}[Level curves of $f(z)=z^2$]
We'll now plot the level curves of the function
  \begin{equation}
    f(z) = z^2 = (x^2 - y^2) + i 2xy.
  \end{equation}

  %TODO: Plot level curves for various $c = u(x,y) = v(x,y)$ in julia.
\end{exmp}

%TODO: An example of computing the harmonic conjugate of a function. 

\section{Complex Integration}%
\label{sec:complex_integration}

%TODO: Integral in the complex plane along a contour $C$ is the same as the real integral.
%TODO: Contour is piecewise smooth and simple

We can define this with the Riemann sum so that
\begin{equation}
  \int_C f(z)dz = \lim\limits_{L \to 0} \sum_{j=1}^{n} f(\zeta_j)\Delta z_j,
\end{equation}

where $\zeta_j = \frac{1}{2}(z_j + z_{j-1})$, $\Delta z_j = z_{j} - z_{j-1}$, and $L = \max_{j}\abs{\Delta z_j}$. When it comes to actually evaluating this integral, we typically write it in terms of its real and imaginary parts, so that $f(z) = u(x,y) + iv(x,y)$. Then

\begin{equation}
  \int_C f(z)dz = \int_C (u dx -v dy) + i\int_C (u dy + vdx).
\end{equation}
Here, we see that integrals in the complex plane can be reduced to real integrals in the $x-y$ plane. It's our hope that these integrals are path independent to simplify these 2D real integrals.

\paragraph{Cauchy's  Theorem}%
\label{par:cauchys_theorem}

%TODO: Double Check name
\begin{thm}
  If $f(z)$ is analytic inside and on a simple closed curve $C$, then 
  \begin{equation}
    \oint_C f(z) dz = 0.
  \end{equation}
\end{thm}

We can prove this statment using Green's theorem in real variables

\begin{thm}[Green's Theorem]
\begin{equation}
  \oint_C (V_1(x,y)dx + V_2(x,y)dy) = \iint_S \left(\pdiff{}{x}V_2 - \pdiff{}{y}V_1\right)dxdy
\end{equation}
\end{thm}


%TODO: Fill this out and put in proof environment

Applying this to prove Cauchy's  Theroem, we see that 

%TODO: Double check these equations and conclude the result.
\begin{align}
  \oint_C (u dx - vdy) = - \iint_S \left( \pdiff{v}{x} + \pdiff{u}{y}\right) dxdy \\
  \oint_C (v dx - udy) = - \iint_S \left( \pdiff{u}{x} + \pdiff{v}{y}\right) dxdy 
\end{align}

\paragraph{Path Independence}%
\label{par:path_independence}

\begin{prop}
For an analytic function $f$, we have
\begin{equation}
  \int_{z_0}^{z} f(\xi) d\xi = F(z) - F(z_0),
\end{equation}
if $f$ is analytic for all paths $C$ from $z_0$ to $z$ which be continuously deformed without leaving the region of analyticity of $f$. 
%TODO: Rephrase this 
\end{prop}
\begin{proof}
  Using Cauchy's Theorem and taking difference between integrals on curves $C_1$ and $C_2$.
  %TODO: Finish this
\end{proof}

%TODO: Why does this matter? Allows you to choose the contour you want!

%TODO: Path indep example

%TODO: Put into context and expand
\begin{thm}
The integral of an analytic function is also an analytic function of its upper limit. That is, if 
\begin{equation}
  F(z) = \int_{z_0}^z f(\xi)d\xi,
\end{equation}
for some analytic function $f$, then $F$ is analytic. Moreover, the function $F(z)$ is the anti-derivative of $f(z)$ i.e.
\begin{equation}
  F'(z) = f(z).
\end{equation}
\end{thm}

%TODO: Proof. 

%TODO: Add prop 5.4 from undergrad notes with definition of piecewise continuous differentiable curves.

%TODO 5.7: If derivative 0, then f is constant

\paragraph{Several important integrals}%
\label{par:several_important_integrals}

%TODO: Int of z^n = 0, n\in\bbN for closed contour

%TODO: Int of z^{-n} = 0, n > 1 for any closed contour not passing through zero
%TODO: Proof of this: Two cases.
%TODO: 1/z = 0 or 2pi i if contains origin or not. Can borrotw from old notes probably 

%TODO: Our goal is to write the result that a function can be written as a Taylor-Laurent sereies -> a_{-1} 2\pi i...

\paragraph{Cauchy Integral Formula}%
\label{par:cauchy_integral_formula}


Let $f(z)$ be analytic, then the function $\frac{f(z)}{z-\xi}$ is analytic except at $z=\xi$. Given a contour $C$ containing $\xi$ and the circle $C_1$ around $\xi$, we have that
\begin{equation}
  \oint_{C_1} = \oint_{C}.
\end{equation}
Since $C_1$ is a circle, we can write $z - \xi = \epsilon e^{i\theta}$, so that
\begin{equation}
  \oint_{C_1} \frac{f(z)}{z-\xi}dz = \ldots
\end{equation}
%TODO: Finish this proof. Possibly combine with 5.15 from UG

This gives us Cauchy's Integral formula
\begin{equation}
  f(\xi) = \frac{1}{2\pi i} \oint_C \frac{f(z)}{z-\xi} dz
\end{equation}
if $\xi$ is inside $C$. This tells us that if an analytic function is known on a curve $C$, then it is completely determined inside $C$.
%TODO: Add details from lecture 7

%TODO: Derivative of an analytic function using CIT

\begin{thm}
If  function $f$ is analytic at $z_0$, then its derivatives are also analytic at $z_0$ and are given by
\begin{equation}
  f^{(n)}(z_0) = \frac{n!}{2\pi i} \int_{C} \frac{f(z_0)}{(z - z_0)^{n+1}}dz
\end{equation}
\end{thm}
%TODO: Double check this!

This tells us that analytic functions are infintely diffferentiable. We also are able to derive a bound for the derivative as

\begin{equation}
  \abs{ f^{(n)}(z_0) } \leq n!\frac{M}{R^n}
\end{equation}

%TODO: Port over notes and give context for circle C_R used in above equation.
%TODO: Prove above 

\paragraph{Liouville Theorem}%
\label{par:liouville_theorem}
%TODO: Use the above show that if bounded then constant i.e. bound f'

\section{Series}%
\label{sec:series}
%TODO: Port notes on series and covergence over!
\subsection{Series fundamentals}%
\label{sub:series_fundamentals}

\subsection{Taylor Series}%
\label{sub:taylor_series}


%TODO: Cauchy-Taylor Thm
\paragraph{Cauchy-Taylor Theorem}%
\label{par:cauchy_taylor_theorem}


\begin{thm}
  If $f(z)$ is analytic throughout the circular disk $\abs{z-z_0}\leq R$, it can be expanded in a series about the point $z=z_0$ as
  \begin{equation}
    f(z) = \sum_{n=0}^\infty A_n(z-z_0)^n,
  \end{equation}
  where 
  \begin{equation}
  A_n = \frac{f^{(n)}(z_0)}{n!} = \frac{1}{2\pi i} \oint_C \frac{f(\xi)}{(\xi - z_0)^n} d\xi.
  \end{equation}
  Here the radius of convergence is the radious of the largest circle on which and within within $f(z)$ is analytic.
\end{thm}
%TODO: Rephrase as f is equibalent to its power series on the disk if it is analytic 
%TODO: Finish proof
\begin{proof}
  Take $C$ to be the largest circle centered at $z_0$ on which and inside of which $f(\xi)$ is analytic. Then

  \begin{align}
    \frac{1}{\xi - z} &= \frac{1}{(\xi - z_0) - (z - z_0)} \\
                      &= \frac{1}{\xi - z_0} \frac{1}{1-r}\\
                      &= \frac{1}{\xi - z_0}\sum_{n=0}^\infty r^n, \quad r = \frac{z - z_0}{\xi - z_0}
  \end{align}
\end{proof}


%TODO: Finish below examples
\begin{exmp}[Radius of convergence]
\begin{equation}
  \frac{1}{1-z} = 1 + z + z^2 + \cdots.
\end{equation}
Here the radius of convergence is one because the first singularity of the function is given by $z_* = 1$. Therefore, within the circle $\abs{z}<1$, the function is analytic.
\end{exmp}

\begin{exmp}[Radius of convergence]
\begin{equation}
  \frac{1}{1+z^2} = 1 - z^2 + z^4 - \cdots.
\end{equation}
\end{exmp}

\subsection{Laurent Series}%
\label{sub:laurent_series}

%TODO: Include definition of Laurent series

%TODO: Expand on examples and how we rarely use the formula for the Laurent series directly
\begin{exmp}
  Expand $f(z) = \frac{e^z}{z}$ around $z_0 = 0$. Then

  \begin{align}
    f(z) &= \frac{1}{z} \left( 1 + z + \frac{z^1}{2!} + \cdots \right)\\
         &= \frac{1}{z} + 1 + \frac{z}{2!} + \cdots.
  \end{align}
  This is not the form of a Taylor series about $z_0 = 0$, but it is the form of a Laurent series.
\end{exmp}

%TODO: Another example. Expand $f(z) = 1/1-z$
\begin{exmp}
  Expand $f(z) = \frac{1}{1-z}$. There are two cases.

  % TODO: If inside unit circle

  %TODO: Outside unit circle

  %Finish derivation
  \begin{align}
    f(z) = \frac{1}{1-z} &= -\frac{1}{z} \left( \frac{1}{1 - \frac{1}{z}} \right) \\
                         &= - \frac{1}{z} \sum_{n=0}^\infty \left( \frac{1}{z} \right)^n
  \end{align}
\end{exmp}

%TODO: Another example 
\begin{exmp}
  Expand $f(z) = \frac{1}{(z-1)(z-2)} = \frac{1}{z-2} - \frac{1}{z-1}$.
%TODO: The three cases abs{z}<1, \abs{z} < 2, 1< \abs{z} <2
  
  %TODO: Definition of poles and pole order
\end{exmp}
\end{document}
